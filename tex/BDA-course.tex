\documentclass[a4paper,12pt,openany]{article} 


\usepackage[table]{xcolor}
\usepackage[final]{pdfpages} 

\usepackage{times} 
% this is for the margin for binding in the case of two side printing
\usepackage[hmarginratio=2:2]{geometry}

\usepackage{amsmath, amssymb, latexsym}
\usepackage{amsfonts}
\usepackage{amsthm}
\usepackage{mathtools}
%\usepackage{subfigure}
%\captionsetup[subtable]{position=bottom}
%\usepackage[caption=false,font=footnotesize]{subfig}
\usepackage[utf8]{inputenc}
\usepackage{epsfig}
\usepackage{pdfpages}
\usepackage{booktabs}
\usepackage{float}
\usepackage{wrapfig}
\usepackage{colonequals}
\usepackage{tikz}
\usepackage{xspace}
\usepackage{paralist}
\usetikzlibrary{arrows.meta,decorations.pathmorphing,positioning,fit,trees,shapes,shadows,automata,calc,decorations.markings,patterns} 
\tikzset{>=latex}


\usepackage{protocolj,arydshln}

\usepackage[ruled,vlined,linesnumbered]{algorithm2e}

\setlength{\emergencystretch}{3em} % prevent overfull lines
\providecommand{\tightlist}{%
	\setlength{\itemsep}{0pt}\setlength{\parskip}{0pt}}\setlength{\emergencystretch}{3em} % prevent overfull lines
\providecommand{\tightlist}{%
\setlength{\itemsep}{0pt}\setlength{\parskip}{0pt}}

\usepackage{longtable,array}
\usepackage{calc} % for calculating minipage widths
% Correct order of tables after \paragraph or \subparagraph
\usepackage{etoolbox}
\patchcmd\longtable{\par}{\if@noskipsec\mbox{}\fi\par}{}{}


\usepackage[hyphens]{url}
\usepackage{multirow}
%\usepackage[center, font={small,it}]{caption}
\usepackage[font={small},skip=3pt,belowskip=-7pt]{caption}

\usepackage{subcaption}
\captionsetup{compatibility=false}
\usepackage[bookmarks=true]{hyperref}

\usepackage{rotating}   
\usepackage{makecell}
\usepackage{colortbl}

\usepackage{emptypage}
\usepackage{diagbox}

\usepackage{subfloat}

\usepackage{pifont}

\usepackage{hyperref} % draft for non-clickable links
\usepackage[nameinlink,capitalize]{cleveref}
%\usepackage{graphicx}

\textwidth          150mm
\textheight         225mm
%\oddsidemargin      +6mm
%\evensidemargin     +6mm
%\oddsidemargin      +10mm
%\evensidemargin     +10mm
%\topmargin          -15mm
\topmargin          -5mm

%\usepackage{algorithm}
\usepackage{algpseudocode}

\SetKwInput{KwInput}{Input}                % Set the Input
\SetKwInput{KwOutput}{Output}              % set the Output

\newcommand{\diff}[1]{\textcolor{red}{#1}}

\newcommand{\myparagraph}[1]{\vspace{0.1cm}\noindent{\bf #1.}}

\newcolumntype{P}[1]{>{\centering\arraybackslash}p{#1}}
\newcommand{\Z}{\ensuremath{Z}}
\newcommand{\Zt}{\ensuremath{\Z_t}}
\newcommand{\Zp}{\ensuremath{\Z_p}}
\newcommand{\Zq}{\ensuremath{\Z_q}}
\newcommand{\ZN}{\ensuremath{\Z_N}}
\newcommand{\Zps}{\ensuremath{\Z_p^*}}
\newcommand{\ZNs}{\ensuremath{\Z_N^*}}
\newcommand{\calL}{\ensuremath{\mathcal{L}}}
\newcommand{\calR}{\ensuremath{\mathcal{R}}}
\newcommand{\Prove}{\ensuremath{\mathsf{Prove}}}
\newcommand{\Vrfy}{\ensuremath{\mathsf{Vrfy}}}
\newcommand{\Sim}{\ensuremath{\mathsf{Sim}}}
\newcommand{\Ext}{\ensuremath{\mathsf{Ext}}}
\newcommand{\stmt}{\ensuremath{\mathsf{stmt}}}
\newcommand{\wits}{\ensuremath{\mathsf{wits}}}
\newcommand{\resp}{\ensuremath{\mathsf{resp}}}
\newcommand{\chlg}{\ensuremath{\mathsf{chlg}}}
\newcommand{\mask}{\ensuremath{\mathsf{mask}}}

\newcommand{\Hzk}{\ensuremath{\mathsf{H}_{\mathsf{zk}}}}

\iffalse

	\usepackage[
	automake,
	%acronym, % Create "acronym" glossary, so it can be printed separately
	nogroupskip, % The vertical gaps are caused by a change in letter group. Use nogroupskip to omit them
	nonumberlist, % The number list can be suppressed using the nonumberlist option.
	]{glossaries-extra}
	
	% ############################ Glossaries #####################################
	% Has to be outside preamble to work with latexmkrc
	%\usepackage{makeidx}dbb0a64436904c97afe3e4b28e4f149c4bf93a91
	\usepackage[
	%acronym, % Create "acronym" glossary, so it can be printed separately
	nogroupskip, % The vertical gaps are caused by a change in letter group. Use nogroupskip to omit them
	nonumberlist, % The number list can be suppressed using the nonumberlist option.
	]{glossaries-extra}
	
	\makeindex{}
	\makeglossaries{}
	
	% https://tex.stackexchange.com/questions/437345/acronyms-will-only-display-short-version-in-main-text-even-the-first-time-they
	% First time acronym is used, use long version, afterwards short
	\setabbreviationstyle[acronym]{long-short}
	
	\loadglsentries{glossaries}
	
	% Automatically index the location of entry in the glossary
	% for those entries that are in the "general" category:
	% This indexes every glossary entry by name
	\glssetcategoryattribute{general}{dualindex}{true}
	
	% Emphasize only first use of glossary component
	% https://www.dickimaw-books.com/gallery/index.php?label=mixed-glossary-emph2
	% \renewcommand{\glsxtrregularfont}[1]{\glsxtrifwasfirstuse{\emph{#1}}{#1}}
	\renewcommand{\glsxtrregularfont}[1]{\glsxtrifwasfirstuse{\textit{#1}}{#1}}
	% \renewcommand{\glsxtrabbreviationfont}[1]{\glsxtrifwasfirstuse{\emph{#1}}{#1}}
	%\renewcommand{\glsxtrabbreviationfont}[1]{\glsxtrifwasfirstuse{\textit{#1}}{#1}}
	
	% ########################## End Glossaries ###################################
\fi

%%%% -----------------------------------------------------------------------
%%%% Ryzi Nashovo ekvilibrium
\newcommand{\pnes}{s^{*}}

\newcommand{\gls}[1]{#1}

%%%% -----------------------------------------------------------------------
%%%% Hrac honest a malicious
\newcommand{\phon}{P_{hon}}
\newcommand{\pmal}{P_{grd}}
\newcommand{\halfhalf}{(\frac{1}{2}, \frac{1}{2})}
\newcommand{\bracks}[2]{\left( #1, #2 \right)}

\newtheorem{claim}{Claim}
\newtheorem{definition}{Definition}
\newtheorem{corollary}{Corollary}

\newtheorem{lemma}{Lemma}

\definecolor{ao(english)}{rgb}{0.0, 0.5, 0.0}

\newcommand{\varDash}[1]{{\operatorname{\mathit{#1}}}}

\newtheorem{theorem1}{Security Claim}
\newenvironment{proof1} {\begin{proof}[Justification]} {\end{proof}}


\DeclareUnicodeCharacter{00A0}{ }

\newcommand{\pluseq}{\mathrel{{+}{=}}}
\newcommand{\plusplus}[1]{\mathrel{{#1}{+}{+}}}


%%%% -----------------------------------------------------------------------
%%%% Genericka tabulka zisku v bazove hre
\newcommand{\bazka}[4]{
	\begin{tabular}[t]{c|c|c}
		$\phon$/$\pmal$ & \gls{honest} & \gls{greedy} \\
		\hline
		\gls{honest} & (#1,#1) & (#2,#3) \\
		\hline
		\gls{greedy} & (#3,#2) & (#4,#4) \\
\end{tabular}}


\newtheorem{theorem}{Theorem}
\newtheorem{hypothesis}{Hypothesis}


\newcommand{\goes}[1]{\xrightarrow{#1}}

\newcommand{\reducetool}[0]{\textsc{Reduce}}

\newcommand{\specialcell}[2][c]{%
	\begin{tabular}[#1]{@{}l@{}}#2\end{tabular}}

\usepackage[normalem]{ulem}
\newcommand{\cmark}{\textcolor{black}{\ding{51}}}
\newcommand{\xmark}{\textcolor{black}{\ding{55}}}
\newcommand{\trot}[1]{\multicolumn{1}{l}{\rlap{\rotatebox{25}{#1}~}}}


\algrenewcommand\algorithmiccomment[1]{\hfill \textcolor{gray}{$\triangleright$ \textit{#1}}}


\algnewcommand{\IIf}[1]{\State\algorithmicif\ #1\ \algorithmicthen}
\algnewcommand{\EElse}[1]{\State\algorithmicelse\ #1\ }
\algnewcommand{\EndIIf}{\unskip}
\algrenewcommand\algorithmicindent{1.2em}%

\newcommand*{\dittoclosing}{---''---}


\newcommand{\familymc}{\ensuremath{\mathfrak{D}}}
\newcommand{\init}{\ensuremath{s_0}}
\newcommand{\semantics}[1]{[\![\, #1 \, ]\!]}
%\newcommand{\nils}[1]{\color{blue}Nils: #1\color{black}\xspace}
\newcommand{\pathset}{\mathsf{Paths}}
\newcommand{\paths}[2]{\pathset(#1,\allowbreak\,#2)}
\newcommand{\pathsfin}{\pathset_{\mathit{fin}}}
\newcommand{\pathsinf}{\pathset_{\mathit{inf}}}
\newcommand{\act}{\ensuremath{a}}
\newcommand{\Act}{\ensuremath{\mathit{Act}}}
\newcommand{\last}[1]{\mathrm{last}(#1)}
\DeclareMathOperator{\supp}{supp}
\DeclareMathOperator{\successors}{succ}
\newcommand{\Distr}{\mathit{Distr}}
\newcommand{\subDistr}{\mathit{subDistr}}
\newcommand{\distDom}{X}
%\newcommand{\distDomFin}{\DistDom=\{x_1,\, \ldots,\, x_n}
\newcommand{\distFunc}{\mu}
\newcommand{\distDomElem}{x}
\newcommand{\true}{\texttt{true}\xspace}
\newcommand{\false}{\texttt{false}\xspace}
\newcommand{\probOneG}{\mathit{p1G}}
\newcommand{\probPosG}{\mathit{pPG}}
\newcommand{\quotientstates}{\ensuremath{S^\mathfrak{D}_\sim}}
\newcommand{\abs}{\text{abs}}
\newcommand{\queue}{\ensuremath{Q}}
\newcommand{\submc}[2]{\ensuremath{{#1}\!\downarrow\!{#2}}}


\newcommand{\dtmc}{\ensuremath{D}}
\newcommand{\mdp}{\ensuremath{M}}
\newcommand{\probdtmc}{\ensuremath{P}}
\newcommand{\probfam}{\ensuremath{\mathfrak{P}}}
\newcommand{\sched}{\ensuremath{\sigma}}
\newcommand{\induced}[2]{\ensuremath{{#1}_{#2}}}
\newcommand{\Succ}{\ensuremath{\mathsf{Succ}}}

\newcommand{\Option}{\mathsf{Option}}
\newcommand{\Sketch}{\mathfrak{S}_H}
\newcommand{\ProgHoles}{\mathfrak{P}_H}
\newcommand{\program}[2]{\mathfrak{#1}_{#2}}
\newcommand{\SketchConstraints}{\ensuremath{\Gamma}}
\newcommand{\OptionPrices}{\ensuremath{\mathsf{cost}}}

\newcommand{\Assignments}[1]{\mathcal{A}_{#1}}
\newcommand{\var}{\xi}
\newcommand{\dom}{\textsl{dom}}

\newcommand{\actionof}[1]{\mathrm{act}(#1)}
\newcommand{\varHigh}{\xi}
\newcommand{\Vars}{\mathrm{Var}}
\newcommand{\modules}{\mathcal{M}}
\newcommand{\enabled}{\mathsf{enabled}}
\newcommand{\Prob}{\mathsf{Prob}}
\newcommand{\fixdeadlock}{\mathsf{fixdl}}

\newcommand{\crn}{\mathcal{N}}
\newcommand{\concMC}{\gamma(\crn)}
\newcommand{\absMC}{\alpha(\crn)}
\newcommand{\absIMC}{\alpha(\crn)}

\newcommand{\pspace}{\mathcal{P}}
\newcommand{\tspace}{\mathcal{T}}
\newcommand{\uspace}{\mathcal{U}}
\newcommand{\fspace}{\mathcal{F}}
\newcommand{\rspace}{\mathcal{R}}

\newcommand{\model}{\ensuremath{\mathcal{C}}}

\newcommand{\distr}{\pi}
\newcommand{\tmat}{\mathbf{R}}
\newcommand{\states}{S}
\newcommand{\state}{s}
\newcommand{\pmat}{\mathbf{P}}
\newcommand{\propQM}{=?}

\newcommand{\propTrue}{\mbox{\tt true}}
\newcommand{\propFalse}{\mbox{\tt false}}

\newcommand{\pathF}{\ensuremath{\phi}}
\newcommand{\stateF}{\ensuremath{\Phi}}


\newcommand{\propP}{\mbox{P}}
\newcommand{\propU}{\mbox{U}}
\newcommand{\propG}{\mbox{G}}
\newcommand{\propF}{\mbox{F}}
\newcommand{\propC}{\mbox{C}}
\newcommand{\propI}{\mbox{I}}
\newcommand{\propR}{\mbox{R}}
\newcommand{\propX}{\mbox{X}}

\newcommand{\satfun}{\ensuremath{\Lambda}}

\newcommand{\CTMC}{\ensuremath{\mathcal{C}=(S,\pi_0, \mathbf{R}, L)}}
\newcommand{\pp}{\ensuremath{\gamma}}
\newcommand{\ddistr}{\tau}

\newcommand{\Path}{\mbox{Path}}

\newcommand{\toolname}{PRISM-PSY}


\usepackage{listings}


% tikzstyles
\definecolor{prismgreen}{HTML}{009900}
\definecolor{prismred}{HTML}{cc0000}
\definecolor{prismblue}{HTML}{0000cc}



% PRISM syntax highlighting
\lstdefinelanguage{Prism}{
        basicstyle=\color{prismred}\scriptsize\ttfamily,
        literate=*	{0}{{\textcolor{prismblue}{0}}}{1}
			{1}{{\textcolor{prismblue}{1}}}{1}
			{2}{{\textcolor{prismblue}{2}}}{1}
			{3}{{\textcolor{prismblue}{3}}}{1}
			{4}{{\textcolor{prismblue}{4}}}{1}
			{5}{{\textcolor{prismblue}{5}}}{1}
			{6}{{\textcolor{prismblue}{6}}}{1}
			{7}{{\textcolor{prismblue}{7}}}{1}
			{8}{{\textcolor{prismblue}{8}}}{1}
			{9}{{\textcolor{prismblue}{9}}}{1}
			{.0}{{\textcolor{prismblue}{.0}}}{2}
			{.1}{{\textcolor{prismblue}{.1}}}{2}
			{.2}{{\textcolor{prismblue}{.2}}}{2}
			{.3}{{\textcolor{prismblue}{.3}}}{2}
			{.4}{{\textcolor{prismblue}{.4}}}{2}
			{.5}{{\textcolor{prismblue}{.5}}}{2}
			{.6}{{\textcolor{prismblue}{.6}}}{2}
			{.7}{{\textcolor{prismblue}{.7}}}{2}
			{.8}{{\textcolor{prismblue}{.8}}}{2}
			{.9}{{\textcolor{prismblue}{.9}}}{2}
			{[}{{\textcolor{black}{[}}}{1}
			{]}{{\textcolor{black}{]}}}{1}
			{+}{{\textcolor{black}{+}}}{1}
			{-}{{\textcolor{black}{-}}}{1}
			{=}{{\textcolor{black}{=}}}{1}
			{<}{{\textcolor{black}{<}}}{1}
			{>}{{\textcolor{black}{>}}}{1}
			{\&}{{\textcolor{black}{\&}}}{1}
			{|}{{\textcolor{black}{|}}}{1}
			{:}{{\textcolor{black}{:}}}{1}
			{;}{{\textcolor{black}{;}}}{1}
			{(}{{\textcolor{black}{(}}}{1}
			{)}{{\textcolor{black}{)}}}{1}
			{..}{{\textcolor{black}{..}}}{2},
        keywords= {bool,ceil,const,ctmc,double,dtmc,endinit,endmodule,endrewards, endsystem,F,false,floor,formula,G,global,I,init,int,label,max,mdp,min,module,nondeterministic,P,Pmin,Pmax,prob,probabilistic,rate,rewards,Rmin,Rmax,S,stochastic,system,true,U, option, either, assignment, relation, operation, hole, variable},
        keywordstyle={\bfseries\color{black}},
        numberstyle=\footnotesize\color{black},
        comment=[l] {//}, morecomment=[s]{/*}{*/},
        commentstyle= \color{prismgreen},
        tabsize=4,
        captionpos=b,
        escapechar=^,
        moredelim=[is][\color{orange}]{@}{@},
}


\usepackage{nicefrac}


\newcommand{\ih}[1]{\textcolor{blue}{\footnotesize [IH: #1]}}

\hyphenation{block-chain block-chains Smart-OTPs free-ness Strong-Chain gree-dy}

\newcommand{\name}{CHANGE-ME\xspace}%


\begin{document} 

\thispagestyle{empty}

\includegraphics[width=150mm]{figs/FIT_color_CMYK_EN.pdf}
{\fontfamily{ptm}\selectfont

%{\huge Brno University of Technology}         \\[2mm]
%{\LARGE Faculty of Information Technology}    \\[50mm]

\vspace{5cm}


\begin{center}
  {\LARGE Lecture Notes for the Course}      \\[10mm]
  {\Huge \bf Blockchains and Decentralized \\[3mm] Applications}    
  
\end{center}                                            
 
 \vfill
                 
\large doc. Ing. Ivan Homoliak, Ph.D.                                   
  \hfill  Brno 2026                                  


}

\thispagestyle{empty}

\eject

\thispagestyle{empty}

\ \\



\pagenumbering{roman}


% Licence 
\vfill
\noindent \textcopyright\ doc.\ Ing.\ Ivan Homoliak, Ph.D., 2026.\par
\vspace{1em}
\noindent This publication is licensed under the Creative Commons
Attribution 4.0 International License (CC BY 4.0).\par
\begin{figure}[h!]
	%	\vspace{-0.3cm}
		\includegraphics[width=0.25\textwidth]{./figs/by.png}
\end{figure}
\vspace{1em}
\noindent ISBN: 978-80-214-6394-3
\clearpage


%------------------------------------------------------------------------------
\section*{Abstract}
%------------------------------------------------------------------------------
The course Blockchain and Decentralized Applications (BDA) aims to acquaint students with the principles and protocols in fully decentralized (P2P) network communication. While aspects of client-server communication are important, the less traditional but emerging peer-to-peer blockchain scheme and its integration into the Internet is an alternative that allows us to achieve unique features in terms of availability, transparency, and trust. This course focuses on the technical aspects of blockchain systems, smart contracts, and decentralized applications. Students will learn how these systems are built, how to communicate with them, and how to design \& create secure decentralized applications. Students will also exercise the acquired knowledge in practice through a semestral assignment.
Students will learn advanced theoretical and practical knowledge in the field of decentralized computing platforms, their types, consensual protocols, and problems associated with them. Further, students will learn knowledge of terminology, unique properties of blockchain, knowledge of advanced integrity-preserving data structures and algorithms used in blockchains and smart contract platforms. the next goal is to present practical use cases and their potential vulnerabilities. The course also deals with the problem of scalability and anonymity and variants of their solution. In sum, students should obtain the ability to design, deploy, and manage custom decentralized applications and solutions.
%------------------------------------------------------------------------------
\section*{Keywords}
%------------------------------------------------------------------------------
Decentralized platforms, blockchains, integrity-preserving data structures, smart contracts, decentralized applications, DAPPs, consensus protocols, security threats, scalability, anonymity, and privacy.
%Security, privacy, blockchains, distributed ledgers, threat modeling, standardization, decentralized applications, consensus protocols, proof-of-work, selfish mining attacks, undercutting attacks, incentive attacks, DAG-based blockchains, cryptocurrency wallets, two-factor authentication, 2FA, electronic voting, e-voting, public bulletin board, secure logging, data provenance, interoperability, Central Bank Digital Currency, CBDC, Trusted Execution Environment, TEE, cross-chain protocol.


\clearpage

%------------------------------------------------------------------------------
\section*{Acknowledgment}
%------------------------------------------------------------------------------
First, I would like to thank my supervisor and colleague from SUTD -- Pawel Szalachowski -- for collaborating and igniting many interesting research ideas as well as explaining to me that ideas are important but cheap in contrast to realization. 
Second, I would like to thank all my co-authors, especially Sarad Venugopalan, Pieter Hartel, Daniel Reijsbergen, Federico~Matteo Ben{\v{c}}i{\'c}, Fran Casino, and Martin Hrub{\'y}. 
%
Third, I would like to thank Tomas Vojnar, Pavel Zemcik, Pavel Smrz, and Ales Smrcka for their support in funding my research at FIT BUT.
Next, I would like to thank all Ph.D./MSc/BSc students I have had the chance to work with, especially Martin Peresini, Ivana Stan{\v{c}}{\'\i}kov{\'a}, and Dominik Breitenbacher / Tomas Hladky, Jakub Handzus, Jakub Kubik, and Martin Ersek / and Rastislav Budinsk{\'y}. 
%
Also, I would like to thank Milan Ceska (and Tomas Vojnar) for inspiring me in the habilitation process and helping me to maximize cross-domain transfer learning in this process.  
Finally, I would like to thank around 20\% of anonymous reviewers (mostly from security venues) for providing useful and constructive feedback. % or reasonable arguments.




%First, I would like to thank my advisers and mentors: Lubo\v{s} Brim for starting my research and academic career, Marta Kwiatkowska for providing me the world-leading expertise in the area of probabilistic verification and Tom\'{a}\v{s} Vojnar for helping me to establish my own research group.
%Second, I would like to thank all my co-authors, especially David \v{S}afr\'{a}nek, Nicola Paoletti, Alessandro Abate, Radu Calinescu, Joost-Pieter Katoen, Ond\v{r}ej Leng\'{a}l, Luk\'{a}\v{s} Hol\'{i}k, and Jan K\v{r}et\'{i}nsk\'{y}. Third, I would like to thank all Ph.D. students I had the chance to work with especially Sven Dra\v{z}an, Luca Laurenti, Simos Gerasimou, Ji\v{r}\'{i} Maty\'{a}\v{s}, Vojta Havlena and Sebastian Junges. Last but not least, I would like to thank Gabina for supporting me during my entire research career and my parents for inspiring me to take this career. 


\vfill \emph{Over the time, I have been supported by number of projects, in particular, by EU ECSEL projects, EU HORIZON EUROPE projects, National Research Foundation -- Prime Minister's Office in Singapore, Technology Agency of the Czech Republic, and the Czech IT4Innovations Centre of Excellence project.
%Czech Science Foundation project
}

\clearpage

\newtheorem{mydef}{Definition}
\newtheorem{example}{Example}
\newtheorem{myproof}{Proof}
\newtheorem{myproposition}{Proposition}
\newtheorem{mytheorem}{Theorem}
\newtheorem{myexample}{Example}
\newtheorem{myproblem}{Problem}
\newtheorem{mycorollary}{Corollary}
\newtheorem{mymethod}{Approach}
\newtheorem{remark}{Remark}


\renewcommand{\tableautorefname}{Table}
\renewcommand{\algorithmautorefname}{Algorithm}
\renewcommand{\figureautorefname}{Figure}
\renewcommand{\equationautorefname}{Equation}
\renewcommand{\sectionautorefname}{Appendix}

\renewcommand{\chapterautorefname}{Chapter}
\renewcommand{\sectionautorefname}{Section}
\renewcommand{\subsectionautorefname}{Section}
\renewcommand{\subsubsectionautorefname}{Section}
\renewcommand{\paragraphautorefname}{ Section}
\renewcommand{\tablename}{Table}
\renewcommand{\figurename}{Figure}
%\newcommand{\subfigureautorefname}{\figureautorefname}

\newcommand{\mysubsubsection}[1]{\smallskip\subsubsection{#1}}


\tableofcontents

\newcommand{\squishlist}{
   \begin{list}{$\bullet$}
    { \setlength{\itemsep}{5pt}    \setlength{\parsep}{0pt}
      \setlength{\topsep}{5pt}     \setlength{\partopsep}{0pt}
      \setlength{\leftmargin}{1.35em} \setlength{\labelwidth}{1em}
      \setlength{\labelsep}{0.5em} } }

\newcommand{\squishend}{
    \end{list}  }


\clearpage
\pagenumbering{arabic}



%\part{COMMENTARY}
%\part{COMMENTED RESEARCH}

\section{Introduction and Required Cryptographic Constructs}\label{sec:intro}
\input{./sec/1-intro.tex}
\newpage

\section{Consensus Protocols \& Blockchains}\label{sec:consensus-protocols-blockchains}
%\subsection{Introduction}\label{in/troduction}

%This section delves into the fundamental mechanisms that underpin the
%operation of decentralized networks: consensus protocols. These
%protocols are the cornerstone of any blockchain, as they provide the
%means for a distributed network of independent and potentially
%untrusting nodes to reach an agreement on the state of the ledger. As we
%will learn, achieving this ``agreement'' or ``consensus'' in a
%decentralized environment is a non-trivial challenge, especially when
%considering the potential for network delays, failures, and even
%malicious participants.
%
%We will begin by exploring the theoretical foundations of consensus,
%including the different types of blockchain networks and how they manage
%access and participation. We will dissect the inherent trade-offs in
%distributed systems as encapsulated by the famous CAP theorem, and
%understand why blockchains must often prioritize availability and
%partition tolerance, leading to a model of ``eventual consistency.'' We
%will also define the critical properties that a robust consensus
%protocol must satisfy -- namely \textbf{safety, liveness, and finality} -- to
%ensure the network remains secure and functional.
%
%The section will then proceed to examine the two primary paradigms for
%achieving consensus: \textbf{lottery-based} and \textbf{voting-based} mechanisms. We will
%analyze the respective strengths and weaknesses of each approach,
%considering factors such as scalability, network overhead, and the time
%it takes for transactions to be irreversibly confirmed. A significant
%portion of the section will be dedicated to understanding the concept of
%\textbf{forks} -- a natural consequence of the probabilistic nature of some
%consensus protocols -- and the mechanisms used to resolve them.
%
%Furthermore, we will explore the classic \textbf{Byzantine Generals Problem},
%which provides a powerful analogy for understanding the challenges of
%achieving consensus in the presence of malicious actors. This will lead
%to a discussion of Byzantine Fault Tolerance (BFT) and its practical
%implementation in the form of the Practical Byzantine Fault Tolerance
%(PBFT) algorithm, a protocol foundational to many permissioned
%blockchains.
%
%Finally, we will examine the most prominent consensus protocols used in
%practice today. This includes Proof-of-Work (PoW), the original,
%resource-intensive mechanism of Bitcoin; Proof-of-Stake (PoS), a more
%energy-efficient alternative that has been adopted by many modern
%blockchains like Ethereum; and \textbf{Proof-of-Authority} (PoA), a
%reputation-based mechanism well-suited for private, permissioned
%networks. We will also discuss the crucial role of incentive schemes in
%securing permissionless networks and the complex challenges of managing
%decentralized names and identities.

This section aims to uncover consensus protocols that are
essential for the operation of decentralized networks. These
protocols are the foundation of any blockchain since they allow
a distributed network consisting of independent and possibly
untrusting nodes to agree on the ledger's state. As we shall see,
getting this ``agreement" or ``consensus" in a
decentralized setting is quite difficult, particularly when
there is network latency, breakdowns, and even
malicious parties.

First, we will discuss the theoretical background of consensus
including the different types of blockchain networks and their
methods of managing access and participation. We will describe
the inherent compromises in distributed systems as represented by the
well-known CAP theorem, and comprehend why blockchains have to
favor availability and partition tolerance thus leading to a
model of ``\textbf{eventual consistency.}'' We will also point out the
essential characteristics that a resilient consensus
protocol must have -- namely \textbf{safety, liveness, and finality} --
to keep the network secure and operational.
The section will go on to review the major two approaches for
attaining consensus: \textbf{lottery-based} and \textbf{voting-based}
mechanisms. We will evaluate the pros and cons of both methods
with respect to aspects such as scalability, network overhead, and
the duration for transactions to be permanently confirmed. Most of
the section will deal with explanation of the notion of
\textbf{forks} -- the direct product of the random nature of some
consensus protocols -- as well as the methods used to settle
them.
On top of that, we will be touching on the well-known
\textbf{Byzantine Generals Problem}, which is a striking metaphor for grasping the difficulties of
reaching consensus in a hostile environment. This will transition
us to the subject of Byzantine Fault Tolerance (BFT) protocols and their real-world model known as Practical Byzantine Fault Tolerance (PBFT) protocol, an algorithm that is central to many permissioned blockchains.

We will end the section by identifying the leading consensus
protocols that are extensively employed nowadays. This covers
Proof-of-Work (PoW), which is the first and Bitcoin's
original, energy-consuming, validation mechanism; Proof-of-Stake (PoS),
a more eco-friendly substitute that is in line with the
requirements of many new blockchains such as Ethereum; and
\textbf{Proof-of-Authority} (PoA), a
trust-based method that fits networks which are private and
permissioned. We will also illustrate how incentive systems
play an important part in the security of permissionless networks
and the intricate problems that come with the governance of
decentralized names and identities.


\subsection{Learning Objectives}\label{learning-objectives}

\begin{itemize}
	\tightlist
	\item
	Understand the different types of blockchain networks (permissionless,
	permissioned, and semi-permissionless) and their methods for
	controlling entry and preventing Sybil attacks.
	\item
	Grasp the concepts of the CAP theorem and its implications for
	blockchain design, particularly the trade-off between consistency and
	availability.
	\item
	Learn the standard goals of consensus protocols: safety (agreement),
	liveness (termination), and finality (eventual consistency).
	\item
	Differentiate between lottery-based and voting-based consensus
	mechanisms and their respective trade-offs.
	\item
	Understand the concept of forks (accidental, malicious, and
	intentional) and the fork-choice rules used to resolve them.
	\item
	Gain insight into the Byzantine Generals Problem and its relevance to
	fault tolerance in distributed systems.
	\item
	Learn about specific consensus protocols, including Practical
	Byzantine Fault Tolerance (PBFT), Proof-of-Work (PoW), and
	Proof-of-Stake (PoS).
	\item
	Understand the role of incentive schemes in motivating honest
	participation in blockchain networks.
	\item
	Understand the challenges of decentralized identity management as
	described by Zooko's Triangle.
\end{itemize}

\begin{center}\rule{0.5\linewidth}{0.5pt}\end{center}

\subsection{Blockchain Networks and System
	Models}\label{section-1-blockchain-networks-and-system-models}

\subsubsection{Types of Blockchains}\label{types-of-blockchains}

Blockchains can be broadly classified into three categories based on
their access control mechanisms, which determine how new nodes can join
the network and participate in the consensus process.

\begin{itemize}
	\item
	\textbf{Permissionless Blockchains}: These are open networks that
	anyone can join without requiring permission from a central authority.
	The core challenge in such a system is the \textbf{Sybil attack},
	where an attacker could create a large number of pseudonymous
	identities to gain a disproportionate influence on the network. 
%	As the 	lecturer explained, 
	If consensus were to be based on a simple vote, one
	could ``create a thousand fake identities and have a thousand votes.''
	To prevent this, permissionless blockchains employ
	\textbf{Proof-of-Resource} mechanisms. These mechanisms require nodes
	to demonstrate ownership of a scarce resource, such as computational
	power (in Proof-of-Work) or a significant financial stake (in
	Proof-of-Stake). A node's influence in the consensus process is
	directly proportional to the amount of the scarce resource it
	controls, not the number of identities it possesses. Bitcoin and
	Ethereum are prominent examples of permissionless blockchains.
	\item
	\textbf{Permissioned Blockchains}: In contrast to permissionless
	blockchains, these are closed networks where new nodes must obtain
	explicit permission to join from a centralized or federated authority.
	This authority is responsible for vetting and onboarding new
	participants. For instance, a bank operating its own blockchain would
	only allow vetted entities to join. Because participants are known and
	there's a mechanism to expel malicious actors, the risk of Sybil
	attacks is eliminated. This allows permissioned blockchains to use
	more traditional and efficient voting-based consensus protocols, such
	as Practical Byzantine Fault Tolerance (PBFT), where each node
	typically has an equal say (``one node, one vote''). These networks
	are well-suited for enterprise applications where privacy, control,
	and performance are paramount.
	\item
	\textbf{Semi-Permissionless Blockchains}: This category represents a
	hybrid approach, analogous to a joint-stock company where voting power
	is tied to ownership. While new nodes do not require permission from a
	central authority to join, they must acquire a ``stake'' in the
	network to participate in the consensus process. This stake, which is
	typically in the form of the network's native cryptocurrency, can be
	acquired from any existing participant. A node's consensus power is
	proportional to the size of its stake, which serves as a form of
	collateral that can be forfeited in the event of malicious behavior.
	This model aims to strike a balance between the openness of
	permissionless networks and the control of permissioned networks.
\end{itemize}

\subsubsection{Centralized vs.~Decentralized
	Systems}\label{centralized-vs.-decentralized-systems}

In the discourse surrounding blockchain technology, the terms
``distributed'' and ``decentralized'' are often used interchangeably,
but they refer to distinct concepts. A clear understanding of this
distinction is essential for appreciating the unique value proposition
of blockchains.

\begin{itemize}
	\item
	\textbf{Distributed Systems}: A distributed system is one in which
	components are located on different networked computers, which then
	communicate and coordinate their actions. The key characteristic of a
	distributed system is its \textbf{geographical distribution}. For
	example, a company like Google operates a massive distributed system
	with data centers located all over the world. However, despite its
	distributed nature, this system is still \textbf{centrally controlled}
	by a single entity.
	\item
	\textbf{Decentralized Systems}: A decentralized system, on the other
	hand, is defined by its \textbf{control and trust model}. In a fully
	decentralized system, there is no single point of control or trust.
	Instead, control is distributed among the participants, and decisions
	are made through a consensus mechanism. This eliminates the need for a
	trusted third party, as the system is designed to be resilient to the
	failure or malicious behavior of individual nodes.
	See various levels of decentralization in \autoref{fig:distrib-decentralized}.
\end{itemize}

While all decentralized systems are necessarily distributed, the
converse is not true. A system can be geographically distributed but
still be centrally controlled. The true innovation of blockchain
technology lies in its ability to create a fully decentralized system
that can operate in a trustless environment.

\begin{figure}[t]
	%	\vspace{-0.3cm}
	\begin{center}
		\includegraphics[width=0.8\textwidth]{./figs/distrib-decentralized.png} 
		\caption{(A) centralized system, (B) partially decentralized system, (C) fully decentralized system.}		
		\label{fig:distrib-decentralized}
	\end{center}	
\end{figure}

\subsubsection{The CAP Theorem}\label{the-cap-theorem}

The CAP theorem, also known as Brewer's theorem, is a fundamental
principle in distributed systems that describes an inherent trade-off
between three desirable properties: consistency, availability, and
partition tolerance. The theorem states that any distributed data store
can only provide two of these three guarantees simultaneously.

\begin{itemize}
	\tightlist
	\item
	\textbf{Consistency}: This guarantee ensures that all nodes in the
	network see the same data at the same time. In a consistent system,
	every read operation will return the value of the most recent write
	operation or an error.
	\item
	\textbf{Availability}: This guarantee ensures that every request to
	the system receives a response, without the guarantee that it contains
	the most recent write. In an available system, there are no single
	points of failure, and the system remains operational even if some
	nodes are down.
	\item
	\textbf{Partition Tolerance}: This guarantee ensures that the system
	continues to operate even if the network is partitioned, meaning that
	messages between nodes are dropped or delayed.
\end{itemize}

In the context of blockchain technology, which operates over the public
Internet, network partitions are an unavoidable reality. Therefore, any
blockchain protocol \textbf{must be partition tolerant}. This leaves a
fundamental choice between consistency and availability.
%
Most blockchain protocols, particularly those based on Nakamoto
consensus, prioritize \textbf{availability and partition tolerance} over
strong consistency. This means that they will continue to operate and
process transactions even in the presence of network partitions, but at
the cost of immediate consistency. Instead, they offer a weaker form of
consistency known as \textbf{eventual consistency}. This means that
while different nodes may have a different view of the ledger at any
given moment, they will eventually converge on a single, consistent
state over time. This is typically achieved by requiring a certain
number of \textbf{confirmations} (i.e., subsequent blocks) before a
transaction is considered final. For example, in Bitcoin, a transaction
is conventionally considered secure and final after six confirmations,
which takes approximately 60 minutes.

\begin{figure}[t]
	%	\vspace{-0.3cm}
	\begin{center}
		\includegraphics[width=0.8\textwidth]{./figs/btc-finality.png} 
		\caption{Eventual consistency of availability-favored blockchains within CAP theorem.}		
		\label{fig:btc-finality}
	\end{center}	
\end{figure}

\begin{center}\rule{0.5\linewidth}{0.5pt}\end{center}

\subsection{Consensus Protocols in
	Blockchains}\label{section-2-consensus-protocols-in-blockchains}

\subsubsection{Goals of Consensus
	Protocols}\label{goals-of-consensus-protocols}

Consensus protocols are the mechanisms by which a distributed network of
nodes can agree on a single, consistent history of transactions. These
protocols are designed to achieve several key goals, which are essential
for the secure and reliable operation of a blockchain.

\begin{itemize}
	\item
	\textbf{Safety (Agreement)}: This property ensures that all honest
	nodes in the network will eventually agree on the same value. In the
	context of a blockchain, this means that all honest nodes will have an
	identical copy of the ledger. Safety is a critical property, as it
	prevents a ``split-brain'' scenario where different parts of the
	network have conflicting views of the truth.
	\item
	\textbf{Liveness (Termination)}: This property ensures that the
	network will continue to make progress. In other words, all honest
	nodes will eventually decide on a value, and the blockchain does not
	stall. This ensures that new transactions are eventually included in
	the ledger.
	\item
	\textbf{Finality}: This is a concept that reinterprets safety and
	liveness in the context of blockchains that provide eventual
	consistency. A block is said to have reached finality when it is
	computationally infeasible for it to be overturned or removed from the
	blockchain (see \autoref{fig:btc-finality}). The time to finality can vary significantly between
	different consensus protocols. For example, in Bitcoin, a block is
	typically considered final after six subsequent blocks have been added
	to the chain, which takes approximately one hour. In contrast, some
	voting-based protocols can achieve finality in a matter of seconds.
\end{itemize}

\subsubsection{Lottery vs.~Voting}\label{lottery-vs.-voting}

At a high level, consensus protocols can be categorized into two main
approaches: lottery-based and voting-based~\cite{hyperledger1}.

\begin{itemize}
	\item
	\textbf{Lottery-Based Consensus}: In this approach, a single node is
	selected through a lottery-like mechanism to be the leader for a given
	round. This leader is then responsible for producing the next block
	and broadcasting it to the network. The probability of being selected
	as the leader is typically proportional to the amount of a scarce
	resource that a node controls, such as computational power in
	Proof-of-Work or stake in Proof-of-Stake. Lottery-based protocols are
	characterized by their low network overhead and high scalability, as
	they do not require all nodes to communicate with each other. However,
	they are susceptible to forks, which can occur if multiple leaders are
	elected simultaneously and propose conflicting blocks.
	\item
	\textbf{Voting-Based Consensus}: In this approach, all nodes in the
	network participate in a collective voting process to agree on the
	next block. This typically involves multiple rounds of communication,
	where nodes exchange messages to reach a consensus. Voting-based
	protocols, such as PBFT, offer the advantage of low-latency finality,
	as a block is considered final once it has been approved by a
	sufficient number of nodes. However, they suffer from high
	communication complexity, which is typically on the order of
	$O(N^2)$, where $N$ is the number of nodes. This makes them unsuitable
	for large, permissionless networks, but well-suited for smaller,
	permissioned blockchains where the number of participants is limited.
\end{itemize}

Many modern protocols, like Algorand~\cite{gilad2017algorand}, use a combination of these two
approaches, using a lottery to select a small committee of nodes which
then vote on the next block.

\subsubsection{Forks}\label{forks}

A fork is a situation where a blockchain diverges into two or more
competing chains. This can happen for a variety of reasons, and it is a
fundamental concept to understand in the context of blockchain
technology.

\begin{itemize}
	\item
	\textbf{Accidental Forks}: These are a natural consequence of the
	probabilistic nature of lottery-based consensus protocols. Due to
	network latency, it is possible for two or more nodes to solve the
	consensus puzzle at approximately the same time and broadcast their
	new blocks to the network. This can result in a temporary fork, where
	different parts of the network have a different view of the latest
	block. These forks are typically resolved quickly as subsequent blocks
	are added to one of the chains, making it the longest and therefore
	the canonical chain.
	\item
	\textbf{Malicious Forks}: These are intentionally created by attackers
	with the aim of disrupting the network or defrauding other users. A
	common example is a \textbf{double-spending attack}, where an attacker
	sends a transaction to a merchant, waits for the merchant to deliver
	the goods, and then creates a fork to reverse the transaction. Another
	example is \textbf{selfish mining}, where a miner with significant
	hash power secretly builds a longer chain and then releases it to the
	network to orphan the blocks of other miners and claim their rewards.
	\item
	\textbf{Intentional Forks (Hard Forks and Soft Forks)}: These are
	planned forks that are created to introduce changes to the protocol.
	
	\begin{itemize}
		\tightlist
		\item
		A \textbf{hard fork} is a backward-incompatible change to the
		protocol, which requires all nodes to upgrade to the new version. If
		some nodes do not upgrade, the blockchain will permanently split
		into two separate chains. The creation of Bitcoin Cash from Bitcoin,
		which increased the block size limit, is a well-known example of a
		hard fork.
		\item
		A \textbf{soft fork} is a backward-compatible change, which allows
		non-upgraded nodes to continue to participate in the network.
	\end{itemize}
\end{itemize}

Forks are typically resolved through a \textbf{fork-choice rule}, which
is a set of rules that nodes use to determine which chain to consider
the canonical one. The most common fork-choice rule is the
\textbf{longest chain rule}, which states that the valid chain is the
one with the most blocks (see \autoref{fig:longest-chain}), as it represents the most accumulated work.
Another approach is the \textbf{strongest chain rule}, which takes into
account the ``quality'' of the blocks, such as the total number of
transactions they contain.

\begin{figure}[t]
	%	\vspace{-0.3cm}
	\begin{center}
		\includegraphics[width=0.8\textwidth]{./figs/longest-chain.png} 
		\caption{Longest chain fork-choice rule.}		
		\label{fig:longest-chain}
	\end{center}	
\end{figure}

\begin{center}\rule{0.5\linewidth}{0.5pt}\end{center}

\subsection{Byzantine Fault
	Tolerance}\label{section-3-byzantine-fault-tolerance}

\subsubsection{The Byzantine Generals
	Problem}\label{the-byzantine-generals-problem}

The Byzantine Generals Problem is a classic thought experiment in
distributed computing that provides a powerful analogy for the
challenges of achieving consensus in a network where some participants
may be unreliable or malicious. The problem is typically stated as
follows (see also \autoref{fig:generals}):

\begin{itemize}
	\item \textit{A group of Byzantine generals are besieging a city. They must decide whether to attack or retreat. If all loyal generals attack, they
		will be victorious. If all loyal generals retreat, they will be saved.
		However, if some generals attack while others retreat, they will be
		defeated. The generals can only communicate with each other via
		messengers, and some of the generals may be traitors who will try to
		disrupt the plan by sending conflicting messages.}
	
\end{itemize}

\begin{figure}[t]
	%	\vspace{-0.3cm}
	\begin{center}
		\includegraphics[width=0.9\textwidth]{./figs/byzantyne-generals.png}
		\caption{Byzantine generals problem.}		
		\label{fig:generals}
	\end{center}	
\end{figure}


The problem illustrates the difficulty of achieving consensus in a
distributed system where there is no central authority and where some
nodes may be faulty or malicious. A loyal general receiving conflicting
messages -- for instance, ``attack'' from the commander but ``retreat''
from a fellow lieutenant -- cannot know who the traitor is.

It has been proven that a solution to the Byzantine Generals Problem is
only possible if the number of loyal generals is greater than two-thirds
of the total number of generals. In other words, a system can tolerate
\texttt{f} Byzantine nodes only if the total number of nodes \texttt{N}
is greater than \texttt{3f} (\texttt{f\ \textless{}\ N/3}). This means
that more than two-thirds of the nodes must be honest and follow the
protocol for the system to be able to reach a consensus. This result has
profound implications for the design of fault-tolerant distributed
systems, including blockchains.

\subsubsection{Practical Byzantine Fault Tolerance
	(PBFT)}\label{practical-byzantine-fault-tolerance-pbft}

Practical Byzantine Fault Tolerance (PBFT)~\cite{castro1999practical} is a landmark consensus
protocol designed to be resilient to Byzantine failures in synchronous and eventually synchronous
systems. Developed by Miguel Castro and Barbara Liskov in 1999, PBFT
provides a practical solution to the Byzantine Generals Problem, making
it suitable for use in real-world distributed systems.

PBFT is a voting-based protocol that operates in a series of rounds,
with each round having a designated leader. The protocol involves a
multi-step process to ensure that all honest nodes agree on the order of
operations before they are committed to the ledger. The key phases of
the PBFT protocol are (see also \autoref{fig:pbft}):

\begin{enumerate}
	\def\labelenumi{\arabic{enumi}.}
	\tightlist
	\item
	\textbf{Request}: A client sends a request to the leader node.
	\item
	\textbf{Pre-prepare}: The leader node multicasts the request to all
	other nodes in the network.
	\item
	\textbf{Prepare}: Upon receiving the pre-prepare message, each node
	verifies the request and, if it is valid, multicasts a prepare message
	to all other nodes.
	\item
	\textbf{Commit}: A node waits until it has received \texttt{2f}
	prepare messages from different nodes that match the pre-prepare
	message. At this point, the node multicasts a commit message to all
	other nodes.
	\item
	\textbf{Reply}: A node waits until it has received \texttt{2f\ +\ 1}
	commit messages from different nodes. At this point, the node executes
	the request and sends a reply to the client.
\end{enumerate}

The client waits for \texttt{f\ +\ 1} replies from different nodes with
the same result. This ensures that the result is valid, as at most
\texttt{f} nodes can be malicious.

PBFT is well-suited for permissioned blockchains where the number of
participants is relatively small, due to its $O(N^2)$ communication
complexity. Its ability to provide low-latency finality makes it an
attractive choice for enterprise applications that require high
performance and strong consistency guarantees.

\begin{figure}[t]
	%	\vspace{-0.3cm}
	\begin{center}
		\includegraphics[width=0.75\textwidth]{./figs/pbft.png}
		\caption{Practical Byzantine Fault Tolerant protocol.}		
		\label{fig:pbft}
	\end{center}	
\end{figure}

\begin{center}\rule{0.5\linewidth}{0.5pt}\end{center}

\subsection{Consensus Protocols in
	Practice}\label{section-4-consensus-protocols-in-practice}

\subsubsection{Proof-of-Work (PoW)}\label{sec:proof-of-work-pow}

Proof-of-Work (PoW) is the pioneering consensus mechanism that was first
introduced with Bitcoin~\cite{nakamoto2008bitcoin}. It is a permissionless, lottery-based protocol
that allows a decentralized network to reach a consensus without relying
on a central authority.

In a PoW system, nodes, known as miners, compete to solve a
computationally intensive puzzle. This puzzle involves finding a value,
called a nonce, which, when combined with the data in the block and
hashed, produces a hash that is below a certain target value. The
difficulty of this puzzle is adjusted periodically to ensure that a new
block is created at a relatively constant rate (approximately every 10
minutes for Bitcoin), regardless of the total computational power of the
network.

The first miner to find a valid nonce is rewarded with a certain amount
of cryptocurrency, known as the block reward, as well as the transaction
fees from the transactions included in the block. This economic
incentive motivates miners to contribute their computational resources
to the network, thereby securing the blockchain.

PoW is highly scalable in terms of the number of participating nodes, as
anyone can become a miner by simply dedicating computational resources
to the network. However, it has been widely criticized for its high
energy consumption, as the mining process requires a vast amount of
electricity, comparable to that of entire countries. This has led to the
development of alternative consensus mechanisms, such as Proof-of-Stake,
which are more energy-efficient.

\subsubsection{Proof-of-Stake (PoS)}\label{proof-of-stake-pos}

Proof-of-Stake (PoS) is a class of consensus mechanisms that has emerged
as a more energy-efficient alternative to Proof-of-Work. In a PoS
system, the right to create a new block is not determined by a
computational race, but by the amount of cryptocurrency that a node
holds and is willing to ``stake'' as collateral.

The fundamental principle of PoS is that nodes with a larger stake have
a higher probability of being selected to create the next block. This is
because they have a greater vested interest in the security and
integrity of the network. If a validator attempts to add a fraudulent
block to the chain, they risk losing their stake (a process known as
``slashing''), which serves as a powerful economic disincentive against
malicious behavior.

PoS offers several advantages over PoW, including improved energy
efficiency, lower barriers to entry for validators, and the ability to
achieve faster finality. However, it also introduces new challenges,
such as the ``nothing at stake'' problem, where validators have an
incentive to vote for multiple conflicting chains, and the potential for
centralization if a small number of entities accumulate a large portion
of the stake \autoref{section-11-proof-of-stake-protocols}.

\subsubsection{Proof-of-Authority
	(PoA)}\label{proof-of-authority-poa}

Proof-of-Authority (PoA) is a reputation-based consensus mechanism that
is designed for permissioned blockchains. In a PoA network, new nodes
must be approved by a central or federated authority to become
validators. Instead of staking computational resources or
cryptocurrency, validators in a PoA network stake their reputation.

The core idea behind PoA is that validators are incentivized to act
honestly to maintain their reputation. If a validator is found to be
malicious, they can be removed from the network, and their reputation
will be tarnished. This makes PoA a suitable choice for private or
consortium blockchains where the participants are known and trusted
entities, such as a group of financial institutions or a consortium of
universities.

PoA offers several advantages over other consensus mechanisms, including
high throughput, low transaction costs, and energy efficiency. However,
it is also more centralized than permissionless consensus mechanisms
like PoW and PoS, as the authority to validate transactions is
concentrated in a small group of approved nodes.

\begin{center}\rule{0.5\linewidth}{0.5pt}\end{center}


\subsection{Incentive Schemes and Decentralized
	Identities}\label{section-5-incentive-schemes-and-decentralized-identities}

%\subsubsection{Incentive Schemes}\label{incentive-schemes}

Incentive schemes are a critical component of permissionless and
semi-permissionless blockchains, as they provide the economic motivation
for nodes to participate in the consensus process and act honestly.
These schemes typically consist of two main components:

\begin{itemize}
	\tightlist
	\item
	\textbf{Block Rewards}: These are newly created units of
	cryptocurrency that are awarded to the node that successfully creates
	a new block. Block rewards serve as the primary incentive for miners
	in a PoW system and validators in a PoS system.
	\item
	\textbf{Transaction Fees}: These are small fees that are paid by users
	to have their transactions included in a block. Transaction fees
	provide an additional incentive for consensus nodes and become
	increasingly important as the block reward diminishes over time.
	Moreover, transaction fees create the market for the speed of inclusion of transactions in the blockchains -- the higher the fee, the faster inclusion of transaction.
\end{itemize}

The design of the incentive scheme is crucial for the security and
stability of the blockchain. Misaligned incentives can create
vulnerabilities that can be exploited by attackers. For example, if the
transaction fees are too low, miners may be incentivized to create empty
blocks, which can reduce the overall utility of the network.

\subsubsection{Decentralized Names \&
	Identities}\label{decentralized-names-identities}

The management of names and identities in a decentralized system
presents a unique set of challenges. In a traditional centralized
system, a central authority is responsible for managing identities and
ensuring that names are unique. In a decentralized system, however,
there is no such authority, which can lead to problems such as name
collisions and Sybil attacks.

\medskip
\textbf{Zooko's Triangle} is a well-known conjecture in computer science
that illustrates the trade-off among three desirable properties of a
naming system:

\begin{itemize}
	\tightlist
	\item
	\textbf{Human-meaningful}: The names should be easy for humans to
	remember and use (e.g., ``google.com'').
	\item
	\textbf{Secure}: The names should be resistant to spoofing and other
	forms of attack, meaning a name correctly resolves to its intended
	entity.
	\item
	\textbf{Decentralized}: The system should not rely on a central
	authority for name resolution.
\end{itemize}

\begin{figure}[t]
	%	\vspace{-0.3cm}
	\begin{center}
		\includegraphics[width=0.3\textwidth]{./figs/zooko.png}
		\caption{Zooko's triangle.}		
		\label{fig:zooko}
	\end{center}	
\end{figure}

Zooko's Triangle posits that it is difficult to achieve all three of
these properties simultaneously. For example, DNS is human-meaningful
and secure (with DNSSec) but centralized. Onion addresses in Tor are
secure and decentralized but not human-meaningful. It is believed that
blockchain technology can help to relax this trade-off by providing a
secure and decentralized platform for managing names and identities.

\textbf{Namecoin} was one of the first projects to attempt to create a
decentralized naming system on a blockchain. It is a fork of Bitcoin
that allows users to register key-value pairs in a censorship-resistant
manner, creating a decentralized domain name system (DNS) with the
top-level domain \texttt{.bit}. However, Namecoin has been plagued by
problems such as \textbf{cyber-squatting} (where users register names
they do not own with the intention of selling them) and
\textbf{front-running} (where an attacker sees a registration
transaction and quickly submits their own for the same name with a
higher fee).

\begin{center}\rule{0.5\linewidth}{0.5pt}\end{center}

\subsection{Summary / Key Takeaways}\label{summary-key-takeaways}

This section has provided a brief exploration of consensus
protocols, the foundational mechanisms that enable the secure and
reliable operation of blockchain networks. We have examined the key
concepts and trade-offs that underpin the design of these protocols,
including:

\begin{itemize}
	\tightlist
	\item
	\textbf{Types of Blockchains}: We have distinguished between
	permissionless, permissioned, and semi-permissionless blockchains,
	highlighting the different access control models and their
	implications for security and decentralization.
	\item
	\textbf{The CAP Theorem}: We have explored the fundamental trade-off
	between consistency, availability, and partition tolerance in
	distributed systems, and how this theorem shapes the design of
	blockchain protocols, leading to eventual consistency.
	\item
	\textbf{Goals of Consensus}: We have defined the key properties of a
	robust consensus protocol: safety, liveness, and finality.
	\item
	\textbf{Lottery vs.~Voting}: We have contrasted the two primary
	approaches to achieving consensus, analyzing their respective
	advantages and disadvantages in terms of scalability, network
	overhead, and time to finality.
	\item
	\textbf{Byzantine Fault Tolerance}: We have delved into the classic
	Byzantine Generals Problem and its solution in the form of Byzantine
	Fault Tolerance, which is essential for building systems that are
	resilient to malicious actors, requiring
	\texttt{N\ \textgreater{}\ 3f} nodes.
	\item
	\textbf{Consensus in Practice}: We have examined the most prominent
	consensus protocols used in real-world blockchains, including
	Proof-of-Work, Proof-of-Stake, and Proof-of-Authority.
	\item
	\textbf{Incentive Schemes and Decentralized Identities}: We have
	discussed the importance of economic incentives in securing blockchain
	networks and the challenges of managing names and identities in a
	decentralized environment, as illustrated by Zooko's Triangle.
\end{itemize}

\begin{center}\rule{0.5\linewidth}{0.5pt}\end{center}

\subsection{Keywords}\label{keywords}

\begin{itemize}
	\tightlist
	\item
	\textbf{Consensus Protocol}: A set of rules and procedures that
	enables a distributed network of computers to achieve agreement on a
	single version of the truth, even in the presence of faults or
	malicious actors.
	\item
	\textbf{Permissionless Blockchain}: A type of blockchain that allows
	anyone to join the network and participate in the consensus process
	without requiring permission from a central authority.
	\item
	\textbf{Permissioned Blockchain}: A type of blockchain that restricts
	access to a limited set of participants who have been granted
	permission to join the network.
	\item
	\textbf{CAP Theorem}: A fundamental theorem in distributed computing
	that states that it is impossible for a distributed data store to
	simultaneously provide more than two of the following three
	guarantees: consistency, availability, and partition tolerance.
	\item
	\textbf{Byzantine Fault Tolerance (BFT)}: The property of a system
	that allows it to tolerate a certain number of faulty or malicious
	nodes (\texttt{f}) without compromising the overall integrity of the
	system, typically requiring more than two-thirds of the nodes to be
	honest (\texttt{N\ \textgreater{}\ 3f}).
	\item
	\textbf{Proof-of-Work (PoW)}: A consensus mechanism that requires
	participants (miners) to solve a computationally intensive puzzle to
	create new blocks, thereby securing the network through the
	expenditure of computational resources.
	\item
	\textbf{Proof-of-Stake (PoS)}: A consensus mechanism in which
	participants (validators) are chosen to create new blocks based on the
	amount of cryptocurrency they hold and are willing to ``stake'' as
	collateral.
	\item
	\textbf{Proof-of-Authority (PoA)}: A consensus mechanism that relies
	on the reputation of a set of pre-approved validators to create new
	blocks, making it suitable for permissioned blockchains.
	\item
	\textbf{Fork}: A situation where a blockchain diverges into two or
	more competing chains.
	\item
	\textbf{Finality}: The property of a consensus protocol that
	guarantees that a transaction, once confirmed, cannot be reversed or
	altered.
	\item
	\textbf{Incentive Scheme}: A set of rules that defines how
	participants in a blockchain network are rewarded (e.g., with block
	rewards and transaction fees) for their contributions to the consensus
	process.
	\item
	\textbf{Zooko's Triangle}: A conjecture that describes the trade-off
	between human-meaningful, secure, and decentralized names in a naming
	system.
\end{itemize}

\begin{center}\rule{0.5\linewidth}{0.5pt}\end{center}

\subsection{Further Reading}\label{further-reading}

\begin{itemize}
	\tightlist
	\item
	\textbf{The Byzantine Generals Problem}:\\
	\url{https://lamport.azurewebsites.net/pubs/byz.pdf}
	\item
	\textbf{Practical Byzantine Fault Tolerance}:\\
	\url{https://www.usenix.org/conference/osdi-99/practical-byzantine-fault-tolerance}
\end{itemize}


\newpage

\section{Bitcoin (Part I)}\label{chapter-3-bitcoin-part-1}
This section provides a brief introduction to Bitcoin, the
world's first decentralized digital currency. We will begin by exploring
the historical context in which Bitcoin was created, with a particular
focus on the 2008 financial crisis and the erosion of trust in
traditional financial institutions that it engendered. We will then
delve into the fundamental principles that underpin the Bitcoin network,
including its peer-to-peer architecture, the roles of various network
participants, and the structure of transactions and blocks.

A significant portion of this section is dedicated to explaining the key
technological innovations that make Bitcoin possible. We will examine
the Unspent Transaction Output (UTXO) model, which is used to track the
ownership of bitcoins, and the Proof-of-Work (PoW) consensus mechanism,
which secures the network and ensures its integrity. We will also
discuss the challenges that Bitcoin faces, such as its limited
scalability and the privacy concerns associated with its public ledger.
Finally, we will introduce the Lightning Network, a promising Layer 2
solution that aims to address Bitcoin's scalability limitations and
enable fast, low-cost transactions.

\subsection{Learning Objectives}\label{learning-objectives}

\begin{itemize}
	\tightlist
	\item
	Understand the historical context and motivation behind the creation
	of Bitcoin.
	\item
	Grasp the basic concepts of the Bitcoin network, including clients,
	miners, addresses, and transactions.
	\item
	Learn about the structure of the Bitcoin blockchain and the role of
	the Proof-of-Work consensus mechanism.
	\item
	Understand the Unspent Transaction Output (UTXO) model and how it is
	used to track ownership of bitcoins.
	\item
	Gain insight into the challenges facing Bitcoin, such as scalability,
	energy consumption, and privacy.
	\item
	Learn about the Lightning Network as a potential solution to Bitcoin's
	scalability issues.
\end{itemize}

\begin{center}\rule{0.5\linewidth}{0.5pt}\end{center}

\subsection{The History and Genesis of
	Bitcoin}\label{section-1-the-history-and-genesis-of-bitcoin}

\subsubsection{The White Paper and the 2008 Financial
	Crisis}\label{the-white-paper-and-the-2008-financial-crisis}

The genesis of Bitcoin is inextricably linked to the global financial
crisis of 2008. In October of that year, as the crisis was unfolding, a
white paper titled ``Bitcoin: A Peer-to-Peer Electronic Cash System''~\cite{nakamoto2008bitcoin}
was published under the pseudonym Satoshi Nakamoto. The paper proposed a
revolutionary new system for electronic transactions that would operate
without the need for a trusted third party, such as a bank or financial
institution.

The timing of the white paper's release was no coincidence. The 2008
financial crisis, triggered by the collapse of the subprime mortgage
market in the United States, led to a widespread loss of faith in the
traditional financial system. The crisis exposed the systemic risks
inherent in a centralized banking system and the potential for
governments to devalue currencies through inflation. In this climate of
distrust and uncertainty, the idea of a decentralized,
censorship-resistant, and mathematically secured form of money was
particularly appealing.

The Bitcoin white paper laid out the technical foundations for a system
that would allow for direct, peer-to-peer transactions, secured by
cryptographic proof instead of trust. This vision of a new financial
paradigm, free from the control of central authorities, resonated with a
growing community of cryptographers, cypherpunks, and libertarians who
were disillusioned with the existing financial order.

\subsubsection{Satoshi Nakamoto and the Early
	Days}\label{satoshi-nakamoto-and-the-early-days}

The identity of Satoshi Nakamoto remains one of the most enduring
mysteries of the digital age. The name is a pseudonym for the person or
group of people who created Bitcoin. From 2008 to 2011, Nakamoto was an
active participant in the development of Bitcoin, collaborating with
other developers on forums and mailing lists. However, in 2011, Nakamoto
abruptly disappeared, leaving the future of Bitcoin in the hands of the
community.

Despite numerous attempts to uncover the true identity of Satoshi
Nakamoto, it remains unknown. Several individuals have been suggested as
potential candidates, but none have been definitively proven to be the
creator of Bitcoin. The legacy of Satoshi Nakamoto is not just the
Bitcoin protocol and its reference implementation, but also the vision
of a decentralized financial system that has inspired a global movement
and a multi-trillion dollar industry.

\begin{center}\rule{0.5\linewidth}{0.5pt}\end{center}

\subsection{The Basics of Bitcoin}\label{section-2-bitcoin-101-the-basics}

\subsubsection{Key Facts and Figures}\label{key-facts-and-figures}

Bitcoin is defined by a set of core parameters that are hard-coded into
the protocol. These parameters govern the issuance of new bitcoins, the
size of blocks, and the rate at which new blocks are created.

\begin{itemize}
	\tightlist
	\item
	\textbf{Maximum Supply}: The total supply of bitcoin is capped at 21
	million. This fixed supply is a fundamental aspect of Bitcoin's
	monetary policy and is designed to make it a deflationary currency.
	\item
	\textbf{Block Size}: The maximum size of a Bitcoin block is 4
	megabytes (MB). This limit was introduced to prevent spam and
	denial-of-service attacks on the network.
	\item
	\textbf{Consensus Mechanism}: Bitcoin uses the Proof-of-Work (PoW)
	consensus mechanism, which requires miners to expend computational
	energy to create new blocks.
	\item
	\textbf{Block Time}: The Bitcoin protocol is designed to target a
	block time of approximately 10 minutes. This means that a new block is
	added to the blockchain, on average, every 10 minutes.
\end{itemize}

\subsubsection{The Bitcoin Network}\label{the-bitcoin-network}

The Bitcoin network is a global, peer-to-peer (P2P) network of computers
that work together to maintain the integrity of the blockchain. The
network is composed of various types of participants (as we mentioned in \autoref{sec:participants-in-a-blockchain-network}), each with a
specific role:

\begin{itemize}
	\tightlist
	\item
	\textbf{Lightweight Clients, a.k.a., SPV (Simple Payment Verification) Clients}: These are
	lightweight nodes that do not store the entire blockchain. Instead,
	they only download the block headers, which allows them to verify
	transactions with the help of full nodes without having to download
	and process the entire blockchain.
	Note that less secure version of clients are represented by some wallets that rely on a trusted centralized server for providing information about the current state of the blockchain. 
	These are software or hardware
	applications that allow users to store and manage their bitcoins.
	Wallets generate and store the user's private keys, which are required
	to sign transactions and authorize the spending of funds.
	\item
	\textbf{Miners / Consensus Nodes}: These are specialized nodes that are responsible for
	creating new blocks. They do this by collecting pending transactions
	from the network, organizing them into a new block, and then competing
	to solve the Proof-of-Work puzzle. The first miner to solve the puzzle
	is rewarded with newly created bitcoins and the transaction fees from
	the transactions included in the block.
	\item
	\textbf{Full Nodes}: These are nodes that store a complete copy of the
	Bitcoin blockchain. They independently validate all transactions and
	blocks against the protocol's consensus rules, ensuring that all
	participants are adhering to the same set of rules. Full nodes are the
	backbone of the network, providing a high level of security and
	decentralization.


\end{itemize}

\subsubsection{Addresses and
	Transactions}\label{addresses-and-transactions}

\begin{itemize}
	\tightlist
	\item
	\textbf{Addresses}: A Bitcoin address is a unique identifier,
	analogous to an email address, that is used to send and receive
	bitcoins. It is a string of alphanumeric characters that is derived
	from a user's public key using the ECDSA algorithm on the secp256k1
	curve, followed by SHA-256 and RIPEMD-160 hashing.
	\item
	\textbf{Transactions}: A Bitcoin transaction is a digitally signed
	message that authorizes the transfer of value from one address to
	another. A transaction consists of one or more inputs and one or more
	outputs. The inputs are references to unspent outputs from previous
	transactions, and the outputs specify the new owners of the bitcoins
	and the amount they are entitled to spend.
\end{itemize}

\begin{center}\rule{0.5\linewidth}{0.5pt}\end{center}

\subsection{Diving Deeper into the Blockchain and Consensus}\label{section-3-diving-deeper-into-the-blockchain}

\subsubsection{Block Structure and the Merkle
	Tree}\label{block-structure-and-the-merkle-tree}

A Bitcoin block is a data structure that contains a batch of
transactions. It is composed of two main parts: a block header and a
block body (see \autoref{fig:block-structure}). The block body contains the list of transactions that are
included in the block. These transactions are organized into a
\textbf{Merkle tree}, a data structure that allows for efficient
verification of the integrity of the transactions. The root of the
Merkle tree, known as the Merkle root, is included in the block header.

\begin{figure}[t]
	%	\vspace{-0.3cm}
	\begin{center}
		\includegraphics[width=0.9\textwidth]{./figs/block-structructure.png}
		\caption{Bitcoin's block structure.}		
		\label{fig:block-structure}
	\end{center}	
\end{figure}

The block header contains several important pieces of information,
including:

\begin{itemize}
	\tightlist
	\item
	\textbf{Version}: The block version number.
	\item
	\textbf{Previous Block Hash}: The hash of the previous block's header,
	which links the blocks together in a chain.
	\item
	\textbf{Merkle Root}: The root of the Merkle tree of all the
	transactions in the block.
	\item
	\textbf{Timestamp}: The time at which the block was created.
	\item
	\textbf{Difficulty Target}: The target value that the hash of the
	block header must be less than.
	\item
	\textbf{Nonce}: A random value that is incremented by miners in the
	process of finding a valid hash.
\end{itemize}

\subsubsection{Proof-of-Work and Nakamoto's
	Consensus}\label{proof-of-work-and-nakamotos-consensus}

The security of the Bitcoin blockchain is underpinned by the
\textbf{Proof-of-Work (PoW)} consensus mechanism, often referred to as
Nakamoto's consensus. This mechanism ensures that new blocks are added
to the blockchain in a secure and decentralized manner.

In the PoW system, miners compete to solve a computationally intensive
puzzle. This puzzle involves finding a nonce that, when combined with
the other fields in the block header and hashed, produces a hash that is
below a certain target value. The first miner to find a valid hash is
rewarded with newly created bitcoins and the transaction fees from the
transactions included in the block.

The schematic diagram of finding a PoW puzzle of Bitcoin is depicted in \autoref{fig:pow-puzzle-scheme}.
\begin{figure}[t]
	%	\vspace{-0.3cm}
	\begin{center}
		\includegraphics[width=1.0\textwidth]{./figs/pow-scheme.png}
		\caption{Finding a PoW puzzle of Bitcoin~\cite{Rybarczyk2020BitcoinHeader}.}		
		\label{fig:pow-puzzle-scheme}
	\end{center}	
\end{figure}

Also a simplified Python code snippet that demonstrates the
Proof-of-Work algorithm is depicted in \autoref{fig:pow-puzzle-python}.
\begin{figure}[t]
	%	\vspace{-0.3cm}
	\begin{center}
		\includegraphics[width=0.9\textwidth]{./figs/pow-python.png}
		\caption{A simplified python code of finding a PoW puzzle in Bitcoin.}		
		\label{fig:pow-puzzle-python}
	\end{center}	
\end{figure}



%\begin{Shaded}
%	\begin{Highlighting}[]
%		\ImportTok{import}\NormalTok{ hashlib, struct}
%		
%		\NormalTok{ver }\OperatorTok{=} \DecValTok{2}
%		\NormalTok{prev\_block }\OperatorTok{=} \StringTok{"000000000000000117c80378b8da0e33559b5997f2ad55e2f7d18ec1975b9717"}
%		\NormalTok{mrkl\_root }\OperatorTok{=} \StringTok{"871714dcbae6c8193a2bb9b2a69fe1c0440399f38d94b3a0f1b447275a29978a"}
%		\NormalTok{time\_ }\OperatorTok{=} \BaseNTok{0x53058b35} \CommentTok{\# 2014{-}02{-}20 04:57:25}
%		\NormalTok{bits }\OperatorTok{=} \BaseNTok{0x19015f53}
%		
%		\NormalTok{exp }\OperatorTok{=}\NormalTok{ bits }\OperatorTok{\textgreater{}\textgreater{}} \DecValTok{24}
%		\NormalTok{mant }\OperatorTok{=}\NormalTok{ bits }\OperatorTok{\&} \BaseNTok{0xffffff}
%		\NormalTok{target\_hexstr }\OperatorTok{=} \StringTok{\textquotesingle{}}\SpecialCharTok{\%064x}\StringTok{\textquotesingle{}} \OperatorTok{\%}\NormalTok{ (mant }\OperatorTok{*}\NormalTok{ (}\DecValTok{1}\OperatorTok{\textless{}\textless{}}\NormalTok{(}\DecValTok{8}\OperatorTok{*}\NormalTok{(exp }\OperatorTok{{-}} \DecValTok{3}\NormalTok{))))}
%		\NormalTok{target\_str }\OperatorTok{=}\NormalTok{ target\_hexstr.decode(}\StringTok{\textquotesingle{}hex\textquotesingle{}}\NormalTok{)}
%		
%		\NormalTok{nonce }\OperatorTok{=} \DecValTok{0}
%		\ControlFlowTok{while}\NormalTok{ nonce }\OperatorTok{\textless{}} \BaseNTok{0x100000000}\NormalTok{:}
%		\NormalTok{    header }\OperatorTok{=}\NormalTok{ ( struct.pack(}\StringTok{"\textless{}L"}\NormalTok{, ver) }\OperatorTok{+}\NormalTok{ prev\_block.decode(}\StringTok{\textquotesingle{}hex\textquotesingle{}}\NormalTok{)[::}\OperatorTok{{-}}\DecValTok{1}\NormalTok{] }\OperatorTok{+}
%		\NormalTok{    mrkl\_root.decode(}\StringTok{\textquotesingle{}hex\textquotesingle{}}\NormalTok{)[::}\OperatorTok{{-}}\DecValTok{1}\NormalTok{] }\OperatorTok{+}\NormalTok{ struct.pack(}\StringTok{"\textless{}LLL"}\NormalTok{, time\_, bits, nonce))}
%		\BuiltInTok{hash} \OperatorTok{=}\NormalTok{ hashlib.sha256(hashlib.sha256(header).digest()).digest()}
%		\BuiltInTok{print}\NormalTok{ nonce, }\BuiltInTok{hash}\NormalTok{[::}\OperatorTok{{-}}\DecValTok{1}\NormalTok{].encode(}\StringTok{\textquotesingle{}hex\textquotesingle{}}\NormalTok{)}
%		\ControlFlowTok{if} \BuiltInTok{hash}\NormalTok{[::}\OperatorTok{{-}}\DecValTok{1}\NormalTok{] }\OperatorTok{\textless{}}\NormalTok{ target\_str:}
%		\BuiltInTok{print} \StringTok{\textquotesingle{}success\textquotesingle{}}
%		\ControlFlowTok{break}
%		\NormalTok{    nonce }\OperatorTok{+=} \DecValTok{1}
%	\end{Highlighting}
%\end{Shaded}

The \textbf{difficulty} of the mining puzzle is a critical parameter
that is adjusted every 2016 blocks (approximately every two weeks) to
maintain a consistent block time of around 10 minutes. If blocks are
being created too quickly, the difficulty is increased. If they are
being created too slowly, the difficulty is decreased. This ensures that
the rate of new bitcoin issuance remains predictable and that the
network remains secure.

\subsubsection{The Coinbase Transaction and the Genesis
	Block}\label{the-coinbase-transaction-and-the-genesis-block}

The first transaction in every block is a special transaction known as
the \textbf{coinbase transaction}. This transaction is created by the
miner who successfully mined the block and has two main purposes:

\begin{enumerate}
	\def\labelenumi{\arabic{enumi}.}
	\tightlist
	\item
	\textbf{Block Reward}: It allows the miner to claim the block reward,
	which is a certain number of newly created bitcoins. The block reward
	is halved approximately every four years in an event known as the
	``halving.''
	\item
	\textbf{Transaction Fees}: It allows the miner to collect the
	transaction fees from all the other transactions included in the
	block.
\end{enumerate}

The very first block in the Bitcoin blockchain, block 0, is known as the
\textbf{Genesis Block}. It was created by Satoshi Nakamoto on January 3,
2009. The coinbase transaction of the Genesis Block contains a
now-famous message: ``\textit{The Times 03/Jan/2009 Chancellor on brink of
second bailout for banks}.'' This message is widely interpreted as a
commentary on the instability of the traditional financial system and a
statement of intent for Bitcoin as a decentralized alternative.

\begin{center}\rule{0.5\linewidth}{0.5pt}\end{center}

\subsection{The UTXO Model}\label{section-4-the-utxo-model}

\subsubsection{Unspent Transaction Outputs
	(UTXOs)}\label{unspent-transaction-outputs-utxos}

Unlike traditional banking systems that use an account-based model,
Bitcoin uses the \textbf{Unspent Transaction Output (UTXO)} model to
track the ownership of funds. In the UTXO model, a user's balance is not
stored as a single value in an account. Instead, it is the sum of all
the individual, unspent transaction outputs that are locked to the
user's address.

\begin{figure}[t]
	%	\vspace{-0.3cm}
	\begin{center}
		\includegraphics[width=0.9\textwidth]{./figs/utxo.png}
		\caption{Example of UTXO spending.}		
		\label{fig:utxo}
	\end{center}	
\end{figure}

A UTXO is an output of a transaction that has not yet been spent and can
be used as an input in a future transaction. Each UTXO is a discrete
amount of bitcoin that is associated with a specific address. When a
user wants to send bitcoins, their wallet selects a set of UTXOs that
are sufficient to cover the amount of the transaction. These UTXOs are
then used as inputs in the new transaction (see \autoref{fig:utxo}), and new UTXOs are created as
outputs, which are locked to the recipient's address.

Every full node in the Bitcoin network maintains a database of all the
currently unspent transaction outputs, known as the \textbf{UTXO set}. This
database is used to validate new transactions and to ensure that users
can only spend bitcoins that they actually own.

\subsubsection{Transaction Fees and Change
	Outputs}\label{transaction-fees-and-change-outputs}

\begin{itemize}
	\tightlist
	\item
	\textbf{Transaction Fees}: To incentivize miners to include their
	transactions in a block, users can include a transaction fee. The fee
	is the difference between the total value of the inputs and the total
	value of the outputs of a transaction. Miners will typically
	prioritize transactions with higher fees, as this increases their
	profitability.
	\item
	\textbf{Change Outputs}: If the total value of the UTXOs used as
	inputs in a transaction is greater than the amount the user wants to
	send, the excess amount is sent back to the user in a new UTXO, known
	as a \textbf{change output}. This is analogous to receiving change
	when paying for something with cash.
\end{itemize}

\subsubsection{Double-Spending}\label{double-spending}
Double-spending is the act of attempting to spend the same UTXO in more
than one transaction of two or more different temporary or malicious forks. The Bitcoin protocol is designed to prevent
double-spending by ensuring that only one of the conflicting
transactions can be included in the blockchain. Once a transaction is
included in a block and confirmed by the network, the UTXOs it uses as
inputs are considered spent and cannot be used again. Any subsequent
transaction that attempts to spend the same UTXOs will be rejected by
the network as invalid.

\begin{center}\rule{0.5\linewidth}{0.5pt}\end{center}

\subsection{The Lightning Network: A Scalability
	Solution}\label{section-5-the-lightning-network-a-scalability-solution}

One of the most significant challenges facing Bitcoin is its limited
scalability. The combination of a 4 MB block size limit and a 10-minute block time restricts the network's
transaction throughput to approximately 7 transactions per second. This
is a major bottleneck that prevents Bitcoin from being used as a global
payment system for everyday transactions, which would require a much
higher throughput.

\subsubsection{The Lightning Network}\label{the-lightning-network}

The Lightning Network is a Layer 2 scaling solution that is designed to
address Bitcoin's scalability limitations. It is a decentralized network
of off-chain payment channels that allows for instant, low-cost
transactions.

The core idea behind the Lightning Network is to move the majority of
transactions off the main Bitcoin blockchain. Users can open payment
channels with each other by committing a certain amount of bitcoin to a
multi-signature address. Once a channel is open, the two parties can
transact with each other an unlimited number of times without having to
broadcast each transaction to the main blockchain. In this way pairwise network is formed and LN transactions can be route through intermediaries that have sufficient channel capacity (see \autoref{fig:ln}) -- in turn these intermediaries earn routing fees. This allows for instant and virtually free transactions.

\begin{figure}[t]
	%	\vspace{-0.3cm}
	\begin{center}
		\includegraphics[width=0.9\textwidth]{./figs/ln.png}
		\caption{Lightning network routing and channel update (i.e., balance change).}		
		\label{fig:ln}
	\end{center}	
\end{figure}

The main blockchain is only used to open and close the payment channels.
When the two parties decide to close the channel, the final balance is
settled on the Bitcoin blockchain.

The Lightning Network uses a clever mechanism called \textbf{Hashed
	Time-Locked Contracts (HTLCs)} to enable trustless, multi-hop payments
across the network. This means that two users can transact with each
other even if they do not have a direct payment channel, as long as
there is a path of connected channels between them. In particular, hash time locks enable to conditionally redeem funds by recipient from the sender's output if the pre-image of a hash is revealed in specified time windows. If it is not revealed the funds can be redeemed back by the sender.

\begin{center}\rule{0.5\linewidth}{0.5pt}\end{center}

\subsection{Summary / Key Takeaways}\label{summary-key-takeaways}

This section has provided a detailed introduction to the world of
Bitcoin, the first and most prominent cryptocurrency. We have explored
its origins in the context of the 2008 financial crisis, its core
principles of decentralization and censorship resistance, and the key
technological components that make it function.

We have delved into the peer-to-peer architecture of the Bitcoin
network, identifying the various roles of its participants, from clients
and full nodes to miners. We have also examined the structure of Bitcoin
transactions and blocks, and the crucial role of the Merkle tree in
ensuring data integrity.

A key focus of this section has been the UTXO model, which is Bitcoin's
unique approach to tracking the ownership of funds. We have also
explored the Proof-of-Work consensus mechanism, which is the engine that
secures the Bitcoin blockchain and ensures its immutability.

Finally, we have acknowledged the challenges that Bitcoin faces,
particularly in the areas of scalability and privacy, and we have
introduced the Lightning Network as a promising Layer 2 solution that
aims to address these challenges. This section has laid the groundwork
for a deeper understanding of more advanced topics in blockchain
technology and decentralized applications.

\begin{center}\rule{0.5\linewidth}{0.5pt}\end{center}

\subsection{Keywords}\label{keywords}

\begin{itemize}
	\tightlist
	\item
	\textbf{Bitcoin}: A decentralized digital currency that enables
	peer-to-peer transactions without the need for a central authority.
	\item
	\textbf{Proof-of-Work (PoW)}: A consensus mechanism that secures a
	blockchain by requiring participants to solve a computationally
	intensive puzzle.
	\item
	\textbf{UTXO (Unspent Transaction Output)}: The fundamental building
	block of a Bitcoin transaction, representing a discrete amount of
	bitcoin that can be spent.
	\item
	\textbf{Mining}: The process of creating new blocks in a Proof-of-Work
	blockchain by solving the computational puzzle.
	\item
	\textbf{Lightning Network}: A Layer 2 payment protocol that operates
	on top of the Bitcoin blockchain, enabling fast and low-cost
	transactions.
	\item
	\textbf{Satoshi Nakamoto}: The pseudonymous creator of Bitcoin.
	\item
	\textbf{Genesis Block}: The first block in the Bitcoin blockchain.
	\item
	\textbf{Coinbase Transaction}: A special transaction in each block
	that rewards the miner with newly created bitcoins and transaction
	fees.
	\item
	\textbf{Double-Spending}: The act of spending the same UTXO in
	multiple transactions.
	\item
	\textbf{Hashed Time-Locked Contract (HTLC)}: A type of smart contract
	used in the Lightning Network to enable trustless, multi-hop payments.
\end{itemize}

\begin{center}\rule{0.5\linewidth}{0.5pt}\end{center}

\subsection{Further Reading}\label{further-reading}

\begin{itemize}
	\tightlist
	\item
	\textbf{Bitcoin: A Peer-to-Peer Electronic Cash System}:\\
	\url{https://bitcoin.org/bitcoin.pdf}
	\item
	\textbf{Bitcoin Developer Guide}:\\
	\url{https://bitcoin.org/en/developer-guide}
	\item
	\textbf{Mastering Bitcoin by Andreas M. Antonopoulos}:\\
	\url{https://github.com/bitcoinbook/bitcoinbook}
\end{itemize}

\newpage

\section{Bitcoin (Part II)}\label{chapter-3-bitcoin-part-2}
Building upon the foundational concepts introduced in the previous
section, this section delves deeper into the technical intricacies of
the Bitcoin protocol. We will begin by exploring Bitcoin Script, the
simple yet powerful scripting language that underpins all Bitcoin
transactions. We will examine how this stack-based language is used to
create sophisticated spending conditions, enabling a wide range of
transaction types beyond simple payments.

The section will then trace the evolution of Bitcoin's standard script
types, from the early Pay-to-Public-Key (P2PK) and
Pay-to-Public-Key-Hash (P2PKH) scripts to the more advanced
Pay-to-Script-Hash (P2SH) and multi-signature (P2MS) scripts. We will
analyze the motivation behind each of these developments and the new
capabilities they introduced.

A significant portion of this section is dedicated to the two most
important upgrades in Bitcoin's history: Segregated Witness (SegWit) and
Taproot. We will explore the technical details of these upgrades,
including how SegWit solved the long-standing problem of transaction
malleability and how Taproot introduced Schnorr signatures and
Merkelized Alternative Script Trees (MAST) to enhance privacy,
efficiency, and smart contract capabilities.

%\ih{drop} Finally, we will discuss the concept of blockchain forks in greater
%detail, examining the different types of forks and their implications
%for the network. We will also analyze the threat of a 51\% attack and
%other potential vulnerabilities that could compromise the security and
%integrity of the Bitcoin network.

\subsection{Learning Objectives}\label{learning-objectives}

\begin{itemize}
	\tightlist
	\item
	Understand the fundamentals of Bitcoin Script and its role in defining
	spending conditions.
	\item
	Learn about the different standard script types, including P2PK,
	P2PKH, P2SH, and the various SegWit and Taproot formats.
	\item
	Grasp the concepts of transaction malleability and how Segregated
	Witness (SegWit) addresses this issue.
	\item
	Understand the benefits of the Taproot upgrade, including improved
	privacy and efficiency.
%	\item
%	\ih{drop} Learn about the different types of blockchain forks and their
%	implications for the network.
	\item
	Gain insight into the 51\% attack and other potential vulnerabilities
	in the Bitcoin network.
\end{itemize}

\begin{center}\rule{0.5\linewidth}{0.5pt}\end{center}

\subsection{The UTXO Model and Bitcoin
	Script}\label{section-1-the-utxo-model-and-bitcoin-script}

\subsubsection{Recap of the UTXO
	Model}\label{recap-of-the-utxo-model}

As established in the preceding section, the Bitcoin protocol employs
the \textbf{Unspent Transaction Output (UTXO)} model as its fundamental
accounting mechanism. This model represents a paradigm shift from the
traditional account-based systems used in conventional finance. Instead
of maintaining balances in accounts, the Bitcoin ledger comprises a
collection of UTXOs, each representing a discrete and unspent amount of
bitcoins. Every transaction consumes one or more UTXOs as inputs and
generates one or more new UTXOs as outputs, thereby creating a
continuous and verifiable chain of ownership.

\subsubsection{Introduction to Bitcoin
	Script}\label{introduction-to-bitcoin-script}

At the heart of the UTXO model is \textbf{Bitcoin Script}, a simple,
stack-based programming language that governs the spending conditions
for each UTXO. Every transaction output is encumbered with a locking
script, formally known as \texttt{ScriptPubKey}, which specifies the
conditions that must be met to spend the associated funds. To redeem a
UTXO, a subsequent transaction must provide a corresponding unlocking
script, or \texttt{ScriptSig}, that satisfies the requirements of the
\texttt{ScriptPubKey} (see \autoref{fig:utxo-unlock}).


\begin{figure}[t]
	%	\vspace{-0.3cm}
	\begin{center}
		\includegraphics[width=0.9\textwidth]{./figs/utxo-unlocking.png}
		\caption{Bitcoin transaction with locking and unlocking scripts.}		
		\label{fig:utxo-unlock}
	\end{center}	
\end{figure}



The design of Bitcoin Script is intentionally minimalistic. It is not
Turing-complete, meaning that it lacks the ability to perform loops or
complex recursive operations. This limitation is a deliberate security
feature, designed to prevent the execution of overly complex or
malicious scripts that could potentially disrupt the network.

\subsubsection{Standard Script Types}\label{standard-script-types}

While Bitcoin Script allows for a high degree of flexibility in defining
spending conditions, a set of standard script types has emerged over
time to accommodate the most common use cases. These standard scripts
are recognized and relayed by all nodes in the network, ensuring a high
degree of interoperability.

\begin{itemize}
	\item
	\textbf{Pay-to-Public-Key (P2PK)}: This was the original script type
	used in the earliest days of Bitcoin. It locks a UTXO directly to a
	specific public key, and the corresponding unlocking script simply
	requires a valid signature from the owner of that public key. While
	simple, P2PK has been largely superseded by more advanced script
	types.
	\item
	\textbf{Pay-to-Public-Key-Hash (P2PKH)}: This is the most common
	script type in use today. Instead of locking a UTXO to a public key,
	it locks it to the hash of a public key. This provides two main
	advantages: it results in a shorter and more convenient address
	format, and it offers a degree of privacy protection, as the public key is not revealed until the UTXO is spent.
	\item
	\textbf{Pay-to-MultiSig (P2MS)}: This script type enables
	multi-signature transactions, which require the approval of multiple
	parties to spend a UTXO. A P2MS script specifies a set of \texttt{n}
	public keys and a threshold \texttt{m}, where
	$m \leq n$. To spend the UTXO, at least \texttt{m}
	valid signatures corresponding to the specified public keys must be
	provided.
	\item
	\textbf{Pay-to-Script-Hash (P2SH)}: This is a more advanced and
	flexible script type that allows a UTXO to be locked to the hash of an
	arbitrary script. This enables the creation of complex spending
	conditions without revealing the underlying script until the UTXO is
	spent. P2SH is commonly used for multi-signature wallets and other
	advanced smart contract-like functionality.
\end{itemize}

\begin{center}\rule{0.5\linewidth}{0.5pt}\end{center}

\subsection{Segregated Witness (SegWit) and
	Taproot}\label{section-2-segregated-witness-segwit-and-taproot}

\subsubsection{The Need for Upgrades}\label{the-need-for-upgrades}

As the Bitcoin network grew and evolved, several limitations of the
original protocol became apparent: 

\begin{itemize}
	\tightlist
	\item
	\textbf{Transaction Malleability}: This was a long-standing
	vulnerability in the Bitcoin protocol that allowed a third party to
	alter the signature data of an unconfirmed transaction, thereby
	changing its transaction ID (TXID) without invalidating the
	transaction itself. This could cause problems for services that relied
	on unconfirmed transactions referred by their original IDs, such as payment processors and exchanges. The fix was provided by SeqWit.
	\item
	\textbf{Scalability}: The 1 MB block size limit severely restricted
	the number of transactions that could be processed by the network,
	leading to congestion and high fees during periods of high demand.
	\item
	\textbf{Privacy}: The public nature of the blockchain made it possible
	to analyze transaction patterns and link addresses to real-world
	identities, raising privacy concerns for users.
\end{itemize}
These challenges prompted the
development of two major upgrades: Segregated Witness (SegWit) and
Taproot. 

\subsubsection{Segregated Witness
	(SegWit)}\label{segregated-witness-segwit}

Segregated Witness (SegWit)~\cite{bip141-segwit} was a soft fork upgrade that was activated
on the Bitcoin network in August 2017. It addressed the issue of
transaction malleability by separating the \textbf{signature data}, or
``\textbf{witness},'' from the main transaction data. By moving the witness data to a separate data structure, SegWit made it impossible for a third
party to alter the signature and change the TXID.

In addition to fixing transaction malleability, SegWit also introduced a
new concept of ``block weight,'' which theoretically increased the block
size limit to 4 MB for blocks containing SegWit transactions.\footnote{The base data has 4x higher weight than witness data, and therefore the block containing only base data has maximum size of 1MB.}
This slightly improved the scalability of the network.

SegWit also introduced new address formats, known as \textbf{P2WPKH
(Pay-to-Witness-Public-Key-Hash)} and \textbf{P2WSH (Pay-to-Witness-Script-Hash)},
which offer lower transaction fees than their legacy counterparts. To
ensure backward compatibility, SegWit also provided a mechanism for
``wrapping'' SegWit transactions in P2SH addresses.
Overview of all existing output types in UTXO is depicted in \autoref{fig:btc-scripts}.

\begin{table}[t]
	%	\vspace{-0.3cm}
	\begin{center}
		\includegraphics[width=0.9\textwidth]{./figs/scripts}
		\caption{Output types in UTXO and their locking and unlocking scripts.}		
		\label{fig:btc-scripts}
	\end{center}	
\end{table}

\subsubsection{Taproot}\label{taproot}

Taproot~\cite{bip341-taproot} is the most significant upgrade to the Bitcoin protocol since
SegWit, and it was activated in November 2021. It introduced a number of
new features that further enhance the privacy, efficiency, and smart
contract capabilities of the network. I  particular, Taproot introduced the following:

\begin{itemize}
	\item
	\textbf{Schnorr Signatures}: Taproot replaced the Elliptic Curve
	Digital Signature Algorithm (ECDSA) with Schnorr signatures. Schnorr
	signatures offer several advantages over ECDSA, including their
	smaller size and their ability to be aggregated. This means that
	multiple signatures can be combined into a single signature, which
	makes multi-signature transactions indistinguishable from
	single-signature transactions on the blockchain. This significantly
	improves the privacy of multi-signature wallets and reduces their
	transaction fees since only one signature is enough to be stored.
	\item
	\textbf{MAST (Merkelized Alternative Script Trees)}: Taproot also
	introduced Merkelized Alternative Script Trees (MAST), a new way of
	constructing complex spending conditions. With MAST, a user can create
	a tree of different scripts, each representing a different spending
	condition. When the UTXO is spent, only the script that is actually
	used is revealed on the blockchain. This improves privacy by hiding
	the other possible spending conditions, and it also reduces the amount
	of data that needs to be stored on the blockchain, thereby improving
	efficiency.
\end{itemize}

\subsubsection{Ordinals and
	Inscriptions}\label{ordinals-and-inscriptions}

The Taproot upgrade had an unforeseen consequence: the emergence of
\textbf{Ordinals and Inscriptions}. The Ordinals protocol is a system
for numbering individual satoshis, the smallest unit of bitcoin, and
tracking them across transactions. The Inscriptions protocol allows
users to inscribe arbitrary data, such as images and text, onto
individual satoshis.

This has led to the creation of NFT-like assets on the Bitcoin
blockchain, which has been a source of considerable controversy within
the community. While some see it as an innovative new use case for
Bitcoin, others are concerned that it is leading to network congestion
and higher transaction fees, and that it is a departure from Bitcoin's
original purpose as a peer-to-peer electronic cash system.

\begin{center}\rule{0.5\linewidth}{0.5pt}\end{center}

\subsection{The 51\% Attack}\label{section-3-blockchain-forks-and-network-security}

A 51\% attack is a potential attack on a Proof-of-Work blockchain where
a single entity or a group of colluding entities gains control of more
than 50\% of the network's total mining hash rate. With this level of
control, an attacker could theoretically:

\begin{itemize}
	\tightlist
	\item
	\textbf{Prevent new transactions from being confirmed}: By refusing to
	include certain transactions in the blocks they mine, an attacker
	could effectively censor users.
	\item
	\textbf{Halt payments between some or all users}: By creating empty
	blocks, an attacker could prevent any transactions from being
	processed.
	\item
	\textbf{Reverse transactions}: An attacker could use their majority
	hash power to create a private chain that is longer than the public
	chain. They could then broadcast this longer chain to the network,
	which would cause the public chain to be orphaned and all the
	transactions in it to be reversed. This would allow the attacker to
	double-spend their coins.
\end{itemize}

While a 51\% attack is a serious threat, it is important to note that it
is extremely difficult and expensive to execute on a large and
decentralized network like Bitcoin. The cost of acquiring the necessary
hardware and electricity to control more than 50\% of the Bitcoin
network's hash rate would be astronomical. Furthermore, a successful
51\% attack would likely undermine confidence in the network, causing
the price of the cryptocurrency to plummet and rendering the attack
unprofitable.

\subsection{A Look at Influential
	Altcoins}\label{section-5-a-look-at-influential-altcoins}

The search for better consensus mechanisms has led to the creation of
thousands of ``altcoins.'' Many of the earliest and most influential
were forks of Bitcoin, created to experiment with different features.
The clone ratio of several coins cloned from Bitcoin  is depicted in \autoref{fig:btc-clones}.
However, cloning the code also implies cloning the vulnerabilities, and if revealed, they should be fixed as soon as possible. 
Nevertheless, this was not the case, as indicated by the study Hum et al~\cite{hum2020coinwatch} -- see \autoref{tab:clones}.



\begin{figure}[t]
	\centering
	\includegraphics[width=0.99\textwidth]{./figs/geneaolgy.png}
	\caption{A genealogy map showing the proliferation of altcoins from
		Bitcoin and other major projects.}
\end{figure}

\begin{figure}
	\centering
	\includegraphics[width=0.6\textwidth]{./figs/btc-clones.png}
	\caption{Bitcoin clones -- ratio of cloned code as of 2020~\cite{hum2020coinwatch}.}\label{fig:btc-clones}
\end{figure}

\begin{table}[bh]
	\centering
	\includegraphics[width=0.5\textwidth]{./figs/clones-vulns.png}
	\caption{Vulnerable period of CVE-2018-17144 across Bitcoin clones.}\label{tab:clones}
\end{table}



\begin{itemize}
	\tightlist
	\item
	\textbf{Namecoin (2011)}: The first altcoin, a fork of Bitcoin that
	implemented a decentralized domain name registration system.
	\item
	\textbf{Litecoin (2011)}: One of the most famous early altcoins, it
	adopted the scrypt mining algorithm for ASIC resistance and featured a
	4x faster block time than Bitcoin.
	\item
	\textbf{Peercoin (2012)}: Pioneered a hybrid
	Proof-of-Work/Proof-of-Stake system to address PoW's energy
	consumption and centralization issues.
	\item
	\textbf{Dogecoin (2013)}: Started as a joke fork of Litecoin, it
	gained massive popularity through its community culture of tipping and
	charity.
\end{itemize}

\begin{center}\rule{0.5\linewidth}{0.5pt}\end{center}

\subsection{Summary / Key Takeaways}\label{summary-key-takeaways}

This section has provided a deep dive into the advanced technical
aspects of the Bitcoin protocol, building upon the foundational
knowledge from the previous section. We have explored the evolution of
Bitcoin's scripting capabilities, from the simple yet powerful Bitcoin
Script to the sophisticated features introduced by the SegWit and
Taproot upgrades.

We have examined the various standard script types, including P2PK,
P2PKH, P2MS, and P2SH, and we have seen how they enable a wide range of
transaction types. We have also analyzed the motivations behind the
SegWit and Taproot upgrades, and we have seen how they have addressed
key challenges such as transaction malleability, scalability, and
privacy.

\begin{center}\rule{0.5\linewidth}{0.5pt}\end{center}

\subsection{Keywords}\label{keywords}

\begin{itemize}
	\tightlist
	\item
	\textbf{Bitcoin Script}: A simple, stack-based programming language
	that is used to define the conditions under which a UTXO can be spent.
	\item
	\textbf{Segregated Witness (SegWit)}: A protocol upgrade that
	addresses transaction malleability and increases the effective block
	size by separating the signature data from the main transaction data.
	\item
	\textbf{Taproot}: A protocol upgrade that enhances privacy,
	efficiency, and smart contract capabilities through the introduction
	of Schnorr signatures and Merkelized Alternative Script Trees (MAST).
	\item
	\textbf{Blockchain Fork}: A divergence in the blockchain that results
	in two or more competing chains.
	\item
	\textbf{51\% Attack}: A potential attack on a Proof-of-Work blockchain
	where a single entity controls more than 50\% of the network's mining
	hash rate.
	\item
	\textbf{P2PKH (Pay-to-Public-Key-Hash)}: The most common type of
	Bitcoin transaction, where the output is locked to the hash of a
	public key.
	\item
	\textbf{P2SH (Pay-to-Script-Hash)}: A type of Bitcoin transaction that
	allows for more complex spending conditions by locking the output to
	the hash of a script.
	\item
	\textbf{Schnorr Signatures}: A digital signature scheme that is more
	efficient and secure than ECDSA and allows for key and signature
	aggregation.
	\item
	\textbf{MAST (Merkelized Alternative Script Trees)}: A mechanism that
	allows for more complex and flexible smart contracts by enabling the
	creation of a tree of possible spending conditions.
	\item
	\textbf{Ordinals and Inscriptions}: A protocol that allows for the
	creation of NFT-like assets on the Bitcoin blockchain.
\end{itemize}

\begin{center}\rule{0.5\linewidth}{0.5pt}\end{center}

\subsection{Further Reading}\label{further-reading}

\begin{itemize}
	\tightlist
	\item
	\textbf{Bitcoin Improvement Proposals (BIPs)}:\\
	\url{https://github.com/bitcoin/bips}
	\item
	\textbf{Learn Me a Bitcoin}: \\ 
	\url{https://learnmeabitcoin.com/}
	\item
	\textbf{Mempool.space}: \\
	\url{https://mempool.space/}
\end{itemize}
\newpage

\section{Proof-of-Resource Protocols}\label{chapter-4-1-proof-of-resource-protocols}
This section delves into the landscape of alternative consensus
mechanisms designed to address the limitations of traditional
Proof-of-Work (PoW), such as Bitcoin that we dealt with in the previous sections. While PoW, the backbone of Bitcoin, is lauded for
its security, it faces significant criticism for its immense energy
consumption, the centralization of mining power through
Application-Specific Integrated Circuits (ASICs), and the inherently
``useless'' nature of its computations.

Here, we will explore the broader category of \textbf{Proof-of-Resource (PoR)} protocols (also encompassing PoW), which leverage alternative resources like memory and storage
to achieve consensus. The central goal is to design mining processes
that are more equitable and energy-efficient. % and resistant to the centralization that  PoW. 
We will examine memory-hard functions
like \textbf{scrypt} and \textbf{Argon2}, investigate the concept of Proof-of-Useful-Work
through projects like \textbf{PrimeCoin}, and dissect the various forms of
\textbf{Proof-of-Storage}, including \textbf{Proof-of-Retrievability} and
\textbf{Proof-of-Replication}, as seen in protocols like Permacoin.

\subsection{Learning Objectives}\label{learning-objectives}

\begin{itemize}
	\tightlist
	\item
	Understand the fundamental drawbacks of Proof-of-Work (PoW) and the
	primary motivations for developing alternative consensus protocols.
	\item
	Identify and explain the essential requirements for a robust
	Proof-of-Resource (PoR) protocol, including fast verification,
	adjustable difficulty, fairness, and memorylessness.
	\item
	Analyze the concept of ASIC resistance and its critical role in
	fostering decentralization.
	\item
	Differentiate between memory-hard and memory-bound mining and
	understand their application in algorithms like scrypt and Argon2.
	\item
	Explore the potential of Proof-of-Useful-Work to generate value beyond
	network security.
	\item
	Distinguish between different Proof-of-Storage models, such as
	Proof-of-Retrievability, Proof-of-Replication, Proof-of-Space, and
	Proof-of-Spacetime.
	\item
	Gain familiarity with influential altcoins that pioneered these
	alternative consensus mechanisms.
\end{itemize}

\begin{center}\rule{0.5\linewidth}{0.5pt}\end{center}

\subsection{PoW vs. PoR}\label{section-1-the-case-against-proof-of-work}

\subsubsection{Limitations of PoW}\label{limitations-of-pow}

The Proof-of-Work model, while foundational to blockchain technology,
presents several significant challenges:

\begin{itemize}
	\tightlist
	\item
	\textbf{Extreme Energy Consumption}: The computational arms race in
	PoW mining has led to an energy footprint comparable to that of entire
	countries. 
	\ih{Put example from PoUW paper.}
	\item
	\textbf{Centralization via ASICs}: The efficiency of ASICs has
	rendered mining with commodity hardware (like CPUs and GPUs)
	unprofitable, concentrating power in the hands of a few large-scale
	mining operators / centralized mining pools. This undermines the core principle of decentralization.
	\item
	\textbf{Wasted Computational Effort}: The hashing computations in PoW
	are performed solely to secure the network and have no external value,
	representing a massive expenditure of resources on problems with no
	intrinsic utility.
\end{itemize}

These issues have spurred a wave of innovation aimed at creating more
sustainable, decentralized, and efficient consensus mechanisms.

\subsubsection{Core Requirements for Proof-of-Resource (PoR)
	Protocols}\label{core-requirements-for-proof-of-resource-por-protocols}

Any viable alternative to PoW must satisfy a set of core principles to
ensure the integrity and functionality of the blockchain:

\begin{itemize}
	\tightlist
	\item
	\textbf{Fast Verification}: While finding a solution may be difficult,
	verifying its correctness must be computationally trivial for all
	nodes.
	\item
	\textbf{Adjustable Difficulty}: The protocol must be able to
	dynamically adjust the difficulty of the puzzle to maintain a
	consistent block time, regardless of fluctuations in total network
	resource commitment.
	\item
	\textbf{Fairness}: The probability of successfully mining a block
	should be directly proportional to the amount of the specified
	resource a miner contributes.
	\item
	\textbf{Independence of Solutions}: Finding one solution should not
	provide any information or advantage in finding subsequent solutions.
	\item
	\textbf{Memorylessness}: The time it takes to find a solution should
	follow an exponential distribution, meaning the probability of finding
	a solution is independent of the time already spent searching. This
	ensures that even miners with fewer resources have a chance to
	succeed.
\end{itemize}

\begin{center}\rule{0.5\linewidth}{0.5pt}\end{center}

\subsection{Memory-Hard Mining and ASIC
	Resistance}\label{section-2-memory-hard-mining-and-asic-resistance}

A primary objective in designing alternative PoR protocols is achieving
\textbf{ASIC resistance}. The idea is to create a puzzle that is
difficult and expensive to implement in specialized hardware, thereby
allowing commodity hardware like CPUs and GPUs to remain competitive.
This fosters greater decentralization by lowering the barrier to entry
for miners.

\subsubsection{Memory-Hard and Memory-Bound
	Functions}\label{memory-hard-and-memory-bound-functions}

Two key concepts in achieving ASIC resistance are:

\begin{itemize}
	\tightlist
	\item
	\textbf{Memory-Hard Mining}: This approach makes the mining process
	dependent on having a large amount of memory (RAM). The difficulty is
	tied to the \emph{quantity} of memory available.
	\item
	\textbf{Memory-Bound Mining}: This focuses on the \emph{speed} of
	memory access. The mining algorithm is designed such that performance
	is limited by how quickly data can be read from and written to memory.
\end{itemize}

Combining these two concepts makes building cost-effective ASICs
significantly more challenging, as high-capacity, high-speed memory is
expensive to integrate into custom chips.


\begin{figure}[t]
	\centering
	\includegraphics[width=0.7\textwidth]{./figs/scrypt.png}
	\caption{The scrypt algorithm, illustrating the time-memory trade-off.}\label{fig:scrypt}
\end{figure}

\subsubsection{Scrypt}\label{scrypt}

Scrypt \ih{todo citation} is a \textbf{password-based key derivation function (PBKDF)} designed in
2009 and later adopted as an IETF standard (RFC 7914). Unlike other
PBKDFs, it was explicitly designed to have high memory requirements to
defend against brute-force attacks.

The algorithm generates a large, pseudo-random vector of data that must
be held in RAM for efficient computation (see \autoref{fig:scrypt}). Accessing elements of this
vector in a pseudo-random order is core to the algorithm. This creates a
\textbf{time-memory trade-off}:

\begin{itemize}
	\tightlist
	\item
	\textbf{With sufficient RAM}: The algorithm has a linear time
	complexity, O(N).
	\item
	\textbf{Without sufficient RAM}: The elements must be re-generated
	on-the-fly, leading to a quadratic time complexity, O(N²), making the
	process prohibitively slow.
\end{itemize}



While adopted by cryptocurrencies like Litecoin and Dogecoin to resist
ASICs, specialized scrypt ASICs were eventually developed, demonstrating
the persistent difficulty of achieving long-term ASIC resistance.

\subsubsection{Argon2}\label{argon2}

Argon2 \ih{citation}, the winner of the 2015 Password Hashing Competition, is a more
modern and robust memory-hard function. It is optimized for modern CPU
architectures and their cache/memory systems.

Based on the password $P$ and salt $S$, Argon2 fills a large memory array with compression function $G$:
\begin{eqnarray}
	B[0] &=& H(P, S) \\
	for ~~~j~ &=&~ [1,\ldots, t] \\
	&&B[j] = G(B[\phi_1(j)], B[\phi_2(j)], \ldots, B[\phi_k(j	)]),
\end{eqnarray}
where $\phi$ represents indexing function.
Argon2 and has two main variants, based on the indexing function $\phi$:
\begin{itemize}
	\tightlist
	\item
	\textbf{Argon2d}: Uses \textbf{data-dependent} memory access. It's
	faster but vulnerable to side-channel attacks, making it suitable for
	applications like cryptocurrencies where the ``password'' (the block
	header and nonce) is public.
	\item
	\textbf{Argon2i}: Uses \textbf{data-independent} memory access. It's
	slower but resistant to side-channel attacks, making it ideal for
	traditional password hashing and password-based key derivation functions.
\end{itemize}
See the sequential and parallel mode of operation of Argon2 in \autoref{fig:argon2}.


\begin{figure}
	\centering
	\includegraphics[width=0.9\textwidth]{./figs/argon2.png}
	\caption{The parallel and sequential modes of the Argon2 algorithm.}\label{fig:argon2}
\end{figure}

\begin{center}\rule{0.5\linewidth}{0.5pt}\end{center}

\subsection{Proof-of-Useful-Work}\label{section-3-proof-of-useful-work}

The concept of Proof-of-Useful-Work (PoUW) aims to harness the immense
computational power of a blockchain network for tasks that have
intrinsic value beyond securing the ledger. Instead of solving arbitrary
puzzles, miners contribute to solving real-world scientific or
mathematical problems.

A notable example is \textbf{PrimeCoin}\ih{citation}, which uses its mining process
to search for special sequences of prime numbers known as
\textbf{Cunningham chains}. While these chains have applications in
number theory, a significant challenge in PoUW is integrating a secure
and fair authentication mechanism to prove \emph{who} found the
solution, a step that is not inherent in many scientific computations, which keeps this proposal flawed.

In particular, the algorithm of mining requires three parameters $m, n, k$.
For the previous block hash $x$, take $m$ bits of $x$ and consider any $k$-long chain in which the first prime in the chain is an $n$-bit prime with $m$ leading bits (i.e., the same prefix as $x$).

Found several world records that are depicted in \autoref{fig:primecoin}.

\begin{figure}
	\centering
	\includegraphics[width=0.4\textwidth]{./figs/primecoin.png}
	\caption{World records of Cunningham chains found by Primecoin.}\label{fig:primecoin}
\end{figure}


\begin{center}\rule{0.5\linewidth}{0.5pt}\end{center}

\subsection{Proof-of-Storage}\label{section-4-proof-of-storage}

Proof-of-Storage (PoS) protocols shift the resource requirement from
computation to data storage. Miners must prove they are dedicating
storage space to the network.

\subsubsection{Key Concepts in
	Proof-of-Storage}\label{key-concepts-in-proof-of-storage}

\begin{itemize}
	\tightlist
	\item
	\textbf{Proof-of-Retrievability (PoRet)}: A prover demonstrates to a
	verifier that they possess a specific file and can correctly retrieve
	it. The proofs often leak small pieces of the data, allowing for
	eventual reconstruction.
	\item
	\textbf{Proof-of-Replication (PoRep)}: A more robust form that ensures
	a prover is storing a \emph{unique physical copy} of the data,
	preventing them from pretending to store multiple copies while only
	holding one (a form of Sybil attack).
	\item
	\textbf{Proof-of-Space (PoSpace)}: A prover convinces a verifier that
	they have allocated a certain amount of storage, where the data itself
	does not need to be useful.
	\item
	\textbf{Proof-of-Spacetime}: An extension where the prover shows they
	have used a certain amount of storage \emph{over a specific duration}.
\end{itemize}

\subsubsection{Permacoin: A Proof-of-Retrievability
	System}\label{permacoin-a-proof-of-retrievability-system}

\begin{figure}[t]
	\centering
	\includegraphics[width=0.99\textwidth]{./figs/permacoin1.png}
	\caption{A simplified ``strawman'' approach to Permacoin, which is
		vulnerable to outsourcing.}\label{fig:permacoin1}	
\end{figure}


\begin{figure}[b]
	\centering
	\includegraphics[width=0.8\textwidth]{./figs/permacoin2.png}
	\caption{The improved Permacoin lottery, which ties the puzzle to the miner's private.}\label{fig:permacoin2}	
\end{figure}

Permacoin \ih{citation} is a protocol designed to repurpose mining efforts for
long-term, distributed data archiving. The core idea is that miners
collectively store a large, valuable dataset (e.g., a web archive).

To prevent outsourcing (where a miner fetches data from another source
on-demand) and ensure fair play, Permacoin's design ties the puzzle
solution to a miner's private key and requires sequential, pseudo-random
access to the stored data. This makes it computationally expensive and
slow to generate proofs without possessing the data locally.

Differences of PermaCoin in contrast to simplified PoR are as follows:
\begin{compactitem}
	\item Verifier $V$ is entire Permacoin network.
	\item  Challenge $c[r1, ..., rk]$ generated non-interactively.
	\item Every node can act as prover $P$.
	\item $F$ is large, so miners store portions of $F$ in a distributed manner.
	\item If $F$ is useful then miners’ equipment is useful.
\end{compactitem}

\medskip
The strawman approach of Permacoin does not resolve outsourcing (i.e., no erasure encoding) and it is depicted in \autoref{fig:permacoin1}.
The improvement of PermaCoin that incorporates erasure coding to boost
data recoverability while binding private key to the puzzle solution is depicted in \autoref{fig:permacoin2}.
Such a binding would require sharing of the private key with the outsourcing party, and thus giving a full control of the earned rewards to such a party, rendering outsourcing attacks economically infeasible.






%\begin{figure}
%\centering
%\pandocbounded{\includegraphics[keepaspectratio,alt={The improved Permacoin lottery, which ties the puzzle to the miner's private key and local data.}]{../../../Input/BDA-10-Proof-of-Resource protocols-17.-4.-2025_files/Image_024.png}}
%\caption{The improved Permacoin lottery, which ties the puzzle to the
	%miner's private key and local data.}
%\end{figure}

%\begin{center}\rule{0.5\linewidth}{0.5pt}\end{center}

\begin{center}\rule{0.5\linewidth}{0.5pt}\end{center}

\subsection{Summary / Key Takeaways}\label{summary-key-takeaways}

This section provided an overview of the motivations,
designs, and challenges of Proof-of-Resource protocols. We moved from
the limitations of PoW to the requirements for viable alternatives,
focusing on the critical goal of ASIC resistance. We analyzed
memory-hard functions like scrypt and Argon2, which leverage RAM to
level the playing field for miners. We also explored the innovative
concepts of Proof-of-Useful-Work and the various forms of
Proof-of-Storage, which aim to make the resources spent on consensus
more productive. The evolution of these protocols, as seen in pioneering
altcoins, highlights the ongoing quest for a consensus mechanism that is
secure, decentralized, and sustainable.

\begin{center}\rule{0.5\linewidth}{0.5pt}\end{center}

\subsection{Keywords}\label{keywords}

\begin{itemize}
	\tightlist
	\item
	\textbf{Proof-of-Resource (PoR)}: A class of consensus algorithms
	where miners dedicate resources other than computation (e.g., memory,
	storage) to secure the network.
	\item
	\textbf{ASIC Resistance}: A design goal for mining algorithms to
	prevent them from being efficiently implemented on specialized
	hardware, thus promoting decentralization.
	\item
	\textbf{Memory-Hard Function}: A function whose computational cost is
	dominated by the amount of memory required, not CPU speed.
	\item
	\textbf{Time-Memory Trade-off}: A computational scenario where an
	algorithm's execution time can be reduced at the cost of increased
	memory usage, and vice-versa.
	\item
	\textbf{Proof-of-Useful-Work (PoUW)}: A protocol where the
	computational work performed by miners has intrinsic value beyond
	securing the blockchain.
	\item
	\textbf{Proof-of-Storage (PoS)}: A protocol where miners prove they
	are dedicating a certain amount of digital storage.
	\item
	\textbf{Erasure Coding}: A method of data protection that transforms
	data into a longer form, allowing the original data to be recovered
	even if parts of the longer form are lost.
\end{itemize}

\begin{center}\rule{0.5\linewidth}{0.5pt}\end{center}

\subsection{Further Reading}\label{further-reading}

\begin{itemize}
	\tightlist
	\item
	\textbf{scrypt: A new key derivation function}: \\
	\url{https://www.tarsnap.com/scrypt/scrypt.pdf}
	\item
	\textbf{Argon2: the memory-hard function for password hashing and
		other applications}:\\
	\url{https://github.com/P-H-C/phc-winner-argon2/blob/master/argon2-specs.pdf}
	\item
	\textbf{Permacoin: Repurposing Bitcoin Work for Long-Term Data
		Preservation}: \\
	\url{https://www.cs.umd.edu/\textasciitilde amiller/permacoin.pdf}
\end{itemize}

\newpage

\section{Ethereum v1.0 (PoW) and Smart Contracts}\label{chapter-4-Eth1}
This section provides a brief introduction to Ethereum v1.0, a
decentralized, open-source blockchain platform that has revolutionized
the field of distributed computing. While Bitcoin introduced the world
to the concept of a peer-to-peer electronic cash system, Ethereum
expanded on this vision by creating a general-purpose blockchain that
enables developers to build and deploy a wide range of decentralized
applications (DAPPs) and smart contracts.

We will begin by exploring the historical context and motivation behind
the creation of Ethereum, including the limitations of Bitcoin's
scripting language that inspired Vitalik Buterin and Gavin Wood to propose a more
flexible and expressive platform. We will then delve into the core
concepts that define the Ethereum ecosystem, such as the Ethereum
Virtual Machine (EVM), the account-based model, and the concept of gas,
which is used to meter the computational resources of the network.

A significant portion of this section will be dedicated to understanding
the technical underpinnings of Ethereum, including the structure of
transactions and blocks, and the use of Merkle-Patricia trees to
efficiently store and verify the global state of the network. By the end
of this section, you will have a solid understanding of the fundamental
principles of Ethereum and the key technological innovations that have
made it the leading platform for smart contracts and decentralized
applications.

\subsection{Learning Objectives}\label{learning-objectives}

\begin{itemize}
	\tightlist
	\item
	Understand the concept of smart contracts and their potential
	applications.
	\item
	Learn about the history and motivation behind the creation of
	Ethereum.
	\item
	Grasp the key features of the Ethereum platform, including its
	Turing-complete scripting language and the Ethereum Virtual Machine
	(EVM).
	\item
	Understand the Ethereum account model and how it differs from
	Bitcoin's UTXO model.
	\item
	Learn about the concept of gas and its role in preventing abuse of the
	network.
	\item
	Gain insight into the structure of Ethereum transactions and blocks.
	\item
	Understand the purpose and function of the Merkle-Patricia Tree in
	managing Ethereum's global state.
\end{itemize}

\begin{center}\rule{0.5\linewidth}{0.5pt}\end{center}

\subsection{The Motivation for Smart
	Contracts}\label{section-1-the-motivation-for-smart-contracts}

\subsubsection{From Legal Contracts to Smart
	Contracts}\label{from-legal-contracts-to-smart-contracts}

Traditional legal contracts, while a cornerstone of modern commerce, are
often fraught with inefficiencies. They can be slow to draft, expensive
to enforce, and subject to the ambiguities of human language. In 1994,
long before the advent of blockchain technology, computer scientist and
cryptographer Nick Szabo envisioned a more efficient and secure
alternative: the \textbf{smart contract}.

Szabo defined a smart contract as a computerized transaction protocol
that executes the terms of a contract. The primary objectives of this
concept were to:

\begin{itemize}
	\tightlist
	\item
	\textbf{Automate Contractual Obligations}: To automatically enforce
	the terms of an agreement, such as payment terms, without the need for
	manual intervention.
	\item
	\textbf{Minimize Exceptions}: To reduce the risk of both malicious and
	accidental exceptions by encoding the terms of the contract in a
	deterministic and unambiguous programming language.
	\item
	\textbf{Reduce Reliance on Trusted Intermediaries}: To minimize the
	need for trusted third parties, such as lawyers and escrow agents,
	thereby reducing transaction costs and increasing efficiency.
\end{itemize}

%A classic example illustrating the need for such automation is an
%\textbf{escrow} service. When buying a house, a significant problem is
%determining the order of operations: should the buyer pay first, or
%should the seller transfer the property title first? An escrow service,
%a trusted third party, solves this by holding the buyer's funds until
%the title transfer is complete. Smart contracts aim to replace such
%intermediaries with automated, trustless code.

\subsubsection{The Limitations of Bitcoin
	Script}\label{the-limitations-of-bitcoin-script}

The Bitcoin protocol, with its simple, stack-based scripting language,
provided the first glimpse of the potential of smart contracts. Bitcoin
Script allows for the creation of basic spending conditions, such as
multi-signature transactions and time-locked payments. However, the
language is intentionally limited in its functionality -- it is not
Turing-complete -- to prioritize security and prevent complex,
potentially malicious scripts from being executed on the network. However, a simple stack-based automaton is insufficient
for Turing-completeness; a more powerful mechanism is required.

This limitation, while a sensible design choice for a digital currency,
made it difficult to build more complex and sophisticated applications
on top of the Bitcoin blockchain. The need for a more general-purpose
blockchain platform with a Turing-complete scripting language was the
primary motivation for the creation of Ethereum.

\begin{center}\rule{0.5\linewidth}{0.5pt}\end{center}

\subsection{An Introduction to
	Ethereum}\label{section-2-an-introduction-to-ethereum}

\subsubsection{The Vision of a World
	Computer}\label{the-vision-of-a-world-computer}

Launched in 2015, Ethereum was conceived with the ambitious vision of
creating a ``world computer'' -- a single, decentralized platform capable
of running any application. This vision was a direct response to the
limitations of Bitcoin, which, while revolutionary, was primarily
designed as a peer-to-peer electronic cash system. Ethereum, in
contrast, was designed to be a general-purpose blockchain, providing
developers with the tools and flexibility to build and deploy their own
smart contracts and decentralized applications (DAPPs).
The key features that enable this vision include:

\begin{itemize}
	\tightlist
	\item
	\textbf{Quasi-Turing-complete Smart Contracts}: Ethereum's primary
	programming language, Solidity, is often described as Turing-complete,
	meaning it can compute anything that is computable. However, this is
	practically limited by the concept of ``gas'' to prevent infinite
	loops. This allows for the creation of sophisticated DAPPs that can
	automate a wide range of processes and interactions.
	\item
	\textbf{The Ethereum Virtual Machine (EVM)}: The EVM is the runtime
	environment for smart contracts on the Ethereum network. It is a
	sandboxed virtual machine that is completely isolated from the host
	node's network, filesystem, or processes, ensuring that smart
	contracts are executed in a secure and deterministic manner.
	\item
	\textbf{Decentralized Finance (DeFi)}: Ethereum's flexible and
	expressive smart contract capabilities have made it the leading
	platform for the burgeoning field of Decentralized Finance (DeFi).
	DeFi applications aim to recreate traditional financial services, such
	as lending, borrowing, and trading, in a decentralized and
	permissionless manner.
\end{itemize}

\subsubsection{The Ethereum Account
	Model}\label{the-ethereum-account-model}

A key architectural difference between Bitcoin and Ethereum is their
respective account models. While Bitcoin uses the \textbf{Unspent
	Transaction Output (UTXO)} model, Ethereum employs an
\textbf{account-balance-based model}, which is more analogous to a traditional
bank account. There are two types of accounts in Ethereum:

\begin{itemize}
	\tightlist
	\item
	\textbf{Externally Owned Accounts (EOAs)}: These accounts are
	controlled by users via their private keys. EOAs can send transactions
	to other accounts and can trigger the execution of smart contracts.
	\item
	\textbf{Contract Accounts}: These accounts are controlled by the code
	of a smart contract. A contract account is created when a smart
	contract is deployed to the blockchain, and it can only execute code
	in response to a transaction or a message call from another account.
\end{itemize}

Each Ethereum account is associated with a unique 20B-long address and maintains
a state that includes its balance of Ether (ETH), the native
cryptocurrency of the Ethereum network, as well as a \textbf{nonce} (a
transaction counter to prevent replay attacks), a \textbf{storage root},
and a \textbf{code hash}.
Nonce field has context-based interpretation and based on the account type.
In the case of EOA account, the value of nonce represents the number of transactions sent from this account while in the case of smart contract account, the nonce value represents the number of smart contract creations made by tbe smart contract account.
Note that the code hash stands for the cryptographically hashed EVM bytecode of the smart contract account, while the code fragments are stored in a dedicated state database, under their hashes -- this is to efficiently optimize the storage for the same codes. The code hash field is immutable (unlike other fields).

\subsubsection{The concept of Gas}\label{gas-and-the-ethereum-virtual-machine-evm}

To prevent the network from being bogged down by infinite loops or other
forms of computational abuse, Ethereum incorporates a mechanism known as
\textbf{gas}. Gas is a unit of measurement for the computational effort
required to execute operations on the EVM. Every operation, from a
simple addition to a complex storage write, has a specific gas cost.

When a user sends a transaction, they must specify a \textbf{gas limit},
which is the maximum amount of gas they are willing to pay for the
transaction. They must also specify a \textbf{gas price}, which is the
amount of Ether they are willing to pay per unit of gas. The total cost
of the transaction is the product of the gas used and the gas price.
This system ensures that the network's computational resources are
allocated efficiently and provides a mechanism for compensating miners
for their work. If a transaction runs out of gas, its operations are
reverted, but the fee is still paid to the miner.

The EVM itself is a stack-based virtual machine with a 256-bit word
size. It provides a secure and deterministic environment for executing
smart contracts, ensuring that all nodes in the network will arrive at
the same result when executing the same code.

\paragraph{\textbf{Gas Price.}}
Every instruction of the EVM has a different gas price, mostly depending on whether it modifies storage or works only with memory. 
In general, the operations with volatile memory are cheap while permanent modification to storage are expensive.
See the example of gas prices of various instructions in \autoref{fig:eth-gas-price}, as introduced by the original version of the Ethereum Yellow paper~\cite{wood2014ethereum}. Note that the transfer transactions costs 21k of gas, which is often used as a baseline for comparison with other smart contract calls.


\begin{figure}[t]
	%	\vspace{-0.3cm}
	\begin{center}
		\includegraphics[width=0.9\textwidth]{./figs/gas-price.png}
		\caption{Gas prices of some EVM instructions.}		
		\label{fig:eth-gas-price}
	\end{center}	
\end{figure}

\paragraph{\textbf{Execution of Transaction Calling Smart Contract.}}
The simple pseudocode of deducting gas while doing transaction execution in EVM is described in the following listing:
\begin{enumerate}
	\tightlist
	\item Specify startGas and gasPrice
	\item  Assert(startGas*gasPrice $\leq$ balanceOfCaller)
	\item  balanceOfCaller -= startGas*gasPrice
	\item tmpGas = startGas ; gasRefund = 0
	\item Execute code, deducting from tmpGas
	\begin{itemize}
		\item deducted = ExecutionCost(tx)
		\item \textbf{If} deducted $<$ 0 \textbf{then} \\
		\hspace{0.5cm} ~~~~gasRefund += $|$deducted$|$ \\
		\textbf{else} \\
		\hspace{0.5cm} ~~~~tmpGas -= deducted \\
		\item Assert(tmpGas$\geq$ 0)
	\end{itemize}
	
	\item Refund gasRefund + tmpGas to balanceOfCaller	
\end{enumerate}



\subsubsection{EIP-1559}\label{gas-and-the-ethereum-virtual-machine-evm}
The EIPT-1559 \ih{ref} was included in London's hardfork that occurred on 5th of August 2021, and its main contribution is to optimize the fee market system.
In particular, this EIP introduces 3 elements of gas:
\begin{itemize}
	\item \textbf{Base fee} -- the minimum amount of gas for a transaction, which is burnt. It is set by the network. Blocks can be extended to 200\% of the base fee.
	
	\item \textbf{Tip} -- is set by the user, and it represents a premium to miners that should prioritize the faster inclusion of their transaction.
	
	\item \textbf{Fee cap} -- represents the the maximum amount of gas a user is willing to pay. Therefore, the refund is computed as  fee cap – (base fee + tip).	
\end{itemize}
This EIP improves economics of Ethereum since base we must always be ``payed'' in ETH, while before it could be paid in any currency or token of a the exchange with the full node that handled the fee internally.

\begin{center}\rule{0.5\linewidth}{0.5pt}\end{center}

\subsection{Ethereum Transactions and
	Blocks}\label{section-3-ethereum-transactions-and-blocks}

\subsubsection{Transactions}\label{transactions}

An Ethereum transaction is a cryptographically signed message that is
broadcast to the network and recorded on the blockchain. There are two
fundamental types of transactions in Ethereum:

\begin{enumerate}
	\tightlist
	\item
	\textbf{Contract Creation Transactions}: These transactions are used
	to deploy new smart contracts to the Ethereum blockchain. The
	transaction's data field contains the compiled bytecode of the smart
	contract, which is then executed by the EVM to create the new contract
	account.
	\item
	\textbf{Message Calls}: These transactions are used to interact with
	existing accounts. This can involve transferring Ether from one
	account to another, or it can involve calling a function on a smart
	contract.
\end{enumerate}

Every Ethereum transaction contains several important fields:

\begin{itemize}
	\tightlist
	\item
	\textbf{Nonce}: A sequence number that is used to prevent replay
	attacks.
	\item
	\textbf{Gas Price}: The price per unit of gas that the sender is
	willing to pay.
	\item
	\textbf{Gas Limit}: The maximum amount of gas that the sender is
	willing to use for the transaction.
	\item
	\textbf{To}: The address of the recipient account.
	\item
	\textbf{Value}: The amount of Ether to be transferred.
	\item
	\textbf{Data}: The input data for a message call, or the compiled
	bytecode for a contract creation transaction.
	\item
	\textbf{Signature}: A digital signature that authenticates the sender.
\end{itemize}

\subsubsection{Blocks}\label{blocks}

An Ethereum block is a collection of transactions that are bundled
together and added to the blockchain. Each block consists of a header
and a body. The body contains a list of all the transactions included in
the block, as well as a list of ``\textbf{ommers},'' which are the
headers of stale blocks that were not included in the main chain but are
still rewarded to incentivize miners and improve security.
The block header (of legacy Ethereum v1.0) contains several crucial pieces of information (see also \autoref{fig:eth-eth-header}):

\begin{figure}[t]
	%	\vspace{-0.3cm}
	\begin{center}
		\includegraphics[width=0.7\textwidth]{./figs/eth-block.png}
		\caption{A structure of the header in Ethereum v1.0.}		
		\label{fig:eth-eth-header}
	\end{center}	
\end{figure}


\begin{itemize}
	\tightlist
	\item
	\textbf{Parent Hash}: The hash of the previous block's header, which
	links the blocks together in a chain.
	\item
	\textbf{Block Number}: The sequence number of the block in the chain.
	\item \textbf{Nonce}: replay protection counter.
	\item \textbf{Beneficiary}: the miner of the block -- replaces a coinbase transaction known from Bitcoin.
	\item
	\textbf{Timestamp}: The time at which the block was created.
	\item \textbf{ommersHash}: aggregates stale parent block headers whose miners get a certain partial reward.	
	\item
	\textbf{Difficulty}: The difficulty of the Proof-of-Work puzzle for
	the block.
	\item
	\textbf{Gas Limit and Gas Used}: The maximum amount of gas allowed in
	the block and the total amount of gas used by all the transactions in
	the block.
	\item
	\textbf{State Root, Transaction Root, and Receipt Root}: The root
	hashes of three Merkle-Patricia trees that store the global state, the
	transactions, and the transaction receipts, respectively.
	\item \textbf{ExtraData}: arbitrary data of max. 32B. (e.g., name of the mining pool)
	\item
	\textbf{Logs Bloom}: A Bloom filter that provides an efficient way to
	search for event logs generated by smart contracts within the block.
\end{itemize}

\subsubsection{The Global State and Merkle-Patricia
	Trees}\label{the-global-state-and-merkle-patricia-trees}

The \textbf{global state} of Ethereum is a massive data structure that
contains a mapping of all account addresses to their corresponding
account states. This state is not stored in the blockchain itself, but
rather in a separate key-value database that is maintained by all full
nodes.
The example of a transaction that calls smart contract account and thus changes the state is depicted in \autoref{fig:eth-state-change}.


\begin{figure}[t]
	%	\vspace{-0.3cm}
	\begin{center}
		\includegraphics[width=0.7\textwidth]{./figs/eth-state-change.png}
		\caption{A transaction call that changes the state of the smart contract and its value.}		
		\label{fig:eth-state-change}
	\end{center}	
\end{figure}


To efficiently store and verify the global state, Ethereum uses a
sophisticated data structure called a \textbf{Merkle-Patricia Tree
	(MPT)}. The MPT is a hybrid data structure that combines the properties
of a Merkle tree (for integrity) and a Radix/Patricia trie (for
efficient lookups). This allows for efficient verification of the state
and enables light clients to securely query the state of an account
without having to download the entire blockchain. The root of the MPT,
known as the \textbf{state root}, is included in the block header, which
provides a cryptographic commitment to the entire state of the network
at that point in time.
MPT contains 3 types of nodes:
\begin{compactenum}
	\item \textbf{Extension nodes}: aggregate symbols in the key.
	\item \textbf{Branch nodes}: splits the path according to the current symbol in the key.
	\item \textbf{Leaf nodes}: they store data of the account states (or storages).	
\end{compactenum}
The example of ste state trie built using MPT is depicted in \autoref{fig:eth-mpt}, and it contains 4 accounts.

\begin{figure}[t]
	%	\vspace{-0.3cm}
	\begin{center}
		\includegraphics[width=0.95\textwidth]{./figs/eth-mpt.png}
		\caption{A Merkle-Patricia Trie for storing the global state in Ethereum.}		
		\label{fig:eth-mpt}
	\end{center}	
\end{figure}



\subsubsection{Bloom Filters and Transaction
	Receipts}\label{bloom-filters-and-transaction-receipts}

For every transaction, Ethereum generates a \textbf{receipt}, which
contains information about the outcome of the transaction, including the
gas used and, importantly, any \textbf{logs} or events emitted by the
smart contract during its execution.

To allow for efficient searching of these logs without requiring nodes
to scan every transaction in every block, Ethereum uses \textbf{Bloom
	filters}. A Bloom filter is a space-efficient probabilistic data
structure that can quickly test whether an element is a member of a set.
While it can produce false positives, it will never produce a false
negative.

Bloom filter is an $m$ bit array that requires $k$ different (cryptographically insecure) hash functions, where $k << m$. 
Bloom filter supports 2 operations:
\begin{compactenum}
	\item \textbf{Add element} -- we hash new element with k hash functions and set 1 on obtained positions.
	\item \textbf{Query element} -- we hash queried element with k hash functions and if 1 is on all positions, return true. Otherwise, return false (see \autoref{fig:eth-bloom-filter}).
\end{compactenum}
Bloom filter is often use as a pre-filter for expensive storage-lookup queries (see \autoref{fig:bloom-app}).

\begin{figure}[t]
	%	\vspace{-0.3cm}
	\begin{center}
		\includegraphics[width=0.9\textwidth]{./figs/bloom.png}
		\caption{An example of the Bloom filter with 3 hashing functions  and 3 elements. Element w in query is not included in the Bloom filter.}		
		\label{fig:eth-bloom-filter}
	\end{center}	
\end{figure}


\begin{figure}[t]
	%	\vspace{-0.3cm}
	\begin{center}
		\includegraphics[width=0.7\textwidth]{./figs/bloom-eg.png}
		\caption{Application of Bloom filter for pre-filtering of queries.}		
		\label{fig:bloom-app}
	\end{center}	
\end{figure}



In Ethereum, each transaction receipt contains a Bloom filter of its logs, and the
block header contains a cumulative Bloom filter that aggregates the
filters from all transaction receipts in the block. This allows DAPPs
and light clients to quickly and efficiently find relevant events by
first checking the Bloom filter in the block header.




\begin{center}\rule{0.5\linewidth}{0.5pt}\end{center}

\subsection{Summary / Key Takeaways}\label{summary-key-takeaways}

This section has provided a comprehensive introduction to Ethereum, the
first and currently the most adopted platform for decentralized applications and smart
contracts. We have explored the historical context and motivation behind
its creation, highlighting the limitations of Bitcoin's scripting
language that spurred the development of a more general-purpose
blockchain.

We have delved into the core concepts that define the Ethereum
ecosystem, including:

\begin{itemize}
	\tightlist
	\item
	\textbf{Smart Contracts}: Self-executing contracts with the terms of
	the agreement directly written into code, enabling automation and
	reducing the need for intermediaries.
	\item
	\textbf{The Ethereum Virtual Machine (EVM)}: The sandboxed,
	deterministic runtime environment for smart contracts on the Ethereum
	network.
	\item
	\textbf{The Account-Based Model}: Ethereum's approach to tracking the
	state of the network, which is more intuitive for developers than
	Bitcoin's UTXO model.
	\item
	\textbf{Gas}: The mechanism for metering the computational resources
	of the network, preventing abuse and compensating miners.
	\item
	\textbf{Merkle-Patricia Trees}: The sophisticated data structure used
	to efficiently and securely store the global state of all accounts.
	\item
	\textbf{Bloom Filters}: A probabilistic data structure used to make
	searching for smart contract events highly efficient.
\end{itemize}


\begin{center}\rule{0.5\linewidth}{0.5pt}\end{center}

\subsection{Keywords}\label{keywords}

\begin{itemize}
	\tightlist
	\item
	\textbf{Ethereum}: A decentralized, open-source blockchain platform
	that enables the creation and deployment of smart contracts and
	decentralized applications (DAPPs).
	\item
	\textbf{Smart Contract}: A computer protocol intended to digitally
	facilitate, verify, or enforce the negotiation or performance of a
	contract.
	\item
	\textbf{Ethereum Virtual Machine (EVM)}: The runtime environment for
	smart contracts in Ethereum. It is a quasi-Turing-complete virtual
	machine that executes code as a sandboxed process.
	\item
	\textbf{Gas}: The unit of measurement for the computational effort
	required to execute operations on the Ethereum network.
	\item
	\textbf{Account Model}: An account model used in Ethereum where the
	state of the blockchain is represented as a set of accounts, each with
	its own balance, storage, and code.
	\item
	\textbf{Merkle-Patricia Tree (MPT)}: A hybrid data structure that
	combines a Merkle tree and a Patricia trie to efficiently store and
	verify the global state of the Ethereum network.
	\item
	\textbf{Externally Owned Account (EOA)}: An Ethereum account that is
	controlled by a user's private key.
	\item
	\textbf{Contract Account}: An Ethereum account that is controlled by
	the code of a smart contract.
	\item
	\textbf{Decentralized Finance (DeFi)}: A new financial system built on
	public blockchains that provides an alternative to traditional
	financial services.
	\item
	\textbf{Bloom Filter}: A space-efficient probabilistic data structure
	used in Ethereum to facilitate fast queries for log events.
\end{itemize}

\begin{center}\rule{0.5\linewidth}{0.5pt}\end{center}

\subsection{Further Reading}\label{further-reading}

\begin{itemize}
	\tightlist
	\item
	\textbf{Ethereum White Paper}: \\ 
	\url{https://ethereum.org/en/whitepaper/}
	\item
	\textbf{Ethereum Yellow Paper}: \\
	\url{https://ethereum.github.io/yellowpaper/paper.pdf}
	\item
	\textbf{Mastering Ethereum by Andreas M. Antonopoulos and Gavin Wood}: \\
	\url{https://github.com/ethereumbook/ethereumbook}
\end{itemize}

\newpage

\section{Smart Contract Programming}\label{chapter-6-smart-contract-programming}
This section provides a practical introduction to the smart contract programming on the Ethereum blockchain.
Building upon the theoretical foundations laid in the previous section,
we will now transition to the practical aspects of designing,
developing, and deploying smart contracts. The primary focus of this
section will be on Solidity, the most widely adopted programming
language for writing smart contracts on the Ethereum platform.

We will begin by exploring the fundamental elements of the Solidity
language, including its syntax, data types, and control structures. We
will then move on to more advanced topics, such as the use of function
modifiers to create reusable code, the importance of visibility
specifiers for controlling access to functions and state variables, and
the use of libraries to promote code reuse and modularity.

A significant portion of this section will be dedicated to the practical
application of these concepts. We will examine the standards for
creating both fungible (ERC20) and non-fungible (ERC721) tokens, which
are two of the most common use cases for smart contracts on Ethereum.

Finally, we will address the critical topic of smart contract security.
We will discuss common vulnerabilities, such as reentrancy attacks and
integer overflows, and we will explore the best practices and tools that
developers can use to write secure and robust smart contracts. By the
end of this section, you will have the foundational knowledge and
practical skills necessary to begin your journey as a smart contract
developer.

\subsection{Learning Objectives}\label{learning-objectives}

\begin{itemize}
	\tightlist
	\item
	Understand the basics of the Solidity programming language.
	\item
	Learn how to define and interact with smart contracts.
	\item
	Grasp the concepts of visibility specifiers, function modifiers, and
	data locations.
	\item
	Understand the role of fallback and receive functions in handling
	Ether transfers.
	\item
	Learn about the standards for fungible (ERC20) and non-fungible
	(ERC721) tokens.
	\item
	Gain insight into common smart contract vulnerabilities and security
	best practices.
	\item
	Become familiar with the tools and techniques used for smart contract
	development and analysis.
\end{itemize}

\begin{center}\rule{0.5\linewidth}{0.5pt}\end{center}

\subsection{Introduction to
	Solidity}\label{section-1-introduction-to-solidity}

\subsubsection{What is Solidity?}\label{what-is-solidity}

Solidity is a high-level, object-oriented programming language that has
become the de facto standard for writing smart contracts on the Ethereum
blockchain and other EVM-compatible platforms. It is a statically-typed
language with a syntax that is similar to that of JavaScript and C++,
making it relatively easy for developers with experience in these
languages to learn.

The primary purpose of Solidity is to provide a means for developers to
create self-executing contracts that can be deployed to a blockchain.
These contracts can be used to automate a wide range of processes, from
simple token transfers to complex financial instruments.
The users interact with smart contracts through decentralized applications (DAPPs), which run completely server-less -- on the client side (see \autoref{fig:dapps}).


\begin{figure}[t]
	%	\vspace{-0.3cm}
	\begin{center}
		\includegraphics[width=0.8\textwidth]{./figs/smart-contracts-flow.png}
		\caption{Users interacting with smart contract through DAPPs (that run on the client-side).}		
		\label{fig:dapps}
	\end{center}	
\end{figure}
  

\subsubsection{The Structure of a Solidity
	Contract}\label{the-structure-of-a-solidity-contract}

A Solidity contract is a collection of code and data that is deployed to
a specific address on the blockchain. It is analogous to a class in an
object-oriented programming language, in that it encapsulates both data
(in the form of state variables) and behavior (in the form of
functions).

A typical Solidity source file has the following structure:

\begin{itemize}
	\tightlist
	\item
	\textbf{Pragma Directive}: This is a declaration that specifies the
	version of the Solidity compiler that should be used to compile the
	code. This is important for ensuring that the code is compiled
	correctly and that it is compatible with the target version of the
	EVM.
	\item
	\textbf{License Identifier}: This is a comment that specifies the
	license under which the code is released. This is important for legal
	and ethical reasons, as it informs other developers of the terms under
	which they can use and modify the code.
	\item
	\textbf{Contract Definition}: This is the main body of the contract,
	which contains the state variables, functions, events, and modifiers
	that define the contract's behavior.
\end{itemize}
See an example of a smart contract in \autoref{fig:smart-contract}.


\begin{figure}[t]
	%	\vspace{-0.3cm}
	\begin{center}
		\includegraphics[width=0.5\textwidth]{./figs/smart-contract-example.png}
		\caption{Example of simple smart contract written in Solidity.}		
		\label{fig:smart-contract}
	\end{center}	
\end{figure}


\subsubsection{Data Types and
	Variables}\label{data-types-and-variables}

Solidity supports a rich set of data types, which can be broadly
classified into two categories:

\begin{itemize}
	\tightlist
	\item
	\textbf{Value Types}: These are data types that are passed by value,
	meaning that a copy of the value is created when it is assigned to a
	new variable or passed as an argument to a function. Value types
	include \texttt{bool}, \texttt{int}, \texttt{uint}, \texttt{address},
	and \texttt{bytes}.
	\item
	\textbf{Reference Types}: These are data types that are passed by
	reference, meaning that the new variable or function argument refers
	to the same location in memory as the original. Reference types
	include \texttt{arrays}, \texttt{structs}, and \texttt{mappings}.
\end{itemize}

Variables in Solidity can be stored in one of three data locations:

\begin{itemize}
	\tightlist
	\item
	\textbf{Storage}: This is the persistent storage of the contract,
	which is stored on the blockchain. State variables are stored in
	storage by default. It is the most expensive place in terms of gas since it is permanent.
	\item
	\textbf{Memory}: This is a temporary storage location and is reset between separate external calls and transactions, and it is not persistent across transactions.
	It is used to store local variables and
	function arguments. In general, the memory is not permanent storage and thus operations with memory are cheap in terms of gas.
	\item
	\textbf{Calldata}: This is a read-only storage location that is used
	to store the input data for a function call. In fact, calldata location is also stored in memory and thus has the same cost implications in terms of gas.
\end{itemize}

\begin{center}\rule{0.5\linewidth}{0.5pt}\end{center}

\subsection{Functions and Control
	Structures}\label{section-2-functions-and-control-structures}

\subsubsection{Functions}\label{functions}

Functions are the fundamental units of execution in a Solidity smart
contract. They encapsulate the logic of the contract and can be called
either internally by other functions within the same contract or
externally by other contracts or users.

Solidity provides four visibility specifiers for functions, which
control how they can be accessed:

\begin{itemize}
	\tightlist
	\item
	\textbf{\texttt{public}}: Public functions can be called from
	anywhere, both internally and externally.
	\item
	\textbf{\texttt{private}}: Private functions can only be called from
	within the contract in which they are defined.
	\item
	\textbf{\texttt{internal}}: Internal functions can be called from
	within the contract in which they are defined and from any contracts
	that inherit from it.
	\item
	\textbf{\texttt{external}}: External functions can only be called from
	outside the contract.
\end{itemize}

In addition to visibility specifiers, functions can also have modifiers
that alter their behavior. The most common modifiers are:

\begin{itemize}
	\tightlist
	\item
	\textbf{\texttt{view}}: This modifier indicates that the function is
	read-only and does not modify the state of the contract.
	\item
	\textbf{\texttt{pure}}: This modifier indicates that the function does
	not read or modify the state of the contract.
	\item
	\textbf{\texttt{payable}}: This modifier indicates that the function
	can receive Ether.
\end{itemize}

Note that Solidity provides several globally available variables and functions that are specific either to the current transaction or the block in which the transaction is included. 
The most important such functions and variables are presented in \autoref{fig:special-vars} and \autoref{fig:special-vars2}.

\begin{figure}[t]
	%	\vspace{-0.3cm}
	\begin{center}
		\includegraphics[width=0.9\textwidth]{./figs/vars-funcs.png}
		\caption{Globally available variables and functions.}		
		\label{fig:special-vars}
	\end{center}	
\end{figure}

\begin{figure}[t]
	%	\vspace{-0.3cm}
	\begin{center}
		\includegraphics[width=0.99\textwidth]{./figs/functions2.png}
		\caption{Globally available variables and functions.}		
		\label{fig:special-vars2}
	\end{center}	
\end{figure}

\subsubsection{Control Structures}\label{control-structures}

Solidity supports a range of standard control structures that are common
to many programming languages, including:

\begin{itemize}
	\tightlist
	\item
	\textbf{Conditional statements}: \texttt{if}, \texttt{else}
	\item
	\textbf{Loops}: \texttt{while}, \texttt{for}, \texttt{do-while}
\end{itemize}

Solidity also provides three error-handling mechanisms that can be used
to revert the state of the contract if an error occurs:

\begin{itemize}
	\tightlist
	\item
	\textbf{\texttt{require}}: This function is used to validate inputs
	and conditions before executing a function. If the condition is not
	met, the transaction is reverted.
	\item
	\textbf{\texttt{assert}}: This function is used to check for internal
	errors and to validate invariants. If the condition is not met, the
	transaction is reverted.
	\item
	\textbf{\texttt{revert}}: This function is used to revert the
	transaction and provide an error message.
\end{itemize}

\subsubsection{\texorpdfstring{Special Functions: \texttt{receive}
		and
		\texttt{fallback}}{Special Functions: receive and fallback}}\label{special-functions-receive-and-fallback}

Solidity includes two special functions that are used to handle Ether
transfers and calls to non-existent functions:

\begin{itemize}
	\tightlist
	\item
	\textbf{\texttt{receive} function}: This function is executed when a
	contract receives a plain Ether transfer without any accompanying
	data. It must be declared as \texttt{external} and \texttt{payable}.
	\item
	\textbf{\texttt{fallback} function}: This function is executed when a
	contract receives a call to a function that does not exist in the
	contract. It can also be used to receive Ether if no \texttt{receive}
	function is defined.
\end{itemize}

These functions are essential for creating contracts that can interact
with the Ethereum network in a flexible and robust manner.


\subsubsection{Custom function modifiers}\label{sec:modifiers}
Solidity provides custom modifiers of functions that act in a similar fashion as macros in C/C++.
These custom modifiers are usually used to check authorization for calling the function or conditions to be met.
See two examples of custom function modifiers in \autoref{fig:modifiers}.

\begin{figure}[t]
	%	\vspace{-0.3cm}
	\begin{center}
		\includegraphics[width=0.9\textwidth]{./figs/modifiers-custom.png}
		\caption{Examples of custom function modifiers.}		
		\label{fig:modifiers}
	\end{center}	
\end{figure}

\subsubsection{Abstract contracts and interfaces}\label{sec:interfaces}
Similarly to  many other object-oriented programming languages, Solidity also supports abstract contracts and interfaces.
Contract is marked as abstract if at least one of its functions is not implemented. In contrast, interfaces must have not any function implemented.
See examples of abstract contract and interface in \autoref{fig:inteface}. 

\begin{figure}[bt]
	%	\vspace{-0.3cm}
	\begin{center}
		\includegraphics[width=0.7\textwidth]{./figs/abstract-contract.png}
		\caption{Examples of abstract contract and interface.}		
		\label{fig:inteface}
	\end{center}	
\end{figure}



\subsubsection{Libraries}\label{sec:libs}
Libraries are similar to contracts but they are deployed only once and their code is reused using the \texttt{DELEGATECALL} instruction.
If library functions are called, their code is executed in the context of the calling contract (especially modifying the storage of the calling contract).
Therefore, libraries are assumed to be stateless.
Libraries can be also seen as implicit base contracts of the contracts that use them.
An example of the library and its application to a specific data type is depicted in \autoref{fig:library}. 
Another example of library is depicted in \autoref{fig:library2}.


\begin{figure}[bt]
	%	\vspace{-0.3cm}
	\begin{center}
		\includegraphics[width=0.99\textwidth]{./figs/library.png}
		\caption{The example of library contract and its application on a data type.}		
		\label{fig:library}
	\end{center}	
\end{figure}

\begin{figure}[bt]
	%	\vspace{-0.3cm}
	\begin{center}
		\includegraphics[width=0.8\textwidth]{./figs/library2.png}
		\caption{The example of library contract and its application using ``using'' and ``for''.}		
		\label{fig:library2}
	\end{center}	
\end{figure}



\begin{center}\rule{0.5\linewidth}{0.5pt}\end{center}

\subsection{Token Standards and Smart Contract
	Security}\label{section-3-token-standards-and-smart-contract-security}

\subsubsection{ERC20 and ERC721 Token
	Standards}\label{erc20-and-erc721-token-standards}

To promote interoperability and composability within the Ethereum
ecosystem, the community has developed a set of standards for different
types of tokens. The two most widely adopted standards are ERC20 and
ERC721.

\begin{itemize}
	\tightlist
	\item
	\textbf{ERC20}: This is the standard for fungible tokens, which are
	tokens that are identical and interchangeable. Examples of ERC20
	tokens include stablecoins like USDT and DAI, as well as governance
	tokens for various DeFi protocols. The ERC20 standard defines a common
	interface that includes functions for transferring tokens, checking
	account balances, and approving other accounts to spend tokens on
	behalf of the owner.
	The interface of ERC20 contract is depicted in \autoref{fig:erc20}.
\end{itemize}

\begin{figure}[t]
	%	\vspace{-0.3cm}
	\begin{center}
		\includegraphics[width=0.9\textwidth]{./figs/ERC20.png}
		\caption{ERC20 interface.}		
		\label{fig:erc20}
	\end{center}	
\end{figure}
\begin{figure}
	%	\vspace{-0.3cm}
	\begin{center}
		\includegraphics[width=0.9\textwidth]{./figs/ERC721.png}
		\caption{ERC721 interface.}		
		\label{fig:erc721}
	\end{center}	
\end{figure}


\begin{itemize}
	\tightlist
	\item
	\textbf{ERC721}: This is the standard for non-fungible tokens (NFTs),
	which are unique and non-interchangeable. Each ERC721 token has a
	unique ID and can be used to represent a wide range of assets, from
	digital art and collectibles to real estate and intellectual property.
	The ERC721 standard defines a common interface for tracking the
	ownership of each individual token.
	The interface of ERC721 contract is depicted in \autoref{fig:erc721}.
\end{itemize}




These token standards have been instrumental in the growth of the
Ethereum ecosystem, as they allow for seamless integration between
different applications, such as wallets, exchanges, and DeFi protocols.




\clearpage
\subsubsection{Smart Contract
	Security}\label{smart-contract-security}

The immutable nature of blockchains means that bugs and vulnerabilities
in smart contracts can have severe and irreversible consequences, often
leading to significant financial losses. Therefore, smart contract
security is of paramount importance. Some of the most common security
vulnerabilities include:

\begin{itemize}
	
	\item
	\textbf{Reentrancy Attacks}: This is a type of attack where a
	malicious contract can repeatedly call a function on a victim contract
	before the first call has completed. This can be used to drain funds
	from the victim contract if it does not follow the
	checks-effects-interactions pattern. The infamous DAO attack was a
	result of a reentrancy vulnerability (see \autoref{fig:dao-bug}).
	
		\item
	\textbf{Integer Overflows and Underflows}: These vulnerabilities occur
	when an arithmetic operation results in a value that is outside the
	range of the data type. This can lead to unexpected behavior and can
	be exploited by attackers to manipulate the state of the contract.
	It can, for example, cause out of gas exception due to comparing mismatching data types:  \verb| for (var i = 0; i < 2000; i++) { ... } |. Note that i is of type uint8 and thus will never reach the value 2000.
	
	%	\begin{figure}[t]
	%		%	\vspace{-0.3cm}
	%		\begin{center}
	%			\includegraphics[width=0.8\textwidth]{./figs/underflow.png }
	%			\caption{Out of gas exception due to integer underflow.}		
	%			\label{fig:underflow}
	%		\end{center}	
	%	\end{figure}
	
	\item
	\textbf{Denial-of-Service (DoS) Attacks}: An attacker can exploit
	certain vulnerabilities to render a contract unusable, for example, by
	causing it to run out of gas or by making it impossible for other
	users to interact with it.
\end{itemize}


\begin{figure}[t]
	%	\vspace{-0.3cm}
	\begin{center}
		\includegraphics[width=0.9\textwidth]{./figs/DAO.png}
		\caption{Reentrancy vulnerability in DAO bug.}		
		\label{fig:dao-bug}
	\end{center}	
\end{figure}


Another notable example of smart contract vulnerabilities is the Parity
Wallet bug, which led to the freezing of millions of dollars worth of
Ether. This bug was caused by a flaw in the way the wallet's library
contract was initialized, allowing an attacker to take ownership of the
library and self-destruct it.

\subsubsection{Upgradeable Smart
	Contracts}\label{upgradeable-smart-contracts}

While smart contracts are immutable by default, there are scenarios
where it is desirable to be able to modify them, for example, to fix a
bug or to add new functionality. The Proxy and Implementation pattern is
a common approach to creating upgradeable smart contracts. This pattern
involves a proxy contract that stores the state of the contract and
delegates all calls to an implementation contract (see \autoref{fig:upgradable}). The implementation
contract can be replaced with a new version without affecting the state
of the proxy contract.

\begin{figure}[t]
	%	\vspace{-0.3cm}
	\begin{center}
		\includegraphics[width=0.99\textwidth]{./figs/upgradable.png}
		\caption{Proxy and implementation pattern.}		
		\label{fig:upgradable}
	\end{center}	
\end{figure}


\subsubsection{Tools for Smart Contract Development and
	Analysis}\label{tools-for-smart-contract-development-and-analysis}

A rich ecosystem of tools has emerged to support the development and
analysis of secure and robust smart contracts. These tools can be
broadly categorized as follows:

\begin{itemize}
	\tightlist
	\item
	\textbf{Development Frameworks}: These provide a comprehensive
	environment for compiling, testing, and deploying smart contracts.
	Popular frameworks include Foundry\footnote{\url{https://getfoundry.sh}} and Hardhat.\footnote{\url{https://hardhat.org}}
	\item
	\textbf{Browser-Based IDEs}: These provide a convenient and
	user-friendly environment for writing and testing smart contracts
	directly in the browser. Remix\footnote{\url{https://remix.ethereum.org}} is the most widely used browser-based
	IDE.
	\item
	\textbf{Static Analysis Tools}: These tools analyze the source code of
	a smart contract to identify potential vulnerabilities and deviations
	from best practices. Examples include Slither\footnote{\url{https://github.com/crytic/slither}} and Solhint.\footnote{\url{https://protofire.io/solhint}}
	\item
	\textbf{Dynamic Analysis Tools}: These tools analyze the behavior of a
	smart contract as it is being executed to identify vulnerabilities
	that may not be apparent from a static analysis. Examples include
	Echidna\footnote{\url{https://github.com/crytic/echidna}} and Manticore.\footnote{\url{https://secure-contracts.com/program-analysis/manticore/}}
\end{itemize}

\begin{center}\rule{0.5\linewidth}{0.5pt}\end{center}

\subsection{Summary / Key Takeaways}\label{summary-key-takeaways}

This section has provided a practical, hands-on introduction to the
world of smart contract programming with Solidity. We have covered the
essential elements of the language, from its basic syntax and data types
to its more advanced features, such as function modifiers and visibility
specifiers.

We have explored the structure of a Solidity contract and the different
data locations that can be used to store variables. We have also
examined the role of special functions like \texttt{receive} and
\texttt{fallback} in handling Ether transfers and interacting with other
contracts.

A key focus of this section has been on the practical application of
Solidity. We have discussed the ERC20 and ERC721 token standards, which
are the foundation of the vibrant DeFi and NFT ecosystems on Ethereum.

Finally, we have addressed the critical importance of smart contract
security. We have identified common vulnerabilities, such as reentrancy
attacks and integer overflows, and we have highlighted the tools and
best practices that developers can use to write secure and robust code.

By the end of this section, you should have a solid foundation in smart
contract programming and be well-equipped to start building your own
decentralized applications on the Ethereum blockchain.

\begin{center}\rule{0.5\linewidth}{0.5pt}\end{center}

\subsection{Keywords}\label{keywords}

\begin{itemize}
	\tightlist
	\item
	\textbf{Solidity}: A high-level, object-oriented programming language
	for writing smart contracts on the Ethereum blockchain.
	\item
	\textbf{Smart Contract}: A self-executing contract with the terms of
	the agreement between buyer and seller being directly written into
	lines of code.
	\item
	\textbf{ERC20}: A technical standard used for smart contracts on the
	Ethereum blockchain for implementing tokens.
	\item
	\textbf{ERC721}: A standard for non-fungible tokens (NFTs) on the
	Ethereum blockchain.
	\item
	\textbf{Reentrancy}: A common smart contract vulnerability where a
	malicious contract can repeatedly call a function on a victim contract
	before the first call has completed.
	\item
	\textbf{Visibility Specifiers}: Keywords in Solidity that control the
	accessibility of functions and state variables.
	\item
	\textbf{Function Modifiers}: Reusable pieces of code that can be used
	to change the behavior of functions in a declarative way.
	\item
	\textbf{Truffle}: A popular development framework for Ethereum that
	provides a suite of tools for compiling, testing, and deploying smart
	contracts.
	\item
	\textbf{Hardhat}: A flexible and extensible Ethereum development
	environment that is widely used for smart contract development.
	\item
	\textbf{Remix}: A browser-based IDE for Solidity that provides a
	convenient environment for writing, testing, and deploying smart
	contracts.
\end{itemize}

\begin{center}\rule{0.5\linewidth}{0.5pt}\end{center}

\subsection{Further Reading}\label{further-reading}

\begin{itemize}
	\tightlist
	\item
	\textbf{Solidity Documentation}: \\
	\url{https://docs.soliditylang.org/}
	\item
	\textbf{OpenZeppelin Contracts}: \\
	\url{https://github.com/OpenZeppelin/openzeppelin-contracts}
	\item
	\textbf{ConsenSys Smart Contract Best Practices}: \\
	\url{https://consensys.github.io/smart-contract-best-practices/}
\end{itemize}

\newpage


\section{Proof-of-Stake Protocols and Oracles}\label{section-11-proof-of-stake-protocols}
This section provides explanation of Proof-of-Stake
(PoS) consensus mechanisms, which have emerged as a potential
alternative to the energy-intensive Proof-of-Work (PoW) systems. We will
explore the foundational principles of PoS, where the right to
validate transactions and create new blocks is granted based on the
quantity of cryptocurrency a participant holds and is willing to
``stake'' as collateral. This economic-driven approach to consensus
offers notable advantages in terms of energy efficiency and scalability,
but also introduces a unique set of security challenges and economic
considerations.

The section will begin by discussing the primary motivations for the
development of PoS, focusing on the well-documented energy consumption
issues associated with PoW. We will then delve into the core concepts of
PoS, including the roles of validators, the process of staking, and the
mechanisms for leader election. A balanced analysis of the advantages
and disadvantages of PoS will be presented, addressing its potential for
both decentralization and wealth concentration.
%
A significant portion of the section is dedicated to analysis
of the security vulnerabilities inherent in PoS protocols. We will
dissect critical issues such as the \textbf{``nothing-at-stake'' problem,
grinding attacks, DoS on leaders, and long-range attacks}, and discuss
the various countermeasures that have been proposed and implemented to
mitigate these risks.

We will then survey several prominent PoS protocols that have been
deployed in real-world blockchain systems, including \textbf{Peercoin, Algorand,
Ouroboros, and Ethereum's Casper and Gasper}. 
%
Finally, we will explore
the challenge of blockchain \textbf{oracles} and how systems can securely
access external data, covering solutions like \textbf{TownCrier, TLS-N, and
PDFS}. 
%Through this exploration, readers will gain an understanding of the theoretical underpinnings and practical implementations of Proof-of-Stake and its surrounding ecosystem.

\subsection{Learning Objectives}\label{learning-objectives}

\begin{itemize}
	\tightlist
	\item
	Understand the fundamental motivations for the development of
	Proof-of-Stake (PoS) as an alternative to Proof-of-Work (PoW).
	\item
	Grasp the core concepts of PoS, including staking, validators, leader
	election, and the role of economic incentives in securing the network.
	\item
	Analyze the advantages and disadvantages of PoS, including its energy
	efficiency, potential for scalability, and the risk of stake
	centralization.
	\item
	Identify and understand the security challenges specific to PoS, such
	as the nothing-at-stake problem, grinding attacks, DoS on leaders, and
	long-range attacks.
	\item
	Explore the design and implementation of prominent PoS protocols,
	including Peercoin, Algorand, Ouroboros, and Ethereum's Casper.
	\item
	Evaluate the various countermeasures developed to address the security
	vulnerabilities of PoS systems.
	\item
	Understand the ``oracle problem'' and evaluate different approaches
	for providing external data to smart contracts.
\end{itemize}

\begin{center}\rule{0.5\linewidth}{0.5pt}\end{center}

\subsection{Introduction to	Proof-of-Stake}\label{section-1-introduction-to-proof-of-stake}

\subsubsection{The Motivation}\label{the-motivation-for-proof-of-stake}

The primary impetus for the creation of Proof-of-Stake consensus
mechanisms was the growing concern over the immense energy consumption
of Proof-of-Work (PoW) blockchains -- see example of the Bitcoin in \autoref{fig:btc-energy-consumption} and \autoref{fig:btc-energy-footprint}.
%
%\pandocbounded{\includegraphics[keepaspectratio,alt={PoW Energy Consumption}]{../../../Input/BDA-11-Proof-of-Stake-protocols-24.-4.-2025_files/Image_004.jpg}}


\begin{figure}[t]
	%	\vspace{-0.3cm}
	\begin{center}
	\includegraphics[width=0.9\textwidth]{./figs/bitcoin-energy-consumpti.png}
	\caption{A visual representation of Bitcoin's energy consumption over time~
		\cite{statista_bitcoin_energy_consumption_2025}.}
		\label{fig:btc-energy-consumption}
	\end{center}	
\end{figure}

\begin{figure}[t]
	%	\vspace{-0.3cm}
	\begin{center}
		\includegraphics[width=0.8\textwidth]{./figs/bitcoin-footprints.png}
		\caption{Annualized total Bitcoin footprints~\cite{digiconomist_bitcoin_energy_consumption}.}
		\label{fig:btc-energy-footprint}
	\end{center}	
\end{figure}

%
PoS was conceived as a more energy-efficient alternative, shifting the
basis of consensus from computational power to economic stake. In a PoW
system, miners compete to solve a computationally intensive puzzle. This
process, while effective, requires vast amounts of electricity and
specialized hardware. PoS, in contrast, eliminates this computational
race. Instead, participants, known as validators, lock up a certain
amount of their cryptocurrency as a ``stake'' in the network. This
``virtual mining'' approach dramatically reduces the energy footprint of
the blockchain.

%\pandocbounded{\includegraphics[keepaspectratio,alt={PoR vs.~PoS}]{../../../Input/BDA-11-Proof-of-Stake-protocols-24.-4.-2025_files/Image_008.png}}
%\emph{A diagram comparing the resource requirements of Proof-of-Resource
	%(PoR) and Proof-of-Stake (PoS).}

\subsubsection{Core Concepts of
	Proof-of-Stake}\label{core-concepts-of-proof-of-stake}

The operation of a Proof-of-Stake system is defined by several key
concepts:

\begin{itemize}
	\tightlist
	\item
	\textbf{Goal}: To translate any Proof-of-Resource (PoR) into a
	``Proof-of-Money.'' Instead of buying hardware, miners buy stake and
	vote with it to earn interest.
	\item
	\textbf{Sybil Resistance}: Achieved by requiring a financial stake,
	making it prohibitively expensive to create numerous fake identities.
	\item
	\textbf{Efficiency}: PoS is both energy-friendly and more efficient in
	terms of transaction throughput.
	\item
	\textbf{Participation}: In theory, any token holder can be a
	``stakeholder.'' In practice, this often involves delegated staking
	due to availability and computational resource requirements.
	\item
	\textbf{Decentralization}: PoS offers partially-reduced centralization
	compared to the ASIC-dominated landscape of PoW.
	\item
	\textbf{Low Operational Costs}: The low operational costs mean
	validators can wait longer for rewards.
\end{itemize}

\subsubsection{Advantages and Disadvantages of
	PoS}\label{advantages-and-disadvantages-of-pos}

\textbf{Advantages:}

\begin{itemize}
	\tightlist
	\item
	\textbf{Energy Efficiency}: PoS is orders of magnitude more
	energy-efficient than PoW.
	\item
	\textbf{Reduced Barriers to Entry}: PoS does not require investment in
	specialized, high-cost mining hardware.
	\item
	\textbf{Higher Throughput}: Faster block creation times might lead to
	higher transaction throughput.
	\item
	\textbf{Economic Security}: Aligns validators' incentives with the
	long-term health of the blockchain.
\end{itemize}

\textbf{Disadvantages:}

\begin{itemize}
	\tightlist
	\item
	\textbf{Stake Centralization}: The ``rich get richer'' problem, where
	wealth can become concentrated among the largest stakers. A 51\% miner
	can potentially control the chain forever, as no new powerful miner
	can emerge as easily as in PoW.
	\item
	\textbf{Semi-Permissionless Nature}: One must acquire stake to
	participate, which typically means buying it from someone, making the
	system not fully permissionless.
	\item
	\textbf{The Nothing-at-Stake Problem}: A validator has an economic
	incentive to validate blocks on all competing chains in a fork, as
	there is no additional cost.
	\item
	\textbf{Long-Range Attacks}: An adversary can acquire old private keys
	to rewrite the blockchain's history.
\end{itemize}

\begin{center}\rule{0.5\linewidth}{0.5pt}\end{center}

\subsection{Security Challenges in
	Proof-of-Stake}\label{section-2-security-challenges-in-proof-of-stake}

\subsubsection{The Nothing-at-Stake
	Problem}\label{the-nothing-at-stake-problem}

The nothing-at-stake problem is a fundamental challenge where a
validator, in the event of a fork, can extend two or more conflicting
blocks without risking their stake, thereby increasing their chance of
being rewarded. This increases the number of forks and the time to
finality.

\textbf{Countermeasures:}

\begin{itemize}
	\tightlist
	\item
	\textbf{Deposit-Based Solutions}: Require nodes to make a deposit that
	is lost in the case of misbehavior (slashing).
	\item
	\textbf{Checkpoints}: Employ ``state freezing'' at periodic snapshots
	of the blockchain, making it irreversible up to the most recent
	checkpoint.
	\item
	\textbf{Backward Penalization}: Penalizing nodes that produced two
	conflicting blocks.
	\item
	\textbf{BFT Integration}: Combining PoS protocols with Byzantine Fault
	Tolerance approaches to decrease the probability of forks.
\end{itemize}

\subsubsection{Grinding Attacks}\label{grinding-attacks}

A grinding attack occurs when a leader, knowing they will produce the
next block, can bias the process to increase their chances of being
selected in the future. For example, if the next leader is determined
solely by the hash of the previous block, the current leader can
``grind'' through different block configurations to find one that
results in a favorable outcome for them.

\textbf{Countermeasures:}

\begin{itemize}
	\tightlist
	\item
	\textbf{Secure Multiparty Computation (SMPC)}: Using a committee of
	consensus nodes to generate a fresh random number for leader election
	(e.g., a secure coin-flipping protocol).
	\item
	\textbf{Verifiable Random Functions (VRFs)}: Allowing a node to
	privately check if its VRF output is below a certain stake-specific
	threshold. The VRF input combines the user's private key and
	randomness from the previous block, ensuring the current leader cannot
	bias the outcome.
\end{itemize}

\subsubsection{Denial of Service (DoS) on a
	Leader}\label{denial-of-service-dos-on-a-leader}

If a leader is publicly determined before their turn to produce a block,
an adversary can launch a DoS attack against them, forcing a round
restart. This can be repeated until the adversary's desired nodes are
elected.

\medskip
\textbf{Countermeasures:}
\begin{itemize}
	\tightlist
	\item \textbf{Private Leader Election (via VRF)}: As pioneered by Algorand~\cite{gilad2017algorand},
	a node privately determines if it is a potential leader and
	immediately releases a block candidate. By the time the leader is
	publicly known, it is too late for a DoS attack to be effective.
	
	\item \textbf{Whisk}: Designed for Ethereum's PoS. It conceals block proposers' identities until their assigned slot by shuffling a candidate list using verifiable random permutations and ZKPs. 
	Operating in a pipelined mode (shown in \autoref{fig:whisk-tereza}), current-round proposers prepare the shuffle for the next round, enabling scalability while preserving pre-slot anonymity.
	
		
	\item \textbf{Homomorphic sortition}: is a cryptographic mechanism that leverages threshold fully homomorphic encryption (ThFHE) to perform proposer selection over encrypted data~\cite{freitas2022homomorphic}. This approach ensures that proposers cannot be identified until a joint decryption is completed (see \autoref{fig:hs-tereza}).
	
	\item \textbf{Network-Level Deanonymization} --
	can be leveraged to enhance validator anonymity. 
	Dandelion++ and RLN~\cite{dandelion} aimed to obfuscate the origin of consensus messages by routing them through a private sub-network. 
	%While conceptually promising, the final analysis concluded that the approach is not feasible for Ethereum’s consensus layer due to strict latency constraints and high implementation complexity, especially under the tightened timing rules. 
	%The trade-offs ultimately limited its applicability despite its strong anonymity model. 
	%
	Polkadot's Sassafras~\cite{sassafras}  combines an SSLE-based leader election with network-layer anonymity. It complements consensus-layer protection with mechanisms to conceal proposer network traffic, offering a more comprehensive defense against deanonymization and targeted attacks.
	%
	CoPoR-PoS~\cite{homoliak2025pos} proposed native embedding of the anonymization layer into the consensus protocol itself.
	
\end{itemize}


\begin{figure}[t]
	%	\vspace{-0.3cm}
	\begin{center}
		\includegraphics[width=0.8\textwidth]{./figs/whisk-tereza.png}
		\caption{The pipeline of the Whisk proposer protection mechanism~\cite{burianova2025secret}.}
		\label{fig:whisk-tereza}
	\end{center}	
\end{figure}

\begin{figure}[t]
	%	\vspace{-0.3cm}
	\begin{center}
		\includegraphics[width=0.8\textwidth]{./figs/hs-tereza.png}
		\caption{Homomorphic sortition protocol~\cite{burianova2025secret}.}
		\label{fig:hs-tereza}
	\end{center}	
\end{figure}



\subsubsection{Long-Range Attacks}\label{long-range-attacks}

Also known as posterior corruption, this attack involves an adversary
bribing or stealing the private keys of previously influential consensus
nodes. Since these nodes may have already exchanged their tokens for
fiat currency, they have no stake left to lose and no incentive to
protect their old keys. If the attacker accumulates enough historical
stake, they can rerun the protocol and rewrite the entire history of the
blockchain.


\medskip
\textbf{Countermeasures~\cite{homoliak2020security}:}
\begin{itemize}
	\tightlist
	\item
	\textbf{Extended Deposit Locking}: Locking the stake deposit for a
	much longer time than the period of participation.
	\item
	\textbf{Frequent Checkpoints}: Making the chain irreversible with
	respect to the last checkpoint. Ethereum PoS uses justified and
	finalized checkpoints for this purpose.
	\item
	\textbf{Key-Evolving Cryptography}: Requiring users to evolve their
	private keys and erase old ones. While this prevents forging
	signatures, it doesn't stop dishonest nodes from voluntarily selling
	their old keys.
	\item
	\textbf{Time-Domain Chain Density}: Enforcing rules about the expected
	number of participants in each round.
	\item
	\textbf{Context-Sensitive Transactions}: Including the hash of a
	recent valid block within a transaction itself, tethering it to a
	specific point in history.

	
	
	
\end{itemize}

\begin{center}\rule{0.5\linewidth}{0.5pt}\end{center}

\subsection{Prominent Proof-of-Stake
	Protocols}\label{section-3-prominent-proof-of-stake-protocols}


\subsubsection{Peercoin}\label{peercoin}

Launched in 2012, Peercoin is the first technical realization of PoS,
using a hybrid PoW/PoS model. Its PoS component is based on
\textbf{coin-age}, calculated as
\texttt{(amount\ of\ UTXO)\ ×\ (\#\ of\ blocks\ it\ remains\ unspent)}.
Miners can balance PoW and coin-age, but it is much easier to find a
solution by consuming some coin-age.

\subsubsection{Algorand}\label{algorand}

Algorand is a pure PoS protocol with no punishments or locked tokens. It
uses cryptographic sortition to choose leaders via a Verifiable Random
Function (VRF) to avoid DoS on the leader attack (see \autoref{denial-of-service-dos-on-a-leader}). A leader is selected privately and proposes a new block.
Since there can exist multiple leaders meeting the threshold for VRF, Agorand requires additional round to vote on the best one.\footnote{Therefore, Algorand belongs to the family of Non-Single Secret Leader Election (NSSLE) protocols.}
For this purpose a committee of \textasciitilde1000 nodes, also selected by VRF, then
votes on the proposed block to ensure Byzantine tolerance.

\subsubsection{Ouroboros}\label{ouroboros}

Ouroboros~\cite{kiayias2017ouroboros} is a family of provably secure PoS protocols developed for the
Cardano blockchain.

\begin{itemize}
	\tightlist
	\item
	\textbf{Ouroboros Classic}: This version uses a Publicly Verifiable
	Secret Sharing (PVSS) scheme for randomness generation. Slot leaders
	are publicly known in advance, making them vulnerable to DoS attacks.
	The randomness generation is a detailed, five-step process:
	
	\begin{enumerate}
		\def\labelenumi{\arabic{enumi}.}
		\tightlist
		\item
		\textbf{Committee Formation}: Slot leaders form a committee and
		privately generate random numbers.
		\item
		\textbf{Commit}: Members post their PVSS data (a commitment and
		encrypted shares for every other member).
		\item
		\textbf{Reveal}: Members reveal their random numbers.
		\item
		\textbf{Recovery}: For any member who did not reveal, other members
		post their shares to reconstruct the secret.
		\item
		\textbf{New Epoch}: The revealed numbers are XORed together to
		create a seed for picking leaders in the next epoch.
	\end{enumerate}
	\item
	\textbf{Ouroboros Praos}: Inspired by Algorand, this version uses
	VRFs. Each node knows privately which slots they will lead, mitigating
	DoS attacks. There can be multiple or no leaders for a given slot.
\end{itemize}

\subsubsection{Ethereum's PoS (Casper \&
	Gasper)}\label{ethereums-pos-casper-gasper}

Ethereum's transition to PoS utilizes a combination of protocols.



\begin{figure}[t]
	%	\vspace{-0.3cm}
	\begin{center}
		\includegraphics[width=0.7\textwidth]{./figs/casper1.png}
		\caption{Finalization in Casper FFG~\cite{buterin2017casper}. The setup assumes 3 validators holding the majority of the stake.}
		\label{fig:casper-forks}
	\end{center}	
\end{figure}


\begin{figure}[t]
	%	\vspace{-0.3cm}
	\begin{center}
		\includegraphics[width=0.9\textwidth]{./figs/casper2.png}
		\caption{Forks and finalization in Casper FFG~\cite{buterin2017casper}. The branch with the majority of validators (the upper one) will resume finalizing checkpoints first. The setup assumes}
		\label{fig:casper-forks2}
	\end{center}	
\end{figure}


\begin{itemize}
	\tightlist
	\item
	\textbf{Casper the Friendly Finality Gadget (FFG)}: Casper FFG~\cite{buterin2017casper} is not
	a standalone consensus protocol but a ``finality gadget'' that
	overlays a block proposal mechanism. It introduces checkpoints and
	attestations (votes).
	
	\begin{itemize}
		\tightlist
		\item
		\textbf{Justification}: A checkpoint B is justified if a previously
		justified checkpoint A exists and there are attestations for the A
		-\textgreater{} B edge with a total weight of at least 2/3 of the
		total stake.
		\item
		\textbf{Finalization}: A checkpoint A is finalized if it is
		justified and its immediate successor is also justified. Finalized
		blocks are considered permanent (see \autoref{fig:casper-forks} and \autoref{fig:casper-forks2}).
		%    \pandocbounded{\includegraphics[keepaspectratio,alt={Casper FFG Finalization}]{../../../Input/BDA-11-Proof-of-Stake-protocols-24.-4.-2025_files/Image_015.png}}
		%\emph{A diagram illustrating the finalization process in CasperFFG.}
		\item
		\textbf{Slashing Conditions}: Validators are slashed for breaking
		rules, such as:
		
		\begin{figure}[b]
			%	\vspace{-0.3cm}
			\begin{center}
				\includegraphics[width=0.4\textwidth]{./figs/ghost.png}
				\caption{The GHOST fork-choice rule.}
				\label{fig:ghost}
			\end{center}	
		\end{figure}
		
		
		\begin{itemize}
			\tightlist
			\item
			\textbf{S1}: Making two distinct attestations for checkpoints at
			the same height (i.e., voting for a fork).
			\item
			\textbf{S2}: Making an attestation that ``surrounds'' another
			(e.g., voting for an edge s1 -\textgreater{} t1 and another s2
			-\textgreater{} t2 where
			\texttt{height(s1)\ \textless{}\ height(s2)\ \textless{}\ height(t2)\ \textless{}\ height(t1)}).
		\end{itemize}
	\end{itemize}
	\item
	\textbf{Gasper}: The full protocol used in Ethereum 2.0, which
	combines the Casper FFG finality gadget with the GHOST (Greediest
	Heaviest Observed SubTree) fork-choice rule (see \autoref{fig:ghost}).
	%  \pandocbounded{\includegraphics[keepaspectratio,alt={Gasper Protocol}]{../../../Input/BDA-11-Proof-of-Stake-protocols-24.-4.-2025_files/Image_016.jpg}}
\end{itemize}





\begin{center}\rule{0.5\linewidth}{0.5pt}\end{center}

\subsection{Oracles and Data
	Feeds}\label{section-4-oracles-and-data-feeds}

A blockchain (and its smart contracts) is an isolated environment and by default has no access to data in external world. The ``oracle problem'' is the
challenge of providing smart contracts with reliable data from the
external world.
There are several approaches to oracle that we will briefly describe.

\subsubsection{TownCrier (TC)}\label{towncrier-tc}

TownCrier~\cite{zhang2016town} acts as a trusted proxy between a blockchain and HTTPS
websites, using a Trusted Execution Environment (TEE) like Intel SGX (see \autoref{fig:oracles-tc}).
\vspace{-0.3cm}
\begin{itemize}
	\tightlist
	\item \textbf{Pros}: Easy integration, no changes needed for website
	operators. 
	
	\item \textbf{Cons}: Trusts the TEE manufacturer (a single point
	of failure), and several attacks have been demonstrated against Intel
	SGX.	
\end{itemize}


\begin{figure}[t]
	%	\vspace{-0.3cm}
	\begin{center}
		\includegraphics[width=0.8\textwidth]{./figs/oracle-problem.png}
		\caption{Oracle problem.}
		\label{fig:oracles}
	\end{center}	
\end{figure}


\begin{figure}[b]
	%	\vspace{-0.3cm}
	\begin{center}
		\includegraphics[width=0.85\textwidth]{./figs/tc1.png}
		\caption{TownCrier~\cite{zhang2016town}.}
		\label{fig:oracles-tc}
	\end{center}	
\end{figure}

\subsubsection{TLS-N}\label{tls-n}

TLS-N~\cite{ritzdorf2018tls} adds a non-repudiation layer to TLS, allowing for the generation
of privacy-preserving, non-interactive proofs of the contents of a TLS
session (see overview in \autoref{fig:tlsn}). 
It uses efficient Merkle-tree-based authentication and produces TLS-N proofs that can be verified by smart contracts.
\begin{itemize}
	\tightlist
	
	\item \textbf{Pros}: More general and powerful, with extra features
	like privacy. 
	\item \textbf{Cons}: Requires changes to the TLS specification
	and updates to TLS servers, making integration difficult and expensive.
	
\end{itemize}

%\pandocbounded{\includegraphics[keepaspectratio,alt={TLS-N Diagram}]{../../../Input/BDA-11-Proof-of-Stake-protocols-24.-4.-2025_files/Image_020.jpg}}

\subsubsection{Practical Data Feed Service
	(PDFS)}\label{provable-data-feeds-pdfs}

In PDFS~\cite{guarnizo2019pdfs}, data providers maintain an append-only database (like a
Merkle tree) and sign their contract locations via TLS (see \autoref{fig:pdfs}). The contract API
allows for updating the database root and proving the authenticity of
data via Merkle proofs. 
\vspace{-0.3cm}
\begin{itemize}
	\tightlist
	\item \textbf{Pros}: Direct TLS authentication, easy
	integration. 
	\item \textbf{Cons}: Website operators have to deploy it.
	
\end{itemize}
%\pandocbounded{\includegraphics[keepaspectratio,alt={PDFS Diagram}]{../../../Input/BDA-11-Proof-of-Stake-protocols-24.-4.-2025_files/Image_021.jpg}}


\begin{figure}[t]
	%	\vspace{-0.3cm}
	\begin{center}
		\includegraphics[width=0.7\textwidth]{./figs/TLSN.png}
		\caption{TLS-N overview~\cite{ritzdorf2018tls}.}
		\label{fig:tlsn}
	\end{center}	
\end{figure}


\begin{figure}[t]
	%	\vspace{-0.3cm}
	\begin{center}
		\includegraphics[width=0.8\textwidth]{./figs/pdfs.png}
		\caption{PDFS overview~\cite{guarnizo2018pdfs}.}
		\label{fig:pdfs}
	\end{center}	
\end{figure}

\begin{center}\rule{0.5\linewidth}{0.5pt}\end{center}

\subsection{Summary / Key Takeaways}\label{summary-key-takeaways}

This section has provided a brief exploration of Proof-of-Stake
consensus protocols. We established the motivation for PoS as a more
energy-efficient alternative to PoW, defined its core concepts, and
presented a balanced view of its advantages and disadvantages. A key
focus was the unique security landscape of PoS, including the
nothing-at-stake problem, grinding attacks, and long-range attacks,
along with their countermeasures. We surveyed influential PoS protocols
like Peercoin, Algorand, Ouroboros, and Ethereum's Casper. Finally, we
addressed the critical oracle problem, examining how external data can
be securely fed to blockchains.

\begin{center}\rule{0.5\linewidth}{0.5pt}\end{center}

\subsection{Keywords}\label{keywords}

\begin{itemize}
	\tightlist
	\item
	\textbf{Proof-of-Stake (PoS)}: A class of consensus mechanisms that
	select block creators based on the number of coins they hold and are
	willing to ``stake.''
	\item
	\textbf{Validator}: A participant in a PoS network responsible for
	validating transactions and creating new blocks.
	\item
	\textbf{Slashing}: A penalty mechanism where a validator who acts
	maliciously has their staked funds confiscated.
	\item
	\textbf{Nothing-at-Stake Problem}: A scenario where a validator has no
	economic disincentive to validate blocks on all competing chains in a
	fork.
	\item
	\textbf{Long-Range Attack}: An attack where an adversary uses old,
	compromised private keys to create an alternative history of the
	chain.
	\item
	\textbf{Verifiable Random Function (VRF)}: A cryptographic function
	used for secure and private leader election.
	\item
	\textbf{Finality}: The guarantee that a transaction or block is
	irreversible.
	\item
	\textbf{Oracle}: A service that provides external data to a smart
	contract.
	\item
	\textbf{Trusted Execution Environment (TEE)}: A secure area inside a
	main processor, used by systems like TownCrier.
\end{itemize}

\begin{center}\rule{0.5\linewidth}{0.5pt}\end{center}

\subsection{Further Reading}\label{further-reading}

\begin{itemize}
	\tightlist
	\item \textbf{Energy consumption of Bitcoin} \\
	\url{https://digiconomist.net/bitcoin-energy-consumption}
	\item \textbf{PeerCoin} \\
	\url{https://peercoin.net/assets/paper/peercoin-paper.pdf}
	\item \textbf{Proof-of-Stake in Ethereum} \\
	\url{https://ethereum.org/en/developers/docs/consensus-mechanisms/pos/}
	\item \textbf{Gasper} \\
	\url{https://arxiv.org/pdf/1903.04205}
	\item \textbf{Ethereum 2.0} \\
	\url{https://arxiv.org/abs/2003.03052} 
	\item \textbf{Ouroboros} \\
	\url{https://www.youtube.com/watch?v=hMgxZOsTlQc} 
	\item \textbf{TownCrier} \\
	\url{https://eprint.iacr.org/2016/168.pdf} 
	\item \textbf{Ouroboros Praos} \\
	\url{https://eprint.iacr.org/2017/578.pdf} 
	\item \textbf{PDFS} \\
	\url{https://arxiv.org/pdf/1808.06641.pdf} 
\end{itemize}

\newpage

\section{Anonymity and Privacy}\label{chapter-6-anonoymity}
This section delves into the critical and often misunderstood concepts
of privacy and anonymity in the context of blockchain technology. While
public blockchains are frequently lauded for their transparency, this
very transparency can pose significant challenges to user privacy. We
will begin by dissecting the nuanced distinctions between anonymity,
pseudonymity, and privacy, and we will explore how these concepts apply
to the blockchain.

A common misconception is that blockchains are inherently anonymous. In
reality, most public blockchains are pseudonymous, meaning that users
are identified by cryptographic addresses rather than their real-world
identities. However, as we will see, this pseudonymity can be
compromised through various de-anonymization techniques. We will examine
several of these techniques, including address clustering and
network-level analysis, to understand how the privacy of blockchain
users can be undermined.
%
In response to these privacy challenges, a range of privacy-enhancing
technologies has been developed. We will explore several of these
technologies in detail, including:

\begin{itemize}
\tightlist
\item
  \textbf{Mixing services}, which aim to obscure the link between the
  sender and receiver of a transaction.
\item
  \textbf{Ring signatures}, a cryptographic technique that allows a user
  to sign a transaction on behalf of a group without revealing their
  individual identity.
\item
  \textbf{Zero-knowledge proofs}, a powerful cryptographic tool that
  enables a user to prove the validity of a statement without revealing
  any information beyond the statement's truthfulness.
\end{itemize}

By the end of this section, you will have an understanding
of the privacy landscape of blockchain technology, from the inherent
challenges of public ledgers to the cutting-edge cryptographic
techniques that are being used to build a more private and anonymous
future.

\subsection{Learning Objectives}\label{learning-objectives}

\begin{itemize}
\tightlist
\item
  Understand the difference between anonymity, pseudonymity, and
  privacy.
\item
  Learn about the techniques used to de-anonymize blockchain users, such
  as address clustering and network-level analysis.
\item
  Grasp the concept of mixing services and how they can be used to
  enhance privacy.
\item
  Understand the principles of ring signatures and their application in
  privacy-focused cryptocurrencies like Monero~\cite{moser2017empirical}.
\item
  Learn about zero-knowledge proofs and their role in enabling private
  transactions.
\item
  Gain insight into the workings of privacy-focused protocols like
  Zerocoin~\cite{miers2013zerocoin} and Zerocash~\cite{sasson2014zerocash}, and their implementation in Zcash~\cite{kappos2018empirical}.
\end{itemize}

\begin{center}\rule{0.5\linewidth}{0.5pt}\end{center}

\subsection{The Landscape of Anonymity and
Privacy}\label{section-1-the-landscape-of-anonymity-and-privacy}

\subsubsection{Defining Anonymity, Privacy, and
Unlinkability}\label{defining-anonymity-privacy-and-unlinkability}

In the context of blockchain technology, the terms anonymity and privacy
are often used interchangeably, but they refer to distinct, albeit
related, concepts.
%
%\begin{figure}
%\centering
%%\pandocbounded{\includegraphics[keepaspectratio,alt={Anonymity vs Privacy}]{../Input/BDA-07-Anonymity-and-privacy-28.-3.-2024_files/slide_3.png}}
%\caption{Anonymity vs Privacy}
%\end{figure}

\begin{itemize}
\item
  \textbf{Anonymity} refers to the state of being unidentifiable. It
  ensures that a user may use a resource or service without disclosing
  their identity. In a truly anonymous system, it is impossible to link
  actions or transactions to a specific individual. As the transcription
  notes, this is about ``hiding identity.'' For example, a post on a
  forum may be attributed to a pseudonym, not a real-world identity.
\item
  \textbf{Privacy} is the ability of individuals or groups to determine
  for themselves when, how, and to what extent information about them is
  communicated to others. In the context of blockchains, this refers to
  the ability to conceal the details of a transaction, such as the
  identities of the sender and receiver, and the amount being
  transferred. It's about ``hiding confidential information/actions.''
\item
  \textbf{Unlinkability} is a key property that contributes to both
  anonymity and privacy. It is the inability of an adversary to link
  different actions or transactions to the same user. Achieving
  unlinkability is a major challenge in blockchain design. Specific
  requirements for unlinkability include making it difficult to:

  \begin{itemize}
  \tightlist
  \item
    Link different addresses of the same user.
  \item
    Link different transactions made by the same user.
  \item
    Link the sender of a payment to its recipient.
  \end{itemize}
\end{itemize}

Most public blockchains, like Bitcoin, do not provide true anonymity.
Instead, they offer \textbf{pseudonymity}, where users are identified by
cryptographic addresses. The combination of pseudonymity and
unlinkability is what constitutes anonymity on a blockchain.

%\begin{figure}
%\centering
%%\pandocbounded{\includegraphics[keepaspectratio,alt={Anonymity, Pseudonymity, Unlinkability}]{../Input/BDA-07-Anonymity-and-privacy-28.-3.-2024_files/slide_4.png}}
%\caption{Anonymity, Pseudonymity, Unlinkability}
%\end{figure}

\subsubsection{The Anonymity Set}\label{the-anonymity-set}

The \textbf{Anonymity Set} is a crucial concept in this domain. It
refers to the set of transactions that an adversary cannot distinguish
from a specific transaction. The larger the anonymity set, the
greater the privacy, as it becomes harder to single out any individual
transaction. This concept is particularly important when we discuss
techniques like mixing and ring signatures.

%\begin{figure}
%\centering
%%\pandocbounded{\includegraphics[keepaspectratio,alt={Unlinkability \& Anonymity Set}]{../Input/BDA-07-Anonymity-and-privacy-28.-3.-2024_files/slide_5.png}}
%\caption{Unlinkability \& Anonymity Set}
%\end{figure}

\begin{center}\rule{0.5\linewidth}{0.5pt}\end{center}

\subsection{De-anonymization
Techniques}\label{section-2-de-anonymization-techniques}

The pseudonymity of public blockchains can be compromised through a
variety of de-anonymization techniques, which aim to link blockchain
addresses to real-world identities.

%\begin{figure}
%\centering
%%\pandocbounded{\includegraphics[keepaspectratio,alt={De-anonymization}]{../Input/BDA-07-Anonymity-and-privacy-28.-3.-2024_files/slide_6.png}}
%\caption{De-anonymization}
%\end{figure}

\subsubsection{Address Clustering and
Linking}\label{address-clustering-and-linking}

This technique involves analyzing the transaction graph to identify
addresses that are likely controlled by the same entity.



\begin{figure}[t]
	%	\vspace{-0.3cm}
	\begin{center}
		\includegraphics[width=0.4\textwidth]{./figs/deanon1.png}
		\caption{Multiple input addresses controlled by the same user.}		
		\label{fig:deanon1}
	\end{center}	
\end{figure}

\begin{itemize}
\tightlist
\item
  \textbf{Common-Input-Ownership Heuristic}: If multiple addresses (and
  their corresponding UTXOs) are used as inputs in a single transaction,
  it is highly probable that they belong to the same user (see \autoref{fig:deanon1}).
\item
  \textbf{Change Addresses}: When a user sends a transaction and the
  input amount is greater than the output, the remaining ``change'' is
  sent back to a new address. This change address is controlled by the
  same user, creating a link between the input addresses and the new
  change address (see \autoref{fig:deanon2}).
\end{itemize}

\begin{figure}[t]
	%	\vspace{-0.3cm}
	\begin{center}
		\includegraphics[width=0.4\textwidth]{./figs/deanon2.png}
		\caption{The change address controlled by the same user.}		
		\label{fig:deanon2}
	\end{center}	
\end{figure}

%\begin{figure}
%\centering
%%\pandocbounded{\includegraphics[keepaspectratio,alt={Linking Addresses Example}]{../Input/BDA-07-Anonymity-and-privacy-28.-3.-2024_files/slide_7.png}}
%\caption{Linking Addresses Example}
%\end{figure}

Once a cluster of addresses has been identified, it can be linked to a
real-world identity through various means:

\begin{itemize}
\tightlist
\item
  \textbf{Transacting with Service Providers}: Centralized services like
  cryptocurrency exchanges often require Know Your Customer (KYC)
  verification, linking a user's identity to their deposit and
  withdrawal addresses.
\item
  \textbf{Publicly Advertised Addresses}: Users may post their addresses
  online for donations or payments, directly linking the address to
  their online persona or real identity.
\item
  \textbf{Carelessness}: Users might inadvertently reveal connections
  between their addresses through their online activities.
\end{itemize}

%\pandocbounded{\includegraphics[keepaspectratio,alt={Clustering and Linking to Identities}]{../Input/BDA-07-Anonymity-and-privacy-28.-3.-2024_files/slide_8.png}}
%\pandocbounded{\includegraphics[keepaspectratio,alt={Linking to Individuals}]{../Input/BDA-07-Anonymity-and-privacy-28.-3.-2024_files/slide_9.png}}

\subsubsection{Network-Level
De-anonymization}\label{network-level-de-anonymization}

A more sophisticated de-anonymization technique involves monitoring
network traffic to link IP addresses to blockchain transactions. An
adversary with control over a significant portion of the network, such
as an Internet Service Provider (ISP), could potentially use this
technique. When a new transaction is broadcast, it often originates from
a non-full node (like an SPV or light client). The first IP address to
broadcast this transaction is likely the originator.

The solution to this is to use network-layer anonymization tools like
Tor, JonDonym, or VPNs to obscure the user's real IP address.

%\begin{figure}
%\centering
%%\pandocbounded{\includegraphics[keepaspectratio,alt={Network-level De-anonymization}]{../Input/BDA-07-Anonymity-and-privacy-28.-3.-2024_files/slide_10.png}}
%\caption{Network-level De-anonymization}
%\end{figure}

\begin{center}\rule{0.5\linewidth}{0.5pt}\end{center}

\subsection{Privacy-Enhancing
Technologies}\label{section-3-privacy-enhancing-technologies}

\subsubsection{Mixing Services
(Tumblers)}\label{mixing-services-tumblers}

Mixing services are designed to obscure the link between the sender and
receiver of a transaction by mixing a user's coins with those of other
users.

%\begin{figure}
%\centering
%%\pandocbounded{\includegraphics[keepaspectratio,alt={Mixing Services}]{../Input/BDA-07-Anonymity-and-privacy-28.-3.-2024_files/slide_11.png}}
%\caption{Mixing Services}
%\end{figure}

\begin{itemize}
\tightlist
\item
  \textbf{Centralized Mixers}: These services operate as trusted
  intermediaries. A user sends coins to the mixer, which puts them into
  a large pool with coins from other users. The mixer then sends out the
  equivalent amount from this pool to the intended recipient, breaking
  the on-chain link. This approach requires users to trust the mixer not
  to steal their funds or to keep logs of their transactions.
\end{itemize}

%\begin{figure}
%\centering
%%\pandocbounded{\includegraphics[keepaspectratio,alt={Centralized Mixing}]{../Input/BDA-07-Anonymity-and-privacy-28.-3.-2024_files/slide_12.png}}
%\caption{Centralized Mixing}
%\end{figure}

\begin{itemize}
\tightlist
\item
  \textbf{Decentralized Mixers (e.g., CoinJoin)}: To address the trust issue
  of centralized mixers, decentralized mixing protocols like CoinJoin
  have been developed. CoinJoin is a peer-to-peer protocol that allows
  multiple users to collaboratively create a single transaction that
  combines their inputs and outputs (see \autoref{fig:coinjoin}). This makes it difficult for an
  external observer to determine the exact mapping between the inputs
  and outputs, thereby enhancing the privacy of the participants. The
  process involves finding peers, exchanging input/output addresses
  anonymously, constructing a joint transaction, and having all
  participants sign it.
\end{itemize}

\begin{figure}[t]
	%	\vspace{-0.3cm}
	\begin{center}
		\includegraphics[width=0.4\textwidth]{./figs/coinjoin.png}
		\caption{A single transaction of CoinJoin.}		
		\label{fig:coinjoin}
	\end{center}	
\end{figure}

%\pandocbounded{\includegraphics[keepaspectratio,alt={CoinJoin}]{../Input/BDA-07-Anonymity-and-privacy-28.-3.-2024_files/slide_13.png}}
%\pandocbounded{\includegraphics[keepaspectratio,alt={CoinJoin Process}]{../Input/BDA-07-Anonymity-and-privacy-28.-3.-2024_files/slide_14.png}}

\subsubsection{Ring Signatures}\label{ring-signatures}

A ring signature is a cryptographic technique that allows a member of a
group to sign a message on behalf of the group, without revealing which
member of the group produced the signature. This provides a high degree
of sender anonymity.

%\begin{figure}
%\centering
%%\pandocbounded{\includegraphics[keepaspectratio,alt={Ring Signatures}]{../Input/BDA-07-Anonymity-and-privacy-28.-3.-2024_files/slide_15.png}}
%\caption{Ring Signatures}
%\end{figure}

The key properties are: (A) The actual signer declares an arbitrary set of
possible signers (the ``ring''), which includes themselves. (B) The
signature is constructed using the signer's private key and the public
keys of all other ring members. (C) No interaction is required between the
ring members. (D) An adversary cannot distinguish which member of the ring
is the true signer.

%\begin{figure}
%\centering
%%\pandocbounded{\includegraphics[keepaspectratio,alt={Ring Signature Properties}]{../Input/BDA-07-Anonymity-and-privacy-28.-3.-2024_files/slide_17.png}}
%\caption{Ring Signature Properties}
%\end{figure}

\paragraph{Generating and Verifying a Ring
Signature.}\label{generating-and-verifying-a-ring-signature}

The process, illustrated with an RSA example, involves several steps: 
\begin{enumerate}
	\item The signer hashes the message \texttt{m} to create a symmetric key
	\texttt{k}.
	
	\item A random value \texttt{v} is generated by signer. 
	
	\item  For all other ring members, signer generates random values \texttt{$x_i$} and corresponding
	$y_i = g_i(x_i)$ values are computed using a \textbf{trapdoor} function (e.g.,
	$y_i\ =\ x_i^{e_i}\ mod\ n_i$ in RSA). 
	
	\item  Given a symmetric encryption algorithm $E_k$, we define a combining function
	$C_{k,v}$ that is used to create a loop, and the signer solves for their
	own $y_s$ value by fitting the equation 
	($C_{k,v}(...)\ =\ v$).  
	See also \autoref{fig:ring} and \autoref{fig:ring2}. 
	
	\item The signer computes their private key  $x_s = g_{s^{-1}}(y_s)$ using the knowledge of trapdoor from $y_s$. 
	E.g. in RSA: $x_s = y_{s}^{d_s}\ mod\ n_s$ 
	 
	 \item  The final signature consists of the public keys of all ring members, the value \texttt{v}, and all the \texttt{x} values, i.e., ($P_1, P_2, \ldots, P_r, v, x_1, x_2, \ldots, x_s, \ldots, x_r$). 
	
\end{enumerate}


\begin{figure}[b]
	%	\vspace{-0.3cm}
	\begin{center}
		\includegraphics[width=0.7\textwidth]{./figs/ring-sig.png}
		\caption{Combining function in a ring signature.}		
		\label{fig:ring}
	\end{center}	
\end{figure}


\begin{figure}[t]
	%	\vspace{-0.3cm}
	\begin{center}
		\includegraphics[width=0.5\textwidth]{./figs/ring2.png}
		\caption{Combining function equal to $v$ in a ring signature.}		
		\label{fig:ring2}
	\end{center}	
\end{figure}

Verification involves checking that the equation holds true using the
public keys and the provided signature components.
In particular, given a message \textbf{m} and alleged signature ($P_1, P_2, \ldots, P_r, v, x_1, x_2, \ldots, x_s, \ldots, x_r$), the signature is verified as follows:

\begin{enumerate}
	\item For $i = 1,2,3, \ldots, r$, the verifier computes $y_i = g_i (x_i)$.
	
	\item The verifier hashes the message to obtain the symmetric encryption key $k = H(m)$.
	
	\item The verifier checks that the $y_i$’s computed in 1) satisfy the equation $C_{k,v}(y_1, y_2, …, y_r) = v$
	
\end{enumerate}
	


%\pandocbounded{\includegraphics[keepaspectratio,alt={Generating a Ring Signature 1}]{../Input/BDA-07-Anonymity-and-privacy-28.-3.-2024_files/slide_18.png}}
%\pandocbounded{\includegraphics[keepaspectratio,alt={Generating a Ring Signature 2}]{../Input/BDA-07-Anonymity-and-privacy-28.-3.-2024_files/slide_19.png}}
%\pandocbounded{\includegraphics[keepaspectratio,alt={Generating a Ring Signature 3}]{../Input/BDA-07-Anonymity-and-privacy-28.-3.-2024_files/slide_20.png}}
%\pandocbounded{\includegraphics[keepaspectratio,alt={Verifying a Ring Signature}]{../Input/BDA-07-Anonymity-and-privacy-28.-3.-2024_files/slide_21.png}}

\paragraph{Monero.}\label{monero}
The privacy-focused cryptocurrency Monero is the most well-known
application of ring signatures. 
It uses two key pairs: \textbf{viewing
keys} and \textbf{spending keys}. \textbf{Stealth addresses} are used
for recipient anonymity; a new one-time address is generated for each
transaction. \textbf{Key images} are used to prevent double-spending.
A unique key image is derived from each output, and the blockchain
maintains a list of spent key images. \textbf{RingCT (Ring
Confidential Transactions)} hides transaction amounts using Pedersen
commitments and zero-knowledge range proofs.



%\pandocbounded{\includegraphics[keepaspectratio,alt={Monero}]{../Input/BDA-07-Anonymity-and-privacy-28.-3.-2024_files/slide_22.png}}
%\pandocbounded{\includegraphics[keepaspectratio,alt={Monero Details}]{../Input/BDA-07-Anonymity-and-privacy-28.-3.-2024_files/slide_23.png}}
%\pandocbounded{\includegraphics[keepaspectratio,alt={Monero Key Images \& RingCT}]{../Input/BDA-07-Anonymity-and-privacy-28.-3.-2024_files/slide_24.png}}

\subsubsection{Zero-Knowledge Proofs}\label{zero-knowledge-proofs}

A zero-knowledge proof (ZKP) is a powerful cryptographic protocol that
allows one party (the prover) to prove to another party (the verifier)
that a statement is true, without revealing any information beyond the
validity of the statement itself.

%\begin{figure}
%\centering
%%\pandocbounded{\includegraphics[keepaspectratio,alt={Zero-Knowledge Proofs}]{../Input/BDA-07-Anonymity-and-privacy-28.-3.-2024_files/slide_25.png}}
%\caption{Zero-Knowledge Proofs}
%\end{figure}

The classic ``Ali Baba Cave'' example illustrates the concept of an
interactive proof (see \autoref{fig:alibaba}). However, for blockchains, non-interactive proofs are
needed. The \textbf{Fiat-Shamir heuristic} is a technique to convert
interactive proofs into non-interactive ones by replacing the verifier's
challenge with the output of a hash function~\cite{fiat1986prove}.


\begin{figure}[t]
	%	\vspace{-0.3cm}
	\begin{center}
		\includegraphics[width=0.8\textwidth]{./figs/alibaba.png}
		\caption{Interactive zero-knowledge proof example of Ali Baba cave Verifier has to repeat the query multiple times to decrease a chance of prover to guess it~\cite{Quisquater1990}.}		
		\label{fig:alibaba}
	\end{center}	
\end{figure}


%\pandocbounded{\includegraphics[keepaspectratio,alt={Ali Baba Cave}]{../Input/BDA-07-Anonymity-and-privacy-28.-3.-2024_files/slide_26.png}}
%\pandocbounded{\includegraphics[keepaspectratio,alt={Schnorr's Protocol (Interactive)}]{../Input/BDA-07-Anonymity-and-privacy-28.-3.-2024_files/slide_27.png}}
%\pandocbounded{\includegraphics[keepaspectratio,alt={Non-Interactive ZKP}]{../Input/BDA-07-Anonymity-and-privacy-28.-3.-2024_files/slide_28.png}}

\paragraph{Zk-SNARKs.}\label{zk-snarks}

\textbf{Zero-Knowledge Succinct Non-interactive ARguments of Knowledge
(zk-SNARKs)} are a specific type of ZKP ideal for blockchains because
their proofs are very small (``succinct'') and quick to verify. They
allow a party to prove the correctness of a complex computation without
revealing the inputs and without requiring the verifier to re-run the
entire computation.
%
%\begin{figure}
%\centering
%%\pandocbounded{\includegraphics[keepaspectratio,alt={zk-SNARKs}]{../Input/BDA-07-Anonymity-and-privacy-28.-3.-2024_files/slide_29.png}}
%\caption{zk-SNARKs}
%\end{figure}

\subsubsection{Schnorr's Protocol -- Discrete Log. Problem}\label{sec:schnorr}
Schnorr's protocol represents interactive proving of knowledge of discrete logarithm.
In particular, prover wants to prove the knowledge of discrete logarithm $x$ of
$h = g^x \in G$, where $G$ is a group of prime order $q$ (see \autoref{fig:schnorr-inter}).

\begin{figure}[h]
	%	\vspace{-0.3cm}
	\begin{center}
		\includegraphics[width=0.4\textwidth]{./figs/schnor-inter.png}
		\caption{Schnorr's interactive protocol.}		
		\label{fig:schnorr-inter}
	\end{center}	
\end{figure}


\medskip
\textbf{Non-Interactive Variant.} 
Using Fiat-Shamir heuristic, we can convert interactive Schnorr's protocol to non-interactive protocol in random oracle model. 
In particular, the challenge $c$ is replaced by the hash cryptographicaly secure hash function (see \autoref{fig:schnorr-noninter}).
\begin{figure}[h]
	%	\vspace{-0.3cm}
	\begin{center}
		\includegraphics[width=0.5\textwidth]{./figs/schnor-noninter.png}
		\caption{Schnorr's non-interactive protocol.}		
		\label{fig:schnorr-noninter}
	\end{center}	
\end{figure}





\subsubsection{Zcash (Zerocoin and Zerocash
Protocols)}\label{zcash-zerocoin-and-zerocash-protocols}

Zcash is a cryptocurrency that implements these advanced ZKP concepts.

%\begin{figure}
%\centering
%%\pandocbounded{\includegraphics[keepaspectratio,alt={Zcash}]{../Input/BDA-07-Anonymity-and-privacy-28.-3.-2024_files/slide_30.png}}
%\caption{Zcash}
%\end{figure}

\begin{itemize}
\tightlist
\item
  \textbf{Zerocoin Protocol}: The original protocol proposed for
  Bitcoin. It allows a user to ``burn'' a coin and ``mint'' a new,
  anonymous one with no transaction history, using a ZKP to prove
  ownership.
\end{itemize}

%\begin{figure}
%\centering
%%\pandocbounded{\includegraphics[keepaspectratio,alt={Zerocoin Protocol}]{../Input/BDA-07-Anonymity-and-privacy-28.-3.-2024_files/slide_31.png}}
%\caption{Zerocoin Protocol}
%\end{figure}

\begin{itemize}
\tightlist
\item
  \textbf{Zerocash Protocol}: An extension of Zerocoin that uses
  zk-SNARKs to make the proofs much smaller and to hide not only the
  origin but also the destination and amount of the transaction. A
  critical aspect of this protocol is the need for a \textbf{trusted
  setup}, where initial public parameters are generated from random data
  that must then be destroyed to prevent counterfeiting. Modern
  implementations use multi-party computation (MPC) ceremonies to
  distribute this trust.
\end{itemize}

%\begin{figure}
%\centering
%%\pandocbounded{\includegraphics[keepaspectratio,alt={Zerocash Protocol}]{../Input/BDA-07-Anonymity-and-privacy-28.-3.-2024_files/slide_32.png}}
%\caption{Zerocash Protocol}
%\end{figure}

\begin{itemize}
\tightlist
\item
  \textbf{Zcash Implementation}: Zcash uses two types of addresses:

  \begin{itemize}
  \tightlist
  \item
    \textbf{t-addresses} (transparent)
  \item
    \textbf{z-addresses} (private or ``shielded'') These addresses are
    interoperable, allowing for four transaction types: private
    (z-to-z), shielding (t-to-z), deshielding (z-to-t), and public
    (t-to-t).
  \end{itemize}
\end{itemize}

%\begin{figure}
%\centering
%%\pandocbounded{\includegraphics[keepaspectratio,alt={Zcash Blockchain}]{../Input/BDA-07-Anonymity-and-privacy-28.-3.-2024_files/slide_33.png}}
%\caption{Zcash Blockchain}
%\end{figure}

\begin{center}\rule{0.5\linewidth}{0.5pt}\end{center}

\subsection{Summary / Key Takeaways}\label{summary-key-takeaways}

This section has provided an overview of the complex and
often misunderstood topics of privacy and anonymity in the context of
blockchain technology. We have established that while public blockchains
are transparent by design, they are not inherently anonymous, but rather
pseudonymous. This pseudonymity, however, can be compromised through
various de-anonymization techniques.

We have explored several of these techniques, including address
clustering and network-level analysis, which can be used to link
blockchain addresses to real-world identities. In response to these
privacy challenges, a range of privacy-enhancing technologies has been
developed. We have examined several of these technologies in detail,
including:

\begin{itemize}
\tightlist
\item
  \textbf{Mixing services}, which obscure the transaction graph by
  mixing the coins of multiple users.
\item
  \textbf{Ring signatures}, which provide sender anonymity by allowing a
  user to sign a transaction on behalf of a group.
\item
  \textbf{Zero-knowledge proofs}, which enable the verification of
  transactions without revealing any sensitive information.
\end{itemize}

The development of these technologies highlights the ongoing tension
between the transparency and privacy of blockchain systems. As the
technology continues to mature, this will undoubtedly remain a key area
of research and innovation.

\begin{center}\rule{0.5\linewidth}{0.5pt}\end{center}

\subsection{Keywords}\label{keywords}

\begin{itemize}
\tightlist
\item
  \textbf{Anonymity}: The quality or state of being anonymous; the
  condition of having no name or an unknown name.
\item
  \textbf{Privacy}: The state of being free from public attention; the
  ability to control what information is shared with others.
\item
  \textbf{Pseudonymity}: The use of a fictitious name or pseudonym to
  conceal one's true identity.
\item
  \textbf{Unlinkability}: The inability of an adversary to link
  different actions or transactions to the same user.
\item
  \textbf{Mixing Service (Tumbler)}: A service that breaks the link
  between the sender and receiver of a transaction by mixing the coins
  of multiple users.
\item
  \textbf{CoinJoin}: A decentralized mixing protocol that allows
  multiple users to combine their inputs into a single transaction.
\item
  \textbf{Ring Signature}: A type of digital signature that provides
  sender anonymity by allowing a user to sign a transaction on behalf of
  a group of users.
\item
  \textbf{Monero}: A privacy-focused cryptocurrency that uses ring
  signatures, stealth addresses, and RingCT to provide a high degree of
  privacy and anonymity.
\item
  \textbf{Zero-Knowledge Proof (ZKP)}: A cryptographic protocol that
  allows a prover to convince a verifier that a statement is true,
  without revealing any information beyond the validity of the statement
  itself.
\item
  \textbf{zk-SNARKs}: A specific type of zero-knowledge proof that is
  particularly efficient and is used in several privacy-focused
  cryptocurrencies, such as Zcash.
\item
  \textbf{Zcash}: A cryptocurrency that uses zk-SNARKs to provide strong
  privacy guarantees, allowing users to send transactions from shielded
  addresses.
\end{itemize}

\begin{center}\rule{0.5\linewidth}{0.5pt}\end{center}

\subsection{Further Reading}\label{further-reading}

\begin{itemize}
\tightlist
\item
  \textbf{A Fistful of Bitcoins: Characterizing Payments Among Men with
  No Names}: \\
\url{https://cseweb.ucsd.edu/~smeiklejohn/files/imc13.pdf}
\item
  \textbf{How to leak a secret: Theory and applications of ring
  signatures}: \\
  \url{https://people.csail.mit.edu/rivest/pubs/RST06.pdf}
\item
	\textbf{Zero-to-Monero}:\\
	\url{https://www.getmonero.org/library/Zero-to-Monero-2-0-0.pdf}
\item
  \textbf{Traceable ring signature}:\\
  \url{https://eprint.iacr.org/2006/389.pdf}
\item
  \textbf{Ring confidential transactions}:\\
  \url{https://www.getmonero.org/resources/moneropedia/ringCT.html}
\item
  \textbf{Why and how zk-SNARK works}:\\
  \url{https://arxiv.org/pdf/1906.07221.pdf}
\item
  \textbf{Zcash Technology}:\\ \url{https://z.cash/technology/}
\item
  \textbf{Zerocoin: Anonymous distributed e-cash from bitcoin}:\\
  \url{https://ieeexplore.ieee.org/document/6547123}
\item
  \textbf{Zerocash: Decentralized Anonymous Payments from Bitcoin}:\\
  \url{https://ieeexplore.ieee.org/document/6956581}
\end{itemize}

\newpage

\section{Scalability and Throughput}\label{chapter-6-scalability}


%% Allow footnotes in longtable head/foot
%\IfFileExists{footnotehyper.sty}{\usepackage{footnotehyper}}{\usepackage{footnote}}
%\makesavenoteenv{longtable}
%\usepackage{graphicx}
%\makeatletter
%\newsavebox\pandoc@box
%\newcommand*\pandocbounded[1]{% scales image to fit in text height/width
%  \sbox\pandoc@box{#1}%
%  \Gscale@div\@tempa{\textheight}{\dimexpr\ht\pandoc@box+\dp\pandoc@box\relax}%
%  \Gscale@div\@tempb{\linewidth}{\wd\pandoc@box}%
%  \ifdim\@tempb\p@<\@tempa\p@\let\@tempa\@tempb\fi% select the smaller of both
%  \ifdim\@tempa\p@<\p@\scalebox{\@tempa}{\usebox\pandoc@box}%
%  \else\usebox{\pandoc@box}%
%  \fi%
%}
%% Set default figure placement to htbp
%\def\fps@figure{htbp}
%\makeatother
%\setlength{\emergencystretch}{3em} % prevent overfull lines
%\providecommand{\tightlist}{%
%  \setlength{\itemsep}{0pt}\setlength{\parskip}{0pt}}
%\usepackage{bookmark}
%\IfFileExists{xurl.sty}{\usepackage{xurl}}{} % add URL line breaks if available
%\urlstyle{same}
%\hypersetup{
%  hidelinks,
%  pdfcreator={LaTeX via pandoc}}
%
%\author{}
%\date{}
%
%\begin{document}

This chapter confronts one of the most formidable challenges in the
evolution of blockchain technology: scalability. As decentralized
networks gain mainstream adoption, their ability to process a high
volume of transactions in a timely and cost-effective manner becomes
paramount. We will begin by exploring the conceptual framework of the
``blockchain trilemma,'' a term coined by Vitalik Buterin, which posits
that it is inherently difficult to simultaneously optimize for
scalability, security, and decentralization.

We will then embark on a comprehensive survey of the various approaches
that have been proposed to address this challenge. Our exploration will
encompass both Layer 1 and Layer 2 scaling solutions. At Layer 1, we
will examine protocol-level enhancements such as Bitcoin-NG, which
decouples leader election from transaction processing, and sharding, a
technique that partitions the network into smaller, more manageable
units. We will also explore the potential of Directed Acyclic Graph
(DAG) based protocols as an alternative to the traditional linear
blockchain structure.

Furthermore, we will consider the role of permissioned blockchains and
Trusted Execution Environments (TEEs) in achieving scalability,
acknowledging the trade-offs they entail in terms of decentralization.
By the end of this chapter, you will have a nuanced understanding of the
complex and multifaceted challenge of blockchain scalability and the
diverse range of solutions that are being developed to address it.

\subsection{Learning Objectives}\label{learning-objectives}

\begin{itemize}
\tightlist
\item
  Understand the concept of blockchain scalability and the blockchain
  trilemma.
\item
  Learn about different metrics for evaluating blockchain performance,
  such as throughput, latency, and finality.
\item
  Explore naive approaches to improving scalability and their
  limitations.
\item
  Grasp the concept of Bitcoin-NG and how it decouples leader election
  from transaction serialization.
\item
  Understand the principles of sharding and how it can be used to
  parallelize transaction processing.
\item
  Learn about DAG-based protocols as an alternative to traditional
  blockchain structures.
\item
  Gain insight into the role of permissioned blockchains and Trusted
  Execution Environments (TEEs) in achieving scalability.
\end{itemize}

\begin{center}\rule{0.5\linewidth}{0.5pt}\end{center}

\subsection{Section 1: The Scalability
Challenge}\label{section-1-the-scalability-challenge}

\subsubsection{1.1: Defining Scalability}\label{defining-scalability}

Scalability in the context of blockchain can be defined in two primary
ways:

\begin{enumerate}
\def\labelenumi{\arabic{enumi}.}
\tightlist
\item
  \textbf{Throughput:} This refers to the capability of the system to
  handle a growing amount of work, measured by the number of actions
  (transactions) processed in a given time frame. It is commonly
  expressed in \textbf{transactions per second (TPS)}.
\item
  \textbf{Number of Nodes:} This refers to the ability of the network to
  maintain its performance (e.g., TPS) as the number of participating
  nodes grows. A truly scalable system should not see its performance
  degrade as more nodes join the network.
\end{enumerate}

\subsubsection{1.2: The Blockchain
Trilemma}\label{the-blockchain-trilemma}

The blockchain trilemma is a conceptual model, popularized by Ethereum
co-founder Vitalik Buterin, that illustrates the inherent trade-offs in
designing a blockchain system. It posits that a blockchain can only
fully satisfy two of the following three properties at any given time:

\begin{itemize}
\tightlist
\item
  \textbf{Scalability}: The ability of the network to process a high
  volume of transactions (high TPS).
\item
  \textbf{Decentralization}: The ability of the network to run without
  any trust dependencies on a small group of centralized actors.
\item
  \textbf{Security}: The ability of the network to resist a large
  percentage of malicious participating nodes (e.g., 51\% for Bitcoin,
  33\% for BFT).
\end{itemize}

\begin{figure}
\centering

%\pandocbounded{\includegraphics[keepaspectratio,alt={Blockchain Trilemma}]{../../../Input/BDA-09-Scalability-Throughut-10.-4.-2025_files/Image_004.jpg}}
\caption{Blockchain Trilemma}
\end{figure}

This trilemma leads to different classes of solutions, each making a
specific trade-off:

\begin{itemize}
\tightlist
\item
  \textbf{A) Traditional L1 Chains (e.g., Bitcoin, Ethereum 1.0):}
  Prioritize decentralization and security, sacrificing scalability.
\item
  \textbf{B) High-TPS Chains (e.g., BFT-based):} Prioritize scalability
  and security, sacrificing decentralization.
\item
  \textbf{C) Multi-chain Ecosystems (e.g., Sharding, L2s):} Prioritize
  scalability and decentralization, but can have lower security as an
  attacker only needs to compromise a single shard.
\end{itemize}

\subsubsection{1.3: Throughput Limitations in
Practice}\label{throughput-limitations-in-practice}

The difference in throughput between traditional blockchains and
centralized systems is stark, highlighting the need for scaling
solutions.

\begin{longtable}[]{@{}ll@{}}
\toprule\noalign{}
System & Throughput (TPS) \\
\midrule\noalign{}
\endhead
\bottomrule\noalign{}
\endlastfoot
Bitcoin & \textasciitilde3 - 8 TPS \\
Ethereum & \textasciitilde15 - 30 TPS \\
Solana & \textasciitilde5,000 TPS (lab tested) \\
VISA & \textasciitilde11,000 - 25,000 TPS \\
\end{longtable}

\subsubsection{1.4: The Blockchain Stacked
Model}\label{the-blockchain-stacked-model}

Scalability can be addressed at different layers of the blockchain
stack:

\begin{figure}
\centering
%\pandocbounded{\includegraphics[keepaspectratio,alt={Blockchain Stacked Model}]{../../../Input/BDA-09-Scalability-Throughut-10.-4.-2025_files/Image_010.png}}
\caption{Blockchain Stacked Model}
\end{figure}

\begin{itemize}
\tightlist
\item
  \textbf{Network Layer:} Propagating transactions efficiently.
\item
  \textbf{Consensus Layer:} Ordering transactions efficiently.
\item
  \textbf{RSM (Replicated State Machine) Layer:} Executing smart
  contracts and interpreting transactions.
\item
  \textbf{Application Layer:} Interactions between contracts and chains.
\end{itemize}

\begin{center}\rule{0.5\linewidth}{0.5pt}\end{center}

\subsection{Section 2: Forks, Finality, and
Metrics}\label{section-2-forks-finality-and-metrics}

\subsubsection{2.1: Forks in PoW
Blockchains}\label{forks-in-pow-blockchains}

Forks occur when two or more miners find a block at roughly the same
time, creating temporary branches in the chain. This leads to ``stale''
or ``orphaned'' blocks that are eventually discarded.

\begin{figure}
\centering
%\pandocbounded{\includegraphics[keepaspectratio,alt={Fork illustration}]{../../../Input/BDA-09-Scalability-Throughut-10.-4.-2025_files/Image_012.jpg}}
\caption{Fork illustration}
\end{figure}

Forks create the risk of \textbf{double-spending}, where a value spent
in one branch is spent again in another. This is why transaction
confirmation requires waiting for a certain number of blocks to be added
after the transaction's block, a concept known as \textbf{finality}.

\subsubsection{2.2: Finality}\label{finality}

Finality is the time it takes for a transaction to be considered
irreversible. Different systems have vastly different finality times:

\begin{longtable}[]{@{}ll@{}}
\toprule\noalign{}
System & Finality Time \\
\midrule\noalign{}
\endhead
\bottomrule\noalign{}
\endlastfoot
Bitcoin & \textasciitilde60 min (6 blocks) \\
Ethereum PoW & \textasciitilde3 min (12 blocks) \\
Ethereum PoS & \textasciitilde12 min (2 epochs) \\
Algorand \& BFT & Instant \\
\end{longtable}

\subsubsection{2.3: Blockchain Metrics}\label{blockchain-metrics}

Several metrics are used to evaluate the performance and scalability of
a blockchain:

\begin{itemize}
\tightlist
\item
  \textbf{Throughput (TPS):} How many transactions can be processed per
  second.
\item
  \textbf{Latency:} The time it takes to propagate a transaction across
  the entire network.
\item
  \textbf{Finality:} The time it takes for a transaction to become
  irreversible.
\item
  \textbf{Stale Block Rate:} The percentage of mined blocks that are
  orphaned.
\item
  \textbf{Fairness:} The proportion of rewards relative to the invested
  mining power.
\end{itemize}

\begin{center}\rule{0.5\linewidth}{0.5pt}\end{center}

\subsection{Section 3: Layer 1 Scaling
Solutions}\label{section-3-layer-1-scaling-solutions}

Layer 1 scaling solutions involve making fundamental changes to the core
protocol of the blockchain.

\subsubsection{3.1: Naive Improvements and Their
Limitations}\label{naive-improvements-and-their-limitations}

The most intuitive approaches to improving scalability involve simply
adjusting the protocol's parameters, but these have significant
drawbacks.

\begin{itemize}
\item
  \textbf{Increasing the Block Size:} While this allows more
  transactions per block, larger blocks take longer to propagate,
  leading to a higher stale block rate and favoring miners with better
  network connections, thus increasing centralization.

  \begin{figure}
  \centering
%  \pandocbounded{\includegraphics[keepaspectratio,alt={Increased Block Size Problem}]{../../../Input/BDA-09-Scalability-Throughut-10.-4.-2025_files/Image_024.png}}
  \caption{Increased Block Size Problem}
  \end{figure}
\item
  \textbf{Decreasing the Block Creation Time:} This also increases the
  theoretical TPS but leads to more frequent forks and a higher stale
  block rate, which wastes resources and weakens security.

  \begin{figure}
  \centering
%  \pandocbounded{\includegraphics[keepaspectratio,alt={Decreased Block Time Problem}]{../../../Input/BDA-09-Scalability-Throughut-10.-4.-2025_files/Image_029.png}}
  \caption{Decreased Block Time Problem}
  \end{figure}
\end{itemize}

As shown in the graph below, decreasing the block creation time
dramatically increases the orphan block rate.

\begin{figure}
\centering
%\pandocbounded{\includegraphics[keepaspectratio,alt={Orphan Block Rate vs.~Block Time}]{../../../Input/BDA-09-Scalability-Throughut-10.-4.-2025_files/Image_030.png}}
\caption{Orphan Block Rate vs.~Block Time}
\end{figure}

\subsubsection{3.2: Bitcoin-NG (Next
Generation)}\label{bitcoin-ng-next-generation}

Bitcoin-NG is a more sophisticated Layer 1 scaling solution that
improves throughput by decoupling leader election from transaction
serialization.

\textbf{Principle:} 1. \textbf{Leader Election (Key Blocks):} A leader
is elected via Proof-of-Work, creating a ``key block.'' This block does
not contain transactions but secures the leader's right to publish
transactions. 2. \textbf{Transaction Serialization (Microblocks):} The
elected leader then serially publishes transactions in smaller
``microblocks'' without needing to perform PoW for each one. This
continues until a new leader mines a new key block.

\begin{figure}
\centering
%\pandocbounded{\includegraphics[keepaspectratio,alt={Bitcoin-NG Principle}]{../../../Input/BDA-09-Scalability-Throughut-10.-4.-2025_files/Image_035.jpg}}
\caption{Bitcoin-NG Principle}
\end{figure}

This approach allows for much higher transaction throughput, as it is
limited only by the network's propagation capacity and the leader's
processing power, not by the PoW mining interval. However, it introduces
a new vulnerability: the leader becomes a single point of failure and a
target for Denial-of-Service (DoS) attacks.

\textbf{Evaluation:} Bitcoin-NG shows significantly better scalability
than Bitcoin, although it does not achieve perfect linear scaling.

\begin{figure}
\centering
%\pandocbounded{\includegraphics[keepaspectratio,alt={Bitcoin-NG Evaluation}]{../../../Input/BDA-09-Scalability-Throughut-10.-4.-2025_files/Image_041.jpg}}
\caption{Bitcoin-NG Evaluation}
\end{figure}

\subsubsection{3.3: Sharding}\label{sharding}

Sharding is a powerful technique for scaling blockchains that is
borrowed from the world of distributed databases. It involves
partitioning the state of the blockchain (accounts, transactions, etc.)
into smaller, more manageable pieces called \textbf{shards}. Each shard
is then processed by a different subset of nodes (a committee) in the
network. This allows for parallel transaction processing, which can lead
to a linear increase in throughput as the number of nodes in the network
grows.

\begin{figure}
\centering
%\pandocbounded{\includegraphics[keepaspectratio,alt={Horizontal Sharding}]{../../../Input/BDA-09-Scalability-Throughut-10.-4.-2025_files/Image_046.jpg}}
\caption{Horizontal Sharding}
\end{figure}

\paragraph{3.3.1: Elastico}\label{elastico}

Elastico was the first practical sharding protocol proposed for open
blockchains.

\textbf{Design:} 1. \textbf{Identity Establishment:} Nodes perform a PoW
task. The resulting hash determines their identity and committee
assignment. 2. \textbf{Committee Formation:} Nodes are divided into
smaller committees (shards) based on their PoW solution. A final
``directory committee'' is also formed to finalize blocks. 3.
\textbf{Intra-Committee Consensus:} Each committee processes a subset of
transactions and reaches consensus using a BFT-like protocol. 4.
\textbf{Final Consensus:} The directory committee collects the processed
blocks from each shard and broadcasts a final, unified block to the
network.

\begin{figure}
\centering
%\pandocbounded{\includegraphics[keepaspectratio,alt={Elastico Committee Formation}]{../../../Input/BDA-09-Scalability-Throughut-10.-4.-2025_files/Image_058.jpg}}
\caption{Elastico Committee Formation}
\end{figure}

\textbf{Evaluation:} Elastico demonstrated that throughput could scale
nearly linearly with the number of nodes. However, a significant amount
of time was spent on committee formation, and it lacked a mechanism for
secure cross-shard transactions.

\begin{figure}
\centering
%\pandocbounded{\includegraphics[keepaspectratio,alt={Elastico Evaluation}]{../../../Input/BDA-09-Scalability-Throughut-10.-4.-2025_files/Image_073.png}}
\caption{Elastico Evaluation}
\end{figure}

\paragraph{3.3.2: OmniLedger}\label{omniledger}

OmniLedger builds upon Elastico with several key improvements.

\textbf{Design Goals:} * \textbf{Security:} Full decentralization, shard
robustness, and secure cross-shard transactions. * \textbf{Performance:}
Scale-out, low storage, and low latency.

\textbf{Key Innovations:} * \textbf{RandHound:} A secure and unbiased
randomness protocol used for validator assignment to shards, preventing
adversarial manipulation. * \textbf{Atomix:} A protocol for secure,
atomic cross-shard transactions. It ensures that transactions spanning
multiple shards either commit atomically or abort completely. *
\textbf{ByzCoinX:} A robust and efficient intra-shard BFT consensus
protocol.

\begin{figure}
\centering
%\pandocbounded{\includegraphics[keepaspectratio,alt={OmniLedger Architecture}]{../../../Input/BDA-09-Scalability-Throughut-10.-4.-2025_files/Image_080.png}}
\caption{OmniLedger Architecture}
\end{figure}

\textbf{Evaluation:} OmniLedger demonstrated significant performance
gains over previous systems, achieving high throughput with low latency.

\begin{figure}
\centering
%\pandocbounded{\includegraphics[keepaspectratio,alt={OmniLedger Throughput}]{../../../Input/BDA-09-Scalability-Throughut-10.-4.-2025_files/Image_087.jpg}}
\caption{OmniLedger Throughput}
\end{figure}

\paragraph{3.3.3: RapidChain}\label{rapidchain}

RapidChain aims for \textbf{full sharding}, partitioning not only
computation but also storage and communication. This approach promises
even greater scalability by minimizing redundancy across the network.

\begin{figure}
\centering
%\pandocbounded{\includegraphics[keepaspectratio,alt={RapidChain Comparison}]{../../../Input/BDA-09-Scalability-Throughut-10.-4.-2025_files/Image_093.png}}
\caption{RapidChain Comparison}
\end{figure}

\subsubsection{3.4: DAG-Based Protocols}\label{dag-based-protocols}

Directed Acyclic Graph (DAG) based protocols represent a fundamental
departure from the linear chain structure of traditional blockchains. In
a DAG-based system, each transaction confirms one or more previous
transactions, creating a graph-like structure. This allows for parallel
transaction processing and can result in significantly higher throughput
and lower confirmation times.

\begin{figure}
\centering
%\pandocbounded{\includegraphics[keepaspectratio,alt={PHANTOM DAG}]{../../../Input/BDA-09-Scalability-Throughut-10.-4.-2025_files/Image_100.png}}
\caption{PHANTOM DAG}
\end{figure}

However, DAGs introduce new challenges, such as achieving a total
ordering of transactions, which is an NP-hard problem. Protocols like
\textbf{IOTA} and \textbf{PHANTOM} have been developed to address these
challenges, but they remain an active area of research.

\paragraph{Sycomore}\label{sycomore}

Sycomore is a DAG-based protocol that features a self-adapting
structure. The DAG can split into more parallel chains when transaction
demand is high and merge back together when demand is low, dynamically
adjusting the level of parallelism.

\begin{figure}
\centering
%\pandocbounded{\includegraphics[keepaspectratio,alt={Sycomore Self-Adaptivity}]{../../../Input/BDA-09-Scalability-Throughut-10.-4.-2025_files/Image_106.jpg}}
\caption{Sycomore Self-Adaptivity}
\end{figure}

\begin{center}\rule{0.5\linewidth}{0.5pt}\end{center}

\subsection{Section 4: Layer 2 and Other Scaling
Solutions}\label{section-4-layer-2-and-other-scaling-solutions}

\subsubsection{4.1: Layer 2 Scaling}\label{layer-2-scaling}

Layer 2 solutions operate on top of an existing Layer 1 blockchain,
handling transactions off-chain to reduce the load on the main chain.

\begin{itemize}
\item
  \textbf{Payment Channels (e.g., Bitcoin Lightning):} Users open a
  channel by locking funds on the main chain. They can then conduct
  numerous transactions off-chain within that channel, only settling the
  final balance on the main chain when the channel is closed.

  \begin{figure}
  \centering
%  \pandocbounded{\includegraphics[keepaspectratio,alt={Layer 2 Payment Channels}]{../../../Input/BDA-09-Scalability-Throughut-10.-4.-2025_files/Image_107.png}}
  \caption{Layer 2 Payment Channels}
  \end{figure}
\item
  \textbf{ZK-Rollups:} Transactions are processed off-chain, and a
  cryptographic proof (a ZK-SNARK) is generated to attest to their
  validity. This single proof is then submitted to the Layer 1 chain,
  allowing a large batch of transactions to be verified with a single
  on-chain transaction.
\end{itemize}

\subsubsection{4.2: Permissioned
Blockchains}\label{permissioned-blockchains}

Permissioned blockchains offer a different approach to scalability by
restricting network access to a known set of participants. This allows
for the use of more efficient, non-PoW consensus mechanisms like
Byzantine Fault Tolerance (BFT).

\begin{itemize}
\tightlist
\item
  \textbf{Hyperledger Fabric:} A popular enterprise-grade framework that
  uses a modular architecture with components for ordering services,
  membership providers, and smart contract execution.
\end{itemize}

\subsubsection{4.3: Trusted Execution Environments
(TEEs)}\label{trusted-execution-environments-tees}

A Trusted Execution Environment (TEE) is a secure and isolated area of a
main processor that provides confidentiality and integrity guarantees
for code and data. TEEs, such as \textbf{Intel SGX}, can be used to
create hybrid blockchain solutions that offload some of the
computational work from the main blockchain to a trusted hardware
environment, improving scalability and privacy.

\begin{figure}
\centering
%\pandocbounded{\includegraphics[keepaspectratio,alt={Intel SGX}]{../../../Input/BDA-09-Scalability-Throughut-10.-4.-2025_files/Image_119.png}}
\caption{Intel SGX}
\end{figure}

\begin{center}\rule{0.5\linewidth}{0.5pt}\end{center}

\subsection{Summary / Key Takeaways}\label{summary-key-takeaways}

This chapter has provided a comprehensive overview of the critical
challenge of blockchain scalability. We have established that there is
no single, perfect solution, but rather a spectrum of approaches that
involve different trade-offs between scalability, security, and
decentralization, as encapsulated by the blockchain trilemma.

We have explored a variety of Layer 1 scaling solutions, from naive
approaches like increasing the block size to more sophisticated
techniques like Bitcoin-NG, sharding, and DAG-based protocols. We have
also examined Layer 2 solutions like payment channels and ZK-Rollups, as
well as alternative architectures like permissioned blockchains and
TEEs.

The key takeaway is that the pursuit of scalability is an ongoing and
dynamic area of research. As the technology matures, we can expect to
see the emergence of new and innovative solutions that combine these
different approaches to push the boundaries of what is possible in terms
of performance, security, and decentralization.

\begin{center}\rule{0.5\linewidth}{0.5pt}\end{center}

\subsection{Keywords}\label{keywords}

\begin{itemize}
\tightlist
\item
  \textbf{Scalability}: The ability of a blockchain network to handle a
  growing number of transactions and users without compromising
  performance.
\item
  \textbf{Blockchain Trilemma}: The widely held belief that a blockchain
  system can only fully satisfy two of the following three properties:
  scalability, security, and decentralization.
\item
  \textbf{Throughput (TPS)}: The number of transactions that a
  blockchain network can process per second.
\item
  \textbf{Finality}: The time it takes for a transaction to be
  considered permanent and irreversible.
\item
  \textbf{Layer 1 Scaling}: Protocol-level changes that aim to improve
  the scalability of the blockchain itself.
\item
  \textbf{Bitcoin-NG}: A Layer 1 scaling solution that decouples leader
  election from transaction serialization.
\item
  \textbf{Sharding}: A technique for partitioning the state of a
  blockchain into smaller, more manageable pieces called shards.
\item
  \textbf{DAG (Directed Acyclic Graph)}: A data structure that allows
  for parallel transaction processing and can be used as an alternative
  to a linear blockchain.
\item
  \textbf{Layer 2 Scaling}: Solutions that operate on top of a Layer 1
  blockchain to improve scalability.
\item
  \textbf{Permissioned Blockchain}: A type of blockchain that restricts
  access to a limited set of participants.
\item
  \textbf{Trusted Execution Environment (TEE)}: A secure and isolated
  area of a main processor that can be used to offload computation from
  the main blockchain.
\end{itemize}

\begin{center}\rule{0.5\linewidth}{0.5pt}\end{center}

\subsection{Further Reading}\label{further-reading}

\begin{itemize}
\tightlist
\item
  \textbf{The Blockchain Trilemma}:
  https://vitalik.ca/general/2021/04/07/sharding.html
\item
  \textbf{Bitcoin-NG: A Scalable Blockchain Protocol}:
  https://www.usenix.org/system/files/conference/nsdi16/nsdi16-paper-eyal.pdf
\item
  \textbf{Elastico: A Secure Sharding Protocol for Open Blockchains}:
  https://www.comp.nus.edu.sg/\textasciitilde loiluu/papers/elastico.pdf
\item
  \textbf{OmniLedger: A Secure, Scale-Out, Decentralized Ledger}:
  https://eprint.iacr.org/2017/406.pdf
\item
  \textbf{RapidChain: A Secure, Fast and Scalable Blockchain}:
  https://www.usenix.org/conference/usenixsecurity19/presentation/zamani
\end{itemize}
\newpage

\section{A Layered Security Model For Blockchain}\label{chapter-12-sra}

\hypertarget{introduction}{%
\subsection{Introduction}\label{introduction}}

This chapter introduces a layered security model for
blockchain technology, providing a structured framework for analyzing
the diverse security and privacy challenges inherent in decentralized
systems. Drawing an analogy to the well-established ISO/OSI model for
computer networks, this framework, termed the Security Reference
Architecture (SRA)~\cite{homoliak2019security}, deconstructs the complex blockchain ecosystem into
four distinct layers: \textbf{the network layer, the consensus
layer, the replicated state machine (RSM) layer, and the application
layer.}

Throughout this chapter, we will systematically explore the unique
security threats, vulnerabilities, and potential countermeasures
pertinent to each layer. This structured approach allows for a granular
understanding of how security is enforced at different levels of the
blockchain stack, from the fundamental peer-to-peer communication
protocols to the sophisticated decentralized applications (DApps) that
users interact with.

Furthermore, we will examine a broad spectrum of blockchain
applications, categorizing them from foundational ecosystem components,
such as wallets and tokens, to higher-level, real-world use cases like
e-voting, supply chain management, and digital notarization. By applying
the SRA framework to these applications, we can better appreciate their
specific security considerations and the interplay between the different
layers in ensuring their overall resilience and integrity.

\hypertarget{learning-objectives}{%
\subsection{Learning Objectives}\label{learning-objectives}}

\begin{itemize}
\tightlist
\item
  Understand the concept and utility of a layered security model for
  analyzing blockchain systems.
\item
  Identify the four layers of the Security Reference Architecture (SRA):
  network, consensus, replicated state machine, and application.
\item
  Recognize and describe the primary security threats and defensive
  mechanisms at the network layer, including DNS attacks, routing
  attacks, and eclipse attacks.
\item
  Analyze the security vulnerabilities at the consensus layer, such as
  51\% attacks, double-spending, and selfish mining.
\item
  Evaluate the security risks at the replicated state machine (RSM)
  layer, with a focus on smart contract bugs and data privacy.
\item
  Categorize and explore the security considerations for a wide range of
  blockchain applications, from ecosystem-level tools to higher-level
  use cases.
\end{itemize}

\begin{center}\rule{0.5\linewidth}{0.5pt}\end{center}

\hypertarget{section-1-the-security-reference-architecture-sra-for-blockchains}{%
\subsection{The Security Reference Architecture for Blockchains}\label{section-1-the-security-reference-architecture-sra-for-blockchains}}

\hypertarget{a-layered-approach-to-blockchain-security}{%
\subsubsection{A Layered Approach to Blockchain
Security}\label{a-layered-approach-to-blockchain-security}}

The Security Reference Architecture (SRA) for blockchains~\cite{homoliak2020security} provides a
conceptual model for organizing and understanding the multifaceted
security aspects of decentralized ledger technology. By dividing the
system into layers, we can isolate and analyze specific functionalities
and their associated vulnerabilities. The four layers of the SRA are (see also \autoref{fig:SRA-overview}):


\begin{figure}[t]
	%	\vspace{-0.3cm}
	\begin{center}
		\includegraphics[width=0.55\textwidth]{./figs/overview-SRA.pdf}
		\caption{Overview of SRA~\cite{homoliak2020security}.}		
		\label{fig:SRA-overview}
	\end{center}	
\end{figure}

\begin{itemize}
\tightlist
\item
  \textbf{Network Layer}: This foundational layer deals with the
  peer-to-peer (P2P) communication protocols that enable nodes to
  connect, discover each other, and exchange data. It encompasses
  everything from peer discovery and message routing to the underlying
  Internet protocols upon which the blockchain network is built.
\item
  \textbf{Consensus Layer}: This is the core of the blockchain,
  responsible for the mechanism by which all nodes in the network agree
  on a single, consistent, and immutable order of transactions. This
  layer includes the specific consensus algorithms, such as
  Proof-of-Work (PoW) or Proof-of-Stake (PoS), that ensure the integrity
  of the ledger.
\item
  \textbf{Replicated State Machine (RSM) Layer}: This layer is
  responsible for interpreting the ordered transactions provided by the
  consensus layer and executing them to update the state of the
  blockchain. It includes the virtual machine (e.g., the EVM) that runs
  smart contracts and the logic for processing transactions.
\item
  \textbf{Application Layer}: This is the topmost layer, consisting of
  the decentralized applications (DApps), user interfaces, and external
  services (like oracles) that are built on top of the blockchain
  infrastructure to provide specific functionalities to end-users.
\end{itemize}


\begin{figure}[t]
	%	\vspace{-0.3cm}
	\begin{center}
		\includegraphics[width=0.95\textwidth]{./figs/threat-risk-assessment.png}
		\caption{Threat-risk assessment according to ISO/IEC15408.}		
		\label{fig:thread-risk-assessment}
	\end{center}	
\end{figure}


\hypertarget{threat-risk-assessment-model-for-blockchains}{%
\subsubsection{Threat Risk Assessment Model for
Blockchains}\label{threat-risk-assessment-model-for-blockchains}}

The SRA framework is conceptually grounded in the Threat Risk Assessment
(TRA) model, which is a standard methodology for security analysis of centralized systems,
notably detailed in the ISO 15408 (Common Criteria) standard (see \autoref{fig:thread-risk-assessment}). 
The TRA
model provides a systematic process for: 
\begin{enumerate}
	\item Identifying valuable
	\textbf{assets} within the system. 
	
	\item  Recognizing potential
	\textbf{threats} to those assets. 
	
	\item  Assessing the
	\textbf{vulnerabilities} that could be exploited by those threats. 
	
	\item  Evaluating the \textbf{risks} associated with these vulnerabilities. 

	\item  Implementing \textbf{countermeasures} to mitigate the identified risks
	to an acceptable level.
	
\end{enumerate}
The projection of SRA to TRA is depicted in \autoref{fig:thread-risk-assessment-SRA}, and it enriches TRA with the environment of blockchain and its specific aspects.


To better visualize these relationships in the context of blockchains,
we can use \textbf{Vulnerability, Threat, and Defense (VTD) graphs}.
These graphs map out the connections between specific vulnerabilities,
the threats that exploit them, and the defenses designed to counter
them, providing a clear and structured view of the security landscape at
each layer.


\begin{figure}[t]
	%	\vspace{-0.3cm}
	\begin{center}
		\includegraphics[width=0.55\textwidth]{./figs/ISO-SRA.pdf}
		\caption{Threat-risk assessment model adapted to SRA~\cite{homoliak2019security}.}		
		\label{fig:thread-risk-assessment-SRA}
	\end{center}	
\end{figure}


\begin{center}\rule{0.5\linewidth}{0.5pt}\end{center}

%\hypertarget{section-2-security-at-each-layer}{%
%\subsection{Security at Each
%Layer}\label{section-2-security-at-each-layer}}

\hypertarget{network-layer-security}{%
\subsection{Network Layer}\label{network-layer-security}}

The network layer is the backbone of any blockchain, and its security is
paramount. As blockchain networks are typically overlays on top of the
public Internet, they inherit many of its vulnerabilities (see also \autoref{fig:vtd-network-public}) such as:

\begin{itemize}
\tightlist
\item
  \textbf{DNS Attacks}: Many blockchain nodes initially find peers
  through DNS seeds. An attacker who can compromise these DNS records
  (e.g., through DNS cache poisoning) can redirect new nodes to a
  network partition controlled by the attacker.
\item
  \textbf{Routing Attacks}: Attacks on the Internet's routing
  infrastructure, such as BGP hijacking, can be used to intercept or
  partition blockchain network traffic, isolating certain regions or
  nodes from the rest of the network.
\item
  \textbf{Eclipse Attacks}: A more targeted attack where an adversary
  monopolizes all of a victim node's incoming and outgoing connections.
  The eclipsed node is thus isolated from the honest network and can be
  fed false information, potentially leading it to accept fraudulent
  transactions or waste its consensus power on an invalid chain.
  
  \begin{figure}[t]
  	%	\vspace{-0.3cm}
  	\begin{center}
  		\includegraphics[width=0.55\textwidth]{./figs/attacks-network-public.pdf}
  		\caption{VTD graph of the network layer -- public networks~\cite{homoliak2019security}.}		
  		\label{fig:vtd-network-public}
  	\end{center}	
  \end{figure}
  
  
  \begin{figure}[t]
  	%	\vspace{-0.3cm}
  	\begin{center}
  		\includegraphics[width=0.55\textwidth]{./figs/attacks-network-private.pdf}
  		\caption{VTD graph of the network layer -- private networks~\cite{homoliak2019security}.}		
  		\label{fig:vtd-network-private}
  	\end{center}	
  \end{figure}
  
  
\item
  \textbf{Denial-of-Service (DoS) Attacks}: These attacks aim to disrupt
  the network by overwhelming nodes with traffic or resource-intensive
  requests. A specific example is the ``penny-flooding'' attack, where
  an attacker spams the network with a large number of low-fee
  transactions to clog the mempool (the holding area for unconfirmed
  transactions) and potentially crash nodes -- these are DoS attacks on local resources.
  DoS attacks on network connectivity reduce effective consensus power and reward opportunities for honest nodes. 
  
  \item \textbf{Identity Revealing Attacks} are aimed at linking IP address with the blockchain address extracted from a transaction.
  
  
\end{itemize}




Besides public network of permissionless and semi-permissionless blockchains, there exist private networks of permissioned blockchains, which might contain different types of vulnerabilities and threats (see \autoref{fig:vtd-network-private}) related to centralization of control, such as external targeted attacks and insider threat~\cite{homoliak2018insight}.



\hypertarget{consensus-layer-security}{%
\subsection{Consensus Layer}\label{consensus-layer-security}}

The consensus layer is where the ``truth'' of the blockchain is decided through inclusion and ordering of transactions,
making it a high-value target for attackers.
The potential threats of this layer can be divided according to the type of consensus protocol as follows.






\begin{itemize}
\tightlist
\item
  \textbf{Generic Attacks} are depicted in \autoref{fig:sra-cons-gen} and they mostly involve:

  \begin{itemize}
  \tightlist
  \item
    \textbf{51\% Attacks}: The most well-known consensus attack, where
    an attacker (or a coalition of attackers) controls a majority of the
    network's consensus power (e.g., hashing power in PoW, or stake in
    PoS). With this control, the attacker can prevent new transactions
    from gaining confirmations, halt payments between some or all users,
    and reverse transactions that were completed while they were in
    control (double-spending).
  \item
    \textbf{Double-Spending Attacks}: This is a primary goal of many
    consensus-level attacks. An attacker sends a transaction to a
    recipient (e.g., paying for goods), waits for the recipient to
    accept the payment, and then uses their consensus power to create a
    longer, alternative chain in which the transaction never occurred,
    allowing them to spend the same coins again.
    
    \item \textbf{Breaking network assumptions:}
    Consensus protocols assume some bounds on message delivery and connectivity. Partitions and high delays can violate these assumptions, then safety or liveness can fail even if cryptography remains intact.
    
    \item \textbf{Time computation weaknesses:}
    Nodes often estimate network time from peers. If an attacker can bias a victim's perception of time, the victim's blocks can be rejected due to time constraints. 
    
    \item \textbf{Shard attacks:} become feasible because each shard has fewer validators, so obtaining a local majority can be easier than attacking the whole system.
    
  \end{itemize}

\begin{figure}[t]
	%	\vspace{-0.3cm}
	\begin{center}
		\includegraphics[width=0.63\textwidth]{./figs/attacks-consensus-generic.pdf}
		\caption{VTD graph of the consensus layer -- generic attacks~\cite{homoliak2019security}.}		
		\label{fig:sra-cons-gen}
	\end{center}	
\end{figure}

\item
  \textbf{Attacks on Proof-of-Resource (PoR) Protocols} are depicted in \autoref{fig:sra-cons-por} and interesting ones involve:

  \begin{itemize}
  \tightlist

	\item \textbf{Selfish mining}: the attacker withholds blocks strategically, releases them to invalidate honest work, and gains a higher reward share than their resource share would suggest. This also increases the chance of reorgs, which hurts applications that assume ``a few confirmations is enough.''	
	
  \item
    \textbf{Feather Forking}: An attacker announces they will censor
    certain transactions, creating an incentive for rational miners to
    join the censorship to avoid mining blocks that will be orphaned.
  \item
    \textbf{Bribery Attacks and Transaction Reordering}: Offering direct rewards to miners to
    reorder or exclude transactions, often for the attacker's financial
    benefit (e.g., front-running).
  
  \item \textbf{Time spoofing attacks}: a miner publishes blocks with delayed timestamps to signal that puzzles are ``too hard'' causing difficulty to drop over time, which reduces the attacker's future cost to win blocks.


\begin{figure}[t]
	%	\vspace{-0.3cm}
	\begin{center}
		\includegraphics[width=0.75\textwidth]{./figs/attacks-consensus-PoR.pdf}
		\caption{VTD graph of the consensus layer -- attacks on PoR protocols~\cite{homoliak2019security}.}		
		\label{fig:sra-cons-por}
	\end{center}	
\end{figure}


	\item \textbf{Pool-specific Attacks}
  \begin{itemize}
		  \item
		    \textbf{Pool Hopping}: Miners switch between mining pools to
		    maximize their rewards, exploiting pools that pay per share.
		  \item
		    \textbf{Block Withholding}: A mining pool finds a block but doesn't
		    publish it, sabotaging the main chain.
		    
		   \item \textbf{Lie-in-wait}: delay submission to increase relative shares before release. 
		   		   
		   \item \textbf{Selfish Mining on a Subchain}: exploit decentralized pool mechanics by selectively dropping honest shares. 
	\end{itemize}
  \end{itemize}



\item
  \textbf{Attacks on BFT Protocols}: are depicted in \autoref{fig:sra-cons-bft} and involve: 



  \begin{itemize}
  \tightlist
  \item
    \textbf{DoS on a Leader}: In protocols with a known leader, an
    attacker can perform a DoS attack on the leader to disrupt the
    consensus process.
  \item
    \textbf{Posterior Corruption}: An attacker acquires the keys of a
    supermajority of validators (e.g., 2/3) and uses them to
    illegitimately control the network.
    
    \begin{figure}[!h]
    	%	\vspace{-0.3cm}
    	\begin{center}
    		\includegraphics[width=0.65\textwidth]{./figs/attacks-consensus-BFT.pdf}
    		\caption{VTD graph of the consensus layer -- attacks on BFT protocols~\cite{homoliak2019security}.}		
    		\label{fig:sra-cons-bft}
    	\end{center}	
    \end{figure}
  \end{itemize}





\item
\textbf{Attacks on Proof-of-Stake (PoS) Protocols} are depicted in \autoref{fig:sra-cons-pos} and involve:

\begin{itemize}
	\tightlist
	\item
	\textbf{Nothing-at-Stake}: Validators can vote for multiple
	conflicting blocks without penalty, increasing network forks and
	slowing down finality.
	\item
	\textbf{Grinding Attack}: An attacker attempts to influence the
	selection of future block proposers to increase their own chances of
	being selected.
	\item
	\textbf{Long-Range Attack}: An attacker obtains the keys of early,
	``retired'' validators to create a long alternative chain from the
	genesis block, potentially rewriting the entire history of the
	blockchain.
	
	\item \textbf{DoS on leader or committee}: attackers repeatedly DoS elected leaders or committees until favorable participants appear, which degrades liveness and can bias progress.
\end{itemize}

\end{itemize}


\begin{figure}[t]
%	\vspace{-0.3cm}
\begin{center}
	\includegraphics[width=0.55\textwidth]{./figs/attacks-consensus-PoS.pdf}
	\caption{VTD graph of the consensus layer -- attacks on PoS protocols~\cite{homoliak2019security}.}		
	\label{fig:sra-cons-pos}
\end{center}	
\end{figure}





\hypertarget{replicated-state-machine-rsm-layer-security}{%
\subsection{Replicated State Machine (RSM) Layer
Security}\label{replicated-state-machine-rsm-layer-security}}

The RSM layer executes the logic defined in transactions and smart
contracts. It splits into (1) privacy of users and transaction data and (2) smart contract security and safety. % Flaws at this layer can lead to direct financial loss.


\subsubsection{Transaction Protection and Privacy Threats}
Transactions are signed and verifiable, but identities are pseudonymous and can be traced to IP addresses by an eavesdropping adversary.
Many platforms expose transaction data by default, so confidentiality is not ensured unless additional mechanisms are used.

Privacy threats are depicted in \autoref{fig:sra-rsm-priv} and are as follows:
\begin{itemize}
	
	\item \textbf{Revealing user identities}: Deanonymization techniques include network flow analysis, address clustering, and transaction fingerprinting.
	
	\item \textbf{Revealing sensitive data}: Plaintext transaction payloads, contract inputs, or logged events can disclose business logic, private attributes, or strategic intent, which can then be exploited at the application layer, for example trading strategies.
\end{itemize}



\begin{figure}[t]
	%	\vspace{-0.3cm}
	\begin{center}
		\includegraphics[width=0.55\textwidth]{./figs/attacks-RSM-TXs.pdf}
		\caption{VTD graph of the RSM layer -- privacy threats~\cite{homoliak2019security}.}		
		\label{fig:sra-rsm-priv}
	\end{center}	
\end{figure}


\subsubsection{Smart Contract Threats}
The vast category of vulnerabilities stem from errors in the code of smart contracts (see \autoref{fig:sra-rsm-sc})
Famous examples include:

\begin{itemize}
\tightlist

  \item
    \textbf{Reentrancy}: Where an attacker can repeatedly call back into
    a vulnerable contract before its state is updated, allowing them to
    drain its funds (e.g., The DAO hack~\cite{DAO}).
  \item
    \textbf{Integer Overflow/Underflow}: Where arithmetic operations on
    numbers exceed the maximum or minimum size for the data type,
    causing the value to wrap around, which can be exploited to
    manipulate balances or other critical parameters.
  \item
    \textbf{Access Control Issues}: Where functions that should be
    restricted to certain users (e.g., the contract owner) are left
    public, allowing unauthorized access.
    
  \item \textbf{Unchecked return values}: failure signals ignored, then state updates proceed incorrectly.
  
  \item \textbf{Delegatecall misuse}: untrusted code executes in caller context.
  
  \item \textbf{Weak randomness sources}: block number or timestamp are predictable or biasable.
  
  \item \textbf{Transaction order dependency and front-running}: attackers see pending transactions and win ordering by fees.
  
  \item \textbf{Timestamp dependency}: malicious producers can adjust timestamps within protocol constraints and trigger or prevent events.
  
\end{itemize}

\begin{figure}[t]
	%	\vspace{-0.3cm}
	\begin{center}
		\includegraphics[width=0.55\textwidth]{./figs/attacks-RSM-SC.pdf}
		\caption{VTD graph of the RSM layer -- smart contract threats~\cite{homoliak2019security}.}		
		\label{fig:sra-rsm-sc}
	\end{center}	
\end{figure}


\begin{center}\rule{0.5\linewidth}{0.5pt}\end{center}

\hypertarget{section-3-the-application-layer}{%
\subsection{The Application
Layer}\label{section-3-the-application-layer}}

The application layer is where users interact with the blockchain. The
security of DAPPs depends not only on the robustness of the underlying
layers but also on the design and implementation of the application
itself.
The application layer of SRA is divided into two parts -- ecosystem applications and higher-level applications. 
There exists a hierarchy in inheritance of security aspects across categories of the application layer -- see \autoref{fig:sra-app-hierarchy}.






\hypertarget{ecosystem-applications}{%
\subsubsection{Ecosystem
Applications}\label{ecosystem-applications}}

These are foundational applications that enable the broader blockchain
ecosystem to function. In the following, we briefly introduce them, while for the further details, we refer the reader to~\cite{homoliak2019security}.

\begin{itemize}
\tightlist
\item
  \textbf{Tokens}: All public cryptocurrencies are equipped with the native crypto-tokens. On the other hand smart contract platforms can have programmed tokes, such fungible (ERC-20) and non-fungible (ERC-721) tokens --they are essentially smart contracts that manage balances. They are  subject to all the standard smart contract vulnerabilities.
  The overview of vulnerabilities, threats, and defenses of this category is depicted in \autoref{fig:sra-app-wallets}.  
  %
%  Token types are as follows: 
%  1) \textbf{Native crypto-tokens} that come with the blockchain itself.  
%  2) \textbf{Counter-party tokens} that represent rights against a third party.  
%  3) \textbf{Ownership or colored tokens} that represent transferable asset ownership, examples include NFTs.

  
\item
  \textbf{Wallets}: Software or hardware used to manage private keys.
  Hosted wallets (where a third party holds the keys) introduce
  custodial risk, while self-sovereign wallets place the full
  responsibility for key security on the user, making them vulnerable to
  malware, phishing, and physical loss.
   The overview of vulnerabilities, threats, and defenses of this category is depicted in \autoref{fig:sra-app-wallets}.  
\item
  \textbf{Exchanges}: Both centralized (CEX) and decentralized (DEX)
  exchanges are prime targets for hackers. CEXs are vulnerable to
  traditional web security breaches, while DEXs, being composed of smart
  contracts, are susceptible to exploits that can drain their liquidity
  pools. Atomic swaps are a specific type of cross-chain exchange
  protocol.
   The overview of vulnerabilities, threats, and defenses of this category is depicted in \autoref{fig:sra-app-exchanges}.  
\item
  \textbf{Oracles}: Services that provide external, real-world data to
  smart contracts. The security of oracles is critical, as a compromised
  oracle can feed false data to a smart contract, triggering incorrect
  and potentially catastrophic outcomes.
   The overview of vulnerabilities, threats, and defenses of this category is depicted in \autoref{fig:sra-app-oracles}.  
\item
  \textbf{Distributed Filesystems}: Systems like IPFS and Storj provide
  decentralized data storage. They are vulnerable to attacks such as
  Sybil attacks (a malicious node pretending to be many nodes),
  de-duplication attacks (colluding nodes claiming to store multiple
  copies of data when only one exists), and outsourcing attacks (a node
  claiming to store more data than it actually does).
  The overview of vulnerabilities, threats, and defenses of this category is depicted in \autoref{fig:sra-app-DFs}.  
\end{itemize}

\begin{figure}[t]
	%	\vspace{-0.3cm}
	\begin{center}
		\includegraphics[width=0.55\textwidth]{./figs/dependencies.pdf}
		\caption{Hierarchy of dependencies in security of application layer subcategories~\cite{homoliak2019security}.}		
		\label{fig:sra-app-hierarchy}
	\end{center}	
\end{figure}

\hypertarget{higher-level-applications}{%
\subsubsection{Higher-Level
Applications}\label{higher-level-applications}}

These applications leverage the unique properties of blockchain for
specific use cases.

\begin{itemize}
\tightlist
\item
  \textbf{E-voting}: While blockchain can offer transparency and
  tamper-resistance, implementing a secure e-voting system is fraught
  with challenges, including ensuring voter privacy, preventing
  coercion, and providing a secure and accessible voting mechanism for
  all users.
  The overview of vulnerabilities, threats, and defenses of this category is depicted in \autoref{fig:sra-app-evoting}.  
\item
  \textbf{Reputation Systems}: These systems use the blockchain to
  create a persistent record of user reputations. They are vulnerable to
  Sybil attacks (where a user creates many fake identities to boost
  their reputation) and ``whitewashing'' (where a user with a bad
  reputation simply creates a new identity).
  The overview of vulnerabilities, threats, and defenses of this category is depicted in \autoref{fig:sra-app-REP}.  
\item
  \textbf{Data Provenance / Supply Chain Management}: Using a blockchain to track the
  provenance of goods can enhance transparency. However, the system's integrity
  depends on trusted oracles or IoT devices to ensure that the data
  entered onto the blockchain accurately reflects the state of the
  physical goods.
  The overview of vulnerabilities, threats, and defenses of this category is depicted in \autoref{fig:sra-app-provenance}.  
  
\item
  \textbf{Notarization}: Using the blockchain as a decentralized
  timestamping service to prove the existence of a document at a certain
  point in time. The security here relies on the immutability of the
  underlying blockchain.
  The overview of vulnerabilities, threats, and defenses of this category is depicted in \autoref{fig:sra-app-Notaries}.
\item
  \textbf{Direct Trading}: This involves the exchange of crypto-tokens
  for off-chain goods. It faces the ``buyer/seller dilemma'' and relies
  on the seller's reputation.
  The overview of vulnerabilities, threats, and defenses of this category is depicted in \autoref{fig:sra-app-trading}.
\item
  \textbf{Escrows}: To solve the direct trading problem without a
  trusted seller, a third-party mediator (escrow) can be used to hold
  funds until both parties are satisfied.
  The overview of vulnerabilities, threats, and defenses of this category is depicted in \autoref{fig:sra-app-escrows}.
\item
  \textbf{Auctions}: Blockchain-based auctions need to ensure the
  privacy of bids and be resistant to DoS attacks.
  The overview of vulnerabilities, threats, and defenses of this category is depicted in \autoref{fig:sra-app-auctions}.
\end{itemize}






\begin{center}\rule{0.5\linewidth}{0.5pt}\end{center}

\hypertarget{summary-key-takeaways}{%
\subsection{Summary / Key Takeaways}\label{summary-key-takeaways}}

This chapter has presented the Security Reference Architecture (SRA) as
a structured model for dissecting and analyzing the security of
blockchain systems. By examining the distinct security challenges and
countermeasures at the network, consensus, replicated state machine, and
application layers, we gain a more holistic understanding of the complex
threat landscape of decentralized technology.

We have seen that security in a blockchain is not a monolithic property
but rather a cumulative result of the defenses implemented at each
layer. A vulnerability at a lower layer, such as a network partitioning
attack, can undermine the guarantees of all layers above it. Conversely,
a secure underlying infrastructure is not sufficient to protect against
a poorly coded smart contract at the application layer.

By understanding this layered model, developers, security auditors, and
users can better identify potential risks, implement appropriate
defenses, and make more informed decisions when building and interacting
with decentralized systems.



\begin{center}\rule{0.5\linewidth}{0.5pt}\end{center}

\hypertarget{keywords}{%
	\subsection{Keywords}\label{keywords}}

\begin{itemize}
	\tightlist
	\item
	\textbf{Security Reference Architecture (SRA)}: A layered model for
	analyzing the security of blockchain systems, comprising the network,
	consensus, RSM, and application layers.
	\item
	\textbf{VTD Graph}: A graph that maps Vulnerabilities, Threats, and
	Defenses.
	\item
	\textbf{Network Layer}: The foundational layer of the SRA responsible
	for peer-to-peer communication between nodes.
	\item
	\textbf{Consensus Layer}: The layer of the SRA responsible for
	achieving agreement on the order and validity of transactions.
	\item
	\textbf{Replicated State Machine (RSM) Layer}: The layer of the SRA
	that interprets transactions and executes smart contracts to update
	the blockchain's state.
	\item
	\textbf{Application Layer}: The highest layer of the SRA, consisting
	of the DApps and user-facing services built on the blockchain.
	\item
	\textbf{Eclipse Attack}: A network-level attack where an adversary
	isolates a node by controlling all of its network connections.
	\item
	\textbf{51\% Attack}: A consensus-level attack where an entity
	controlling a majority of the network's consensus power can manipulate
	the blockchain.
	\item
	\textbf{Reentrancy}: A common smart contract vulnerability where an
	attacker can repeatedly call a function before the contract's state is
	updated.
\end{itemize}

\begin{center}\rule{0.5\linewidth}{0.5pt}\end{center}

\hypertarget{further-reading}{%
	\subsection{Further Reading}\label{further-reading}}

\begin{itemize}
	\tightlist
	\item
	Homoliak, Ivan, et al. "The security reference architecture for blockchains: Toward a standardized model for studying vulnerabilities, threats, and defenses." IEEE Communications Surveys \& Tutorials 23.1 (2020): 341-390.
	\item
	ISO/IEC 15408-1:2009, \emph{Information technology --- Security
		techniques --- Evaluation criteria for IT security --- Part 1:
		Introduction and general model}.
\end{itemize}


\begin{figure}[h]
	%	\vspace{-0.3cm}
	\begin{center}
		\includegraphics[width=0.75\textwidth]{./figs/attacks-APP-wallets.pdf}
		\caption{VTD graph of the tokens and wallets category~\cite{homoliak2019security}.}		
		\label{fig:sra-app-wallets}
	\end{center}	
\end{figure}

\clearpage



\begin{figure}
	%	\vspace{-0.3cm}
	\begin{center}
		\includegraphics[width=0.75\textwidth]{./figs/attacks-APP-exchanges.pdf}
		\caption{VTD graph of the exchanges category~\cite{homoliak2019security}.}		
		\label{fig:sra-app-exchanges}
	\end{center}	
\end{figure}


\begin{figure}
	%	\vspace{-0.3cm}
	\begin{center}
		\includegraphics[width=0.75\textwidth]{./figs/attacks-APP-oracles.pdf}
		\caption{VTD graph of the oracles category~\cite{homoliak2019security}.}		
		\label{fig:sra-app-oracles}
	\end{center}	
\end{figure}

\begin{figure}
	%	\vspace{-0.3cm}
	\begin{center}
		\includegraphics[width=0.75\textwidth]{./figs/attacks-APP-DFs}
		\caption{VTD graph of the decentralized filesystems category~\cite{homoliak2019security}.}		
		\label{fig:sra-app-DFs}
	\end{center}	
\end{figure}


\begin{figure}
	%	\vspace{-0.3cm}
	\begin{center}
		\includegraphics[width=0.55\textwidth]{./figs/attacks-APP-evoting.pdf}
		\caption{VTD graph of the e-voting category~\cite{homoliak2019security}.}		
		\label{fig:sra-app-evoting}
	\end{center}	
\end{figure}


\begin{figure}
	%	\vspace{-0.3cm}
	\begin{center}
		\includegraphics[width=0.55\textwidth]{./figs/attacks-APP-reputation.pdf}
		\caption{VTD graph of the reputation systems category~\cite{homoliak2019security}.}		
		\label{fig:sra-app-REP}
	\end{center}	
\end{figure}

\begin{figure}
	%	\vspace{-0.3cm}
	\begin{center}
		\includegraphics[width=0.55\textwidth]{./figs/attacks-APP-provenance.pdf}
		\caption{VTD graph of the data provenance category~\cite{homoliak2019security}.}		
		\label{fig:sra-app-provenance}
	\end{center}	
\end{figure}


\begin{figure}
	%	\vspace{-0.3cm}
	\begin{center}
		\includegraphics[width=0.55\textwidth]{./figs/attacks-APP-notaries.pdf}
		\caption{VTD graph of the notaries category~\cite{homoliak2019security}.}		
		\label{fig:sra-app-Notaries}
	\end{center}	
\end{figure}


\begin{figure}
	%	\vspace{-0.3cm}
	\begin{center}
		\includegraphics[width=0.55\textwidth]{./figs/attacks-APP-trading.pdf}
		\caption{VTD graph of the direct trading category~\cite{homoliak2019security}.}		
		\label{fig:sra-app-trading}
	\end{center}	
\end{figure}


\begin{figure}
	%	\vspace{-0.3cm}
	\begin{center}
		\includegraphics[width=0.55\textwidth]{./figs/attacks-APP-escrows.pdf}
		\caption{VTD graph of the escrows category~\cite{homoliak2019security}.}		
		\label{fig:sra-app-escrows}
	\end{center}	
\end{figure}


\begin{figure}
	%	\vspace{-0.3cm}
	\begin{center}
		\includegraphics[width=0.55\textwidth]{./figs/attacks-APP-auctions.pdf}
		\caption{VTD graph of the auctions category~\cite{homoliak2019security}.}		
		\label{fig:sra-app-auctions}
	\end{center}	
\end{figure}





%\chapter{tmp}\label{chapter:intro}
%BLA BLA~\cite{homoliak2019security}



%%%%%%%%%%%%%%%%%%%%%%%%%%%%%%%%%%%%%%%%%%%%%%%%%%%%%%%%%%%%%%%%%%%%%%%%%%%%%%%
\clearpage
\addcontentsline{toc}{section}{References}

%\section{Bibliography}


\bibliographystyle{plainurl}
%\bibliographystyle{alpha}
%\bibliography{./refs/ref,./refs/ref2,./refs/ref3,./refs/ref4,./refs/ref5,./refs/ref6,./refs/ref7,./refs/ref8,./refs/ref9}


\bibliography{unique_identifiers}
%%%%%%%%%%%%%%%%%%%%%%%%%%%%%%%%%%%%%%%%%%%%%%%%%%%%%%%%%%%%%%%%%%%%%%%%%%%%%%%




\pagenumbering{gobble}


%%%%%%%%%%%%%%%%%%%%%%%%%%%%%%%%%%%%%%%%%%%%%%%%%%%%%%%%%%%%%%%%%%%%%%%%%%%%%%%
\end{document} 
%%%%%%%%%%%%%%%%%%%%%%%%%%%%%%%%%%%%%%%%%%%%%%%%%%%%%%%%%%%%%%%%%%%%%%%%%%%%%%%
