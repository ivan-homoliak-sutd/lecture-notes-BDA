%------------------------------------------------------------------------------
\section*{Abstract}
%------------------------------------------------------------------------------
The course Blockchain and Decentralized Applications (BDA) aims to acquaint students with the principles and protocols in fully decentralized (P2P) network communication. While aspects of client-server communication are important, the less traditional but emerging peer-to-peer blockchain scheme and its integration into the Internet is an alternative that allows us to achieve unique features in terms of availability, transparency, and trust. This course focuses on the technical aspects of blockchain systems, smart contracts, and decentralized applications. Students will learn how these systems are built, how to communicate with them, and how to design \& create secure decentralized applications. Students will also exercise the acquired knowledge in practice through a semestral assignment.
Students will learn advanced theoretical and practical knowledge in the field of decentralized computing platforms, their types, consensual protocols, and problems associated with them. Further, students will learn knowledge of terminology, unique properties of blockchain, knowledge of advanced integrity-preserving data structures and algorithms used in blockchains and smart contract platforms. the next goal is to present practical use cases and their potential vulnerabilities. The course also deals with the problem of scalability and anonymity and variants of their solution. In sum, students should obtain the ability to design, deploy, and manage custom decentralized applications and solutions.
%------------------------------------------------------------------------------
\section*{Keywords}
%------------------------------------------------------------------------------
Decentralized platforms, blockchains, integrity-preserving data structures, smart contracts, decentralized applications, DAPPs, consensus protocols, security threats, scalability, anonymity, and privacy.
%Security, privacy, blockchains, distributed ledgers, threat modeling, standardization, decentralized applications, consensus protocols, proof-of-work, selfish mining attacks, undercutting attacks, incentive attacks, DAG-based blockchains, cryptocurrency wallets, two-factor authentication, 2FA, electronic voting, e-voting, public bulletin board, secure logging, data provenance, interoperability, Central Bank Digital Currency, CBDC, Trusted Execution Environment, TEE, cross-chain protocol.

