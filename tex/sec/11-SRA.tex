
\hypertarget{introduction}{%
\subsection{Introduction}\label{introduction}}

This chapter introduces a layered security model for
blockchain technology, providing a structured framework for analyzing
the diverse security and privacy challenges inherent in decentralized
systems. Drawing an analogy to the well-established ISO/OSI model for
computer networks, this framework, termed the Security Reference
Architecture (SRA)~\cite{homoliak2019security}, deconstructs the complex blockchain ecosystem into
four distinct layers: \textbf{the network layer, the consensus
layer, the replicated state machine (RSM) layer, and the application
layer.}

Throughout this chapter, we will systematically explore the unique
security threats, vulnerabilities, and potential countermeasures
pertinent to each layer. This structured approach allows for a granular
understanding of how security is enforced at different levels of the
blockchain stack, from the fundamental peer-to-peer communication
protocols to the sophisticated decentralized applications (DApps) that
users interact with.

Furthermore, we will examine a broad spectrum of blockchain
applications, categorizing them from foundational ecosystem components,
such as wallets and tokens, to higher-level, real-world use cases like
e-voting, supply chain management, and digital notarization. By applying
the SRA framework to these applications, we can better appreciate their
specific security considerations and the interplay between the different
layers in ensuring their overall resilience and integrity.

\hypertarget{learning-objectives}{%
\subsection{Learning Objectives}\label{learning-objectives}}

\begin{itemize}
\tightlist
\item
  Understand the concept and utility of a layered security model for
  analyzing blockchain systems.
\item
  Identify the four layers of the Security Reference Architecture (SRA):
  network, consensus, replicated state machine, and application.
\item
  Recognize and describe the primary security threats and defensive
  mechanisms at the network layer, including DNS attacks, routing
  attacks, and eclipse attacks.
\item
  Analyze the security vulnerabilities at the consensus layer, such as
  51\% attacks, double-spending, and selfish mining.
\item
  Evaluate the security risks at the replicated state machine (RSM)
  layer, with a focus on smart contract bugs and data privacy.
\item
  Categorize and explore the security considerations for a wide range of
  blockchain applications, from ecosystem-level tools to higher-level
  use cases.
\end{itemize}

\begin{center}\rule{0.5\linewidth}{0.5pt}\end{center}

\hypertarget{section-1-the-security-reference-architecture-sra-for-blockchains}{%
\subsection{The Security Reference Architecture for Blockchains}\label{section-1-the-security-reference-architecture-sra-for-blockchains}}

\hypertarget{a-layered-approach-to-blockchain-security}{%
\subsubsection{A Layered Approach to Blockchain
Security}\label{a-layered-approach-to-blockchain-security}}

The Security Reference Architecture (SRA) for blockchains~\cite{homoliak2020security} provides a
conceptual model for organizing and understanding the multifaceted
security aspects of decentralized ledger technology. By dividing the
system into layers, we can isolate and analyze specific functionalities
and their associated vulnerabilities. The four layers of the SRA are (see also \autoref{fig:SRA-overview}):


\begin{figure}[t]
	%	\vspace{-0.3cm}
	\begin{center}
		\includegraphics[width=0.55\textwidth]{./figs/overview-SRA.pdf}
		\caption{Overview of SRA~\cite{homoliak2020security}.}		
		\label{fig:SRA-overview}
	\end{center}	
\end{figure}

\begin{itemize}
\tightlist
\item
  \textbf{Network Layer}: This foundational layer deals with the
  peer-to-peer (P2P) communication protocols that enable nodes to
  connect, discover each other, and exchange data. It encompasses
  everything from peer discovery and message routing to the underlying
  Internet protocols upon which the blockchain network is built.
\item
  \textbf{Consensus Layer}: This is the core of the blockchain,
  responsible for the mechanism by which all nodes in the network agree
  on a single, consistent, and immutable order of transactions. This
  layer includes the specific consensus algorithms, such as
  Proof-of-Work (PoW) or Proof-of-Stake (PoS), that ensure the integrity
  of the ledger.
\item
  \textbf{Replicated State Machine (RSM) Layer}: This layer is
  responsible for interpreting the ordered transactions provided by the
  consensus layer and executing them to update the state of the
  blockchain. It includes the virtual machine (e.g., the EVM) that runs
  smart contracts and the logic for processing transactions.
\item
  \textbf{Application Layer}: This is the topmost layer, consisting of
  the decentralized applications (DApps), user interfaces, and external
  services (like oracles) that are built on top of the blockchain
  infrastructure to provide specific functionalities to end-users.
\end{itemize}


\begin{figure}[t]
	%	\vspace{-0.3cm}
	\begin{center}
		\includegraphics[width=0.95\textwidth]{./figs/threat-risk-assessment.png}
		\caption{Threat-risk assessment according to ISO/IEC15408.}		
		\label{fig:thread-risk-assessment}
	\end{center}	
\end{figure}


\hypertarget{threat-risk-assessment-model-for-blockchains}{%
\subsubsection{Threat Risk Assessment Model for
Blockchains}\label{threat-risk-assessment-model-for-blockchains}}

The SRA framework is conceptually grounded in the Threat Risk Assessment
(TRA) model, which is a standard methodology for security analysis of centralized systems,
notably detailed in the ISO 15408 (Common Criteria) standard (see \autoref{fig:thread-risk-assessment}). 
The TRA
model provides a systematic process for: 
\begin{enumerate}
	\item Identifying valuable
	\textbf{assets} within the system. 
	
	\item  Recognizing potential
	\textbf{threats} to those assets. 
	
	\item  Assessing the
	\textbf{vulnerabilities} that could be exploited by those threats. 
	
	\item  Evaluating the \textbf{risks} associated with these vulnerabilities. 

	\item  Implementing \textbf{countermeasures} to mitigate the identified risks
	to an acceptable level.
	
\end{enumerate}
The projection of SRA to TRA is depicted in \autoref{fig:thread-risk-assessment-SRA}, and it enriches TRA with the environment of blockchain and its specific aspects.


To better visualize these relationships in the context of blockchains,
we can use \textbf{Vulnerability, Threat, and Defense (VTD) graphs}.
These graphs map out the connections between specific vulnerabilities,
the threats that exploit them, and the defenses designed to counter
them, providing a clear and structured view of the security landscape at
each layer.


\begin{figure}[t]
	%	\vspace{-0.3cm}
	\begin{center}
		\includegraphics[width=0.55\textwidth]{./figs/ISO-SRA.pdf}
		\caption{Threat-risk assessment model adapted to SRA~\cite{homoliak2019security}.}		
		\label{fig:thread-risk-assessment-SRA}
	\end{center}	
\end{figure}


\begin{center}\rule{0.5\linewidth}{0.5pt}\end{center}

%\hypertarget{section-2-security-at-each-layer}{%
%\subsection{Security at Each
%Layer}\label{section-2-security-at-each-layer}}

\hypertarget{network-layer-security}{%
\subsection{Network Layer}\label{network-layer-security}}

The network layer is the backbone of any blockchain, and its security is
paramount. As blockchain networks are typically overlays on top of the
public Internet, they inherit many of its vulnerabilities (see also \autoref{fig:vtd-network-public}) such as:

\begin{itemize}
\tightlist
\item
  \textbf{DNS Attacks}: Many blockchain nodes initially find peers
  through DNS seeds. An attacker who can compromise these DNS records
  (e.g., through DNS cache poisoning) can redirect new nodes to a
  network partition controlled by the attacker.
\item
  \textbf{Routing Attacks}: Attacks on the Internet's routing
  infrastructure, such as BGP hijacking, can be used to intercept or
  partition blockchain network traffic, isolating certain regions or
  nodes from the rest of the network.
\item
  \textbf{Eclipse Attacks}: A more targeted attack where an adversary
  monopolizes all of a victim node's incoming and outgoing connections.
  The eclipsed node is thus isolated from the honest network and can be
  fed false information, potentially leading it to accept fraudulent
  transactions or waste its consensus power on an invalid chain.
  
  \begin{figure}[t]
  	%	\vspace{-0.3cm}
  	\begin{center}
  		\includegraphics[width=0.55\textwidth]{./figs/attacks-network-public.pdf}
  		\caption{VTD graph of the network layer -- public networks~\cite{homoliak2019security}.}		
  		\label{fig:vtd-network-public}
  	\end{center}	
  \end{figure}
  
  
  \begin{figure}[t]
  	%	\vspace{-0.3cm}
  	\begin{center}
  		\includegraphics[width=0.55\textwidth]{./figs/attacks-network-private.pdf}
  		\caption{VTD graph of the network layer -- private networks~\cite{homoliak2019security}.}		
  		\label{fig:vtd-network-private}
  	\end{center}	
  \end{figure}
  
  
\item
  \textbf{Denial-of-Service (DoS) Attacks}: These attacks aim to disrupt
  the network by overwhelming nodes with traffic or resource-intensive
  requests. A specific example is the ``penny-flooding'' attack, where
  an attacker spams the network with a large number of low-fee
  transactions to clog the mempool (the holding area for unconfirmed
  transactions) and potentially crash nodes -- these are DoS attacks on local resources.
  DoS attacks on network connectivity reduce effective consensus power and reward opportunities for honest nodes. 
  
  \item \textbf{Identity Revealing Attacks} are aimed at linking IP address with the blockchain address extracted from a transaction.
  
  
\end{itemize}




Besides public network of permissionless and semi-permissionless blockchains, there exist private networks of permissioned blockchains, which might contain different types of vulnerabilities and threats (see \autoref{fig:vtd-network-private}) related to centralization of control, such as external targeted attacks and insider threat~\cite{homoliak2018insight}.



\hypertarget{consensus-layer-security}{%
\subsection{Consensus Layer}\label{consensus-layer-security}}

The consensus layer is where the ``truth'' of the blockchain is decided through inclusion and ordering of transactions,
making it a high-value target for attackers.
The potential threats of this layer can be divided according to the type of consensus protocol as follows.






\begin{itemize}
\tightlist
\item
  \textbf{Generic Attacks} are depicted in \autoref{fig:sra-cons-gen} and they mostly involve:

  \begin{itemize}
  \tightlist
  \item
    \textbf{51\% Attacks}: The most well-known consensus attack, where
    an attacker (or a coalition of attackers) controls a majority of the
    network's consensus power (e.g., hashing power in PoW, or stake in
    PoS). With this control, the attacker can prevent new transactions
    from gaining confirmations, halt payments between some or all users,
    and reverse transactions that were completed while they were in
    control (double-spending).
  \item
    \textbf{Double-Spending Attacks}: This is a primary goal of many
    consensus-level attacks. An attacker sends a transaction to a
    recipient (e.g., paying for goods), waits for the recipient to
    accept the payment, and then uses their consensus power to create a
    longer, alternative chain in which the transaction never occurred,
    allowing them to spend the same coins again.
    
    \item \textbf{Breaking network assumptions:}
    Consensus protocols assume some bounds on message delivery and connectivity. Partitions and high delays can violate these assumptions, then safety or liveness can fail even if cryptography remains intact.
    
    \item \textbf{Time computation weaknesses:}
    Nodes often estimate network time from peers. If an attacker can bias a victim's perception of time, the victim's blocks can be rejected due to time constraints. 
    
    \item \textbf{Shard attacks:} become feasible because each shard has fewer validators, so obtaining a local majority can be easier than attacking the whole system.
    
  \end{itemize}

\begin{figure}[t]
	%	\vspace{-0.3cm}
	\begin{center}
		\includegraphics[width=0.63\textwidth]{./figs/attacks-consensus-generic.pdf}
		\caption{VTD graph of the consensus layer -- generic attacks~\cite{homoliak2019security}.}		
		\label{fig:sra-cons-gen}
	\end{center}	
\end{figure}

\item
  \textbf{Attacks on Proof-of-Resource (PoR) Protocols} are depicted in \autoref{fig:sra-cons-por} and interesting ones involve:

  \begin{itemize}
  \tightlist

	\item \textbf{Selfish mining}: the attacker withholds blocks strategically, releases them to invalidate honest work, and gains a higher reward share than their resource share would suggest. This also increases the chance of reorgs, which hurts applications that assume ``a few confirmations is enough.''	
	
  \item
    \textbf{Feather Forking}: An attacker announces they will censor
    certain transactions, creating an incentive for rational miners to
    join the censorship to avoid mining blocks that will be orphaned.
  \item
    \textbf{Bribery Attacks and Transaction Reordering}: Offering direct rewards to miners to
    reorder or exclude transactions, often for the attacker's financial
    benefit (e.g., front-running).
  
  \item \textbf{Time spoofing attacks}: a miner publishes blocks with delayed timestamps to signal that puzzles are ``too hard'' causing difficulty to drop over time, which reduces the attacker's future cost to win blocks.


\begin{figure}[t]
	%	\vspace{-0.3cm}
	\begin{center}
		\includegraphics[width=0.75\textwidth]{./figs/attacks-consensus-PoR.pdf}
		\caption{VTD graph of the consensus layer -- attacks on PoR protocols~\cite{homoliak2019security}.}		
		\label{fig:sra-cons-por}
	\end{center}	
\end{figure}


	\item \textbf{Pool-specific Attacks}
  \begin{itemize}
		  \item
		    \textbf{Pool Hopping}: Miners switch between mining pools to
		    maximize their rewards, exploiting pools that pay per share.
		  \item
		    \textbf{Block Withholding}: A mining pool finds a block but doesn't
		    publish it, sabotaging the main chain.
		    
		   \item \textbf{Lie-in-wait}: delay submission to increase relative shares before release. 
		   		   
		   \item \textbf{Selfish Mining on a Subchain}: exploit decentralized pool mechanics by selectively dropping honest shares. 
	\end{itemize}
  \end{itemize}



\item
  \textbf{Attacks on BFT Protocols}: are depicted in \autoref{fig:sra-cons-bft} and involve: 



  \begin{itemize}
  \tightlist
  \item
    \textbf{DoS on a Leader}: In protocols with a known leader, an
    attacker can perform a DoS attack on the leader to disrupt the
    consensus process.
  \item
    \textbf{Posterior Corruption}: An attacker acquires the keys of a
    supermajority of validators (e.g., 2/3) and uses them to
    illegitimately control the network.
    
    \begin{figure}[!h]
    	%	\vspace{-0.3cm}
    	\begin{center}
    		\includegraphics[width=0.65\textwidth]{./figs/attacks-consensus-BFT.pdf}
    		\caption{VTD graph of the consensus layer -- attacks on BFT protocols~\cite{homoliak2019security}.}		
    		\label{fig:sra-cons-bft}
    	\end{center}	
    \end{figure}
  \end{itemize}





\item
\textbf{Attacks on Proof-of-Stake (PoS) Protocols} are depicted in \autoref{fig:sra-cons-pos} and involve:

\begin{itemize}
	\tightlist
	\item
	\textbf{Nothing-at-Stake}: Validators can vote for multiple
	conflicting blocks without penalty, increasing network forks and
	slowing down finality.
	\item
	\textbf{Grinding Attack}: An attacker attempts to influence the
	selection of future block proposers to increase their own chances of
	being selected.
	\item
	\textbf{Long-Range Attack}: An attacker obtains the keys of early,
	``retired'' validators to create a long alternative chain from the
	genesis block, potentially rewriting the entire history of the
	blockchain.
	
	\item \textbf{DoS on leader or committee}: attackers repeatedly DoS elected leaders or committees until favorable participants appear, which degrades liveness and can bias progress.
\end{itemize}

\end{itemize}


\begin{figure}[t]
%	\vspace{-0.3cm}
\begin{center}
	\includegraphics[width=0.55\textwidth]{./figs/attacks-consensus-PoS.pdf}
	\caption{VTD graph of the consensus layer -- attacks on PoS protocols~\cite{homoliak2019security}.}		
	\label{fig:sra-cons-pos}
\end{center}	
\end{figure}





\hypertarget{replicated-state-machine-rsm-layer-security}{%
\subsection{Replicated State Machine (RSM) Layer
Security}\label{replicated-state-machine-rsm-layer-security}}

The RSM layer executes the logic defined in transactions and smart
contracts. It splits into (1) privacy of users and transaction data and (2) smart contract security and safety. % Flaws at this layer can lead to direct financial loss.


\subsubsection{Transaction Protection and Privacy Threats}
Transactions are signed and verifiable, but identities are pseudonymous and can be traced to IP addresses by an eavesdropping adversary.
Many platforms expose transaction data by default, so confidentiality is not ensured unless additional mechanisms are used.

Privacy threats are depicted in \autoref{fig:sra-rsm-priv} and are as follows:
\begin{itemize}
	
	\item \textbf{Revealing user identities}: Deanonymization techniques include network flow analysis, address clustering, and transaction fingerprinting.
	
	\item \textbf{Revealing sensitive data}: Plaintext transaction payloads, contract inputs, or logged events can disclose business logic, private attributes, or strategic intent, which can then be exploited at the application layer, for example trading strategies.
\end{itemize}



\begin{figure}[t]
	%	\vspace{-0.3cm}
	\begin{center}
		\includegraphics[width=0.55\textwidth]{./figs/attacks-RSM-TXs.pdf}
		\caption{VTD graph of the RSM layer -- privacy threats~\cite{homoliak2019security}.}		
		\label{fig:sra-rsm-priv}
	\end{center}	
\end{figure}


\subsubsection{Smart Contract Threats}
The vast category of vulnerabilities stem from errors in the code of smart contracts (see \autoref{fig:sra-rsm-sc})
Famous examples include:

\begin{itemize}
\tightlist

  \item
    \textbf{Reentrancy}: Where an attacker can repeatedly call back into
    a vulnerable contract before its state is updated, allowing them to
    drain its funds (e.g., The DAO hack~\cite{DAO}).
  \item
    \textbf{Integer Overflow/Underflow}: Where arithmetic operations on
    numbers exceed the maximum or minimum size for the data type,
    causing the value to wrap around, which can be exploited to
    manipulate balances or other critical parameters.
  \item
    \textbf{Access Control Issues}: Where functions that should be
    restricted to certain users (e.g., the contract owner) are left
    public, allowing unauthorized access.
    
  \item \textbf{Unchecked return values}: failure signals ignored, then state updates proceed incorrectly.
  
  \item \textbf{Delegatecall misuse}: untrusted code executes in caller context.
  
  \item \textbf{Weak randomness sources}: block number or timestamp are predictable or biasable.
  
  \item \textbf{Transaction order dependency and front-running}: attackers see pending transactions and win ordering by fees.
  
  \item \textbf{Timestamp dependency}: malicious producers can adjust timestamps within protocol constraints and trigger or prevent events.
  
\end{itemize}

\begin{figure}[t]
	%	\vspace{-0.3cm}
	\begin{center}
		\includegraphics[width=0.55\textwidth]{./figs/attacks-RSM-SC.pdf}
		\caption{VTD graph of the RSM layer -- smart contract threats~\cite{homoliak2019security}.}		
		\label{fig:sra-rsm-sc}
	\end{center}	
\end{figure}


\begin{center}\rule{0.5\linewidth}{0.5pt}\end{center}

\hypertarget{section-3-the-application-layer}{%
\subsection{The Application
Layer}\label{section-3-the-application-layer}}

The application layer is where users interact with the blockchain. The
security of DAPPs depends not only on the robustness of the underlying
layers but also on the design and implementation of the application
itself.
The application layer of SRA is divided into two parts -- ecosystem applications and higher-level applications. 
There exists a hierarchy in inheritance of security aspects across categories of the application layer -- see \autoref{fig:sra-app-hierarchy}.






\hypertarget{ecosystem-applications}{%
\subsubsection{Ecosystem
Applications}\label{ecosystem-applications}}

These are foundational applications that enable the broader blockchain
ecosystem to function. In the following, we briefly introduce them, while for the further details, we refer the reader to~\cite{homoliak2019security}.

\begin{itemize}
\tightlist
\item
  \textbf{Tokens}: All public cryptocurrencies are equipped with the native crypto-tokens. On the other hand smart contract platforms can have programmed tokes, such fungible (ERC-20) and non-fungible (ERC-721) tokens --they are essentially smart contracts that manage balances. They are  subject to all the standard smart contract vulnerabilities.
  The overview of vulnerabilities, threats, and defenses of this category is depicted in \autoref{fig:sra-app-wallets}.  
  %
%  Token types are as follows: 
%  1) \textbf{Native crypto-tokens} that come with the blockchain itself.  
%  2) \textbf{Counter-party tokens} that represent rights against a third party.  
%  3) \textbf{Ownership or colored tokens} that represent transferable asset ownership, examples include NFTs.

  
\item
  \textbf{Wallets}: Software or hardware used to manage private keys.
  Hosted wallets (where a third party holds the keys) introduce
  custodial risk, while self-sovereign wallets place the full
  responsibility for key security on the user, making them vulnerable to
  malware, phishing, and physical loss.
   The overview of vulnerabilities, threats, and defenses of this category is depicted in \autoref{fig:sra-app-wallets}.  
\item
  \textbf{Exchanges}: Both centralized (CEX) and decentralized (DEX)
  exchanges are prime targets for hackers. CEXs are vulnerable to
  traditional web security breaches, while DEXs, being composed of smart
  contracts, are susceptible to exploits that can drain their liquidity
  pools. Atomic swaps are a specific type of cross-chain exchange
  protocol.
   The overview of vulnerabilities, threats, and defenses of this category is depicted in \autoref{fig:sra-app-exchanges}.  
\item
  \textbf{Oracles}: Services that provide external, real-world data to
  smart contracts. The security of oracles is critical, as a compromised
  oracle can feed false data to a smart contract, triggering incorrect
  and potentially catastrophic outcomes.
   The overview of vulnerabilities, threats, and defenses of this category is depicted in \autoref{fig:sra-app-oracles}.  
\item
  \textbf{Distributed Filesystems}: Systems like IPFS and Storj provide
  decentralized data storage. They are vulnerable to attacks such as
  Sybil attacks (a malicious node pretending to be many nodes),
  de-duplication attacks (colluding nodes claiming to store multiple
  copies of data when only one exists), and outsourcing attacks (a node
  claiming to store more data than it actually does).
  The overview of vulnerabilities, threats, and defenses of this category is depicted in \autoref{fig:sra-app-DFs}.  
\end{itemize}

\begin{figure}[t]
	%	\vspace{-0.3cm}
	\begin{center}
		\includegraphics[width=0.55\textwidth]{./figs/dependencies.pdf}
		\caption{Hierarchy of dependencies in security of application layer subcategories~\cite{homoliak2019security}.}		
		\label{fig:sra-app-hierarchy}
	\end{center}	
\end{figure}

\hypertarget{higher-level-applications}{%
\subsubsection{Higher-Level
Applications}\label{higher-level-applications}}

These applications leverage the unique properties of blockchain for
specific use cases.

\begin{itemize}
\tightlist
\item
  \textbf{E-voting}: While blockchain can offer transparency and
  tamper-resistance, implementing a secure e-voting system is fraught
  with challenges, including ensuring voter privacy, preventing
  coercion, and providing a secure and accessible voting mechanism for
  all users.
  The overview of vulnerabilities, threats, and defenses of this category is depicted in \autoref{fig:sra-app-evoting}.  
\item
  \textbf{Reputation Systems}: These systems use the blockchain to
  create a persistent record of user reputations. They are vulnerable to
  Sybil attacks (where a user creates many fake identities to boost
  their reputation) and ``whitewashing'' (where a user with a bad
  reputation simply creates a new identity).
  The overview of vulnerabilities, threats, and defenses of this category is depicted in \autoref{fig:sra-app-REP}.  
\item
  \textbf{Data Provenance / Supply Chain Management}: Using a blockchain to track the
  provenance of goods can enhance transparency. However, the system's integrity
  depends on trusted oracles or IoT devices to ensure that the data
  entered onto the blockchain accurately reflects the state of the
  physical goods.
  The overview of vulnerabilities, threats, and defenses of this category is depicted in \autoref{fig:sra-app-provenance}.  
  
\item
  \textbf{Notarization}: Using the blockchain as a decentralized
  timestamping service to prove the existence of a document at a certain
  point in time. The security here relies on the immutability of the
  underlying blockchain.
  The overview of vulnerabilities, threats, and defenses of this category is depicted in \autoref{fig:sra-app-Notaries}.
\item
  \textbf{Direct Trading}: This involves the exchange of crypto-tokens
  for off-chain goods. It faces the ``buyer/seller dilemma'' and relies
  on the seller's reputation.
  The overview of vulnerabilities, threats, and defenses of this category is depicted in \autoref{fig:sra-app-trading}.
\item
  \textbf{Escrows}: To solve the direct trading problem without a
  trusted seller, a third-party mediator (escrow) can be used to hold
  funds until both parties are satisfied.
  The overview of vulnerabilities, threats, and defenses of this category is depicted in \autoref{fig:sra-app-escrows}.
\item
  \textbf{Auctions}: Blockchain-based auctions need to ensure the
  privacy of bids and be resistant to DoS attacks.
  The overview of vulnerabilities, threats, and defenses of this category is depicted in \autoref{fig:sra-app-auctions}.
\end{itemize}






\begin{center}\rule{0.5\linewidth}{0.5pt}\end{center}

\hypertarget{summary-key-takeaways}{%
\subsection{Summary / Key Takeaways}\label{summary-key-takeaways}}

This chapter has presented the Security Reference Architecture (SRA) as
a structured model for dissecting and analyzing the security of
blockchain systems. By examining the distinct security challenges and
countermeasures at the network, consensus, replicated state machine, and
application layers, we gain a more holistic understanding of the complex
threat landscape of decentralized technology.

We have seen that security in a blockchain is not a monolithic property
but rather a cumulative result of the defenses implemented at each
layer. A vulnerability at a lower layer, such as a network partitioning
attack, can undermine the guarantees of all layers above it. Conversely,
a secure underlying infrastructure is not sufficient to protect against
a poorly coded smart contract at the application layer.

By understanding this layered model, developers, security auditors, and
users can better identify potential risks, implement appropriate
defenses, and make more informed decisions when building and interacting
with decentralized systems.



\begin{center}\rule{0.5\linewidth}{0.5pt}\end{center}

\hypertarget{keywords}{%
	\subsection{Keywords}\label{keywords}}

\begin{itemize}
	\tightlist
	\item
	\textbf{Security Reference Architecture (SRA)}: A layered model for
	analyzing the security of blockchain systems, comprising the network,
	consensus, RSM, and application layers.
	\item
	\textbf{VTD Graph}: A graph that maps Vulnerabilities, Threats, and
	Defenses.
	\item
	\textbf{Network Layer}: The foundational layer of the SRA responsible
	for peer-to-peer communication between nodes.
	\item
	\textbf{Consensus Layer}: The layer of the SRA responsible for
	achieving agreement on the order and validity of transactions.
	\item
	\textbf{Replicated State Machine (RSM) Layer}: The layer of the SRA
	that interprets transactions and executes smart contracts to update
	the blockchain's state.
	\item
	\textbf{Application Layer}: The highest layer of the SRA, consisting
	of the DApps and user-facing services built on the blockchain.
	\item
	\textbf{Eclipse Attack}: A network-level attack where an adversary
	isolates a node by controlling all of its network connections.
	\item
	\textbf{51\% Attack}: A consensus-level attack where an entity
	controlling a majority of the network's consensus power can manipulate
	the blockchain.
	\item
	\textbf{Reentrancy}: A common smart contract vulnerability where an
	attacker can repeatedly call a function before the contract's state is
	updated.
\end{itemize}

\begin{center}\rule{0.5\linewidth}{0.5pt}\end{center}

\hypertarget{further-reading}{%
	\subsection{Further Reading}\label{further-reading}}

\begin{itemize}
	\tightlist
	\item
	Homoliak, Ivan, et al. "The security reference architecture for blockchains: Toward a standardized model for studying vulnerabilities, threats, and defenses." IEEE Communications Surveys \& Tutorials 23.1 (2020): 341-390.
	\item
	ISO/IEC 15408-1:2009, \emph{Information technology --- Security
		techniques --- Evaluation criteria for IT security --- Part 1:
		Introduction and general model}.
\end{itemize}


\begin{figure}[h]
	%	\vspace{-0.3cm}
	\begin{center}
		\includegraphics[width=0.75\textwidth]{./figs/attacks-APP-wallets.pdf}
		\caption{VTD graph of the tokens and wallets category~\cite{homoliak2019security}.}		
		\label{fig:sra-app-wallets}
	\end{center}	
\end{figure}

\clearpage



\begin{figure}
	%	\vspace{-0.3cm}
	\begin{center}
		\includegraphics[width=0.75\textwidth]{./figs/attacks-APP-exchanges.pdf}
		\caption{VTD graph of the exchanges category~\cite{homoliak2019security}.}		
		\label{fig:sra-app-exchanges}
	\end{center}	
\end{figure}


\begin{figure}
	%	\vspace{-0.3cm}
	\begin{center}
		\includegraphics[width=0.75\textwidth]{./figs/attacks-APP-oracles.pdf}
		\caption{VTD graph of the oracles category~\cite{homoliak2019security}.}		
		\label{fig:sra-app-oracles}
	\end{center}	
\end{figure}

\begin{figure}
	%	\vspace{-0.3cm}
	\begin{center}
		\includegraphics[width=0.75\textwidth]{./figs/attacks-APP-DFs}
		\caption{VTD graph of the decentralized filesystems category~\cite{homoliak2019security}.}		
		\label{fig:sra-app-DFs}
	\end{center}	
\end{figure}


\begin{figure}
	%	\vspace{-0.3cm}
	\begin{center}
		\includegraphics[width=0.55\textwidth]{./figs/attacks-APP-evoting.pdf}
		\caption{VTD graph of the e-voting category~\cite{homoliak2019security}.}		
		\label{fig:sra-app-evoting}
	\end{center}	
\end{figure}


\begin{figure}
	%	\vspace{-0.3cm}
	\begin{center}
		\includegraphics[width=0.55\textwidth]{./figs/attacks-APP-reputation.pdf}
		\caption{VTD graph of the reputation systems category~\cite{homoliak2019security}.}		
		\label{fig:sra-app-REP}
	\end{center}	
\end{figure}

\begin{figure}
	%	\vspace{-0.3cm}
	\begin{center}
		\includegraphics[width=0.55\textwidth]{./figs/attacks-APP-provenance.pdf}
		\caption{VTD graph of the data provenance category~\cite{homoliak2019security}.}		
		\label{fig:sra-app-provenance}
	\end{center}	
\end{figure}


\begin{figure}
	%	\vspace{-0.3cm}
	\begin{center}
		\includegraphics[width=0.55\textwidth]{./figs/attacks-APP-notaries.pdf}
		\caption{VTD graph of the notaries category~\cite{homoliak2019security}.}		
		\label{fig:sra-app-Notaries}
	\end{center}	
\end{figure}


\begin{figure}
	%	\vspace{-0.3cm}
	\begin{center}
		\includegraphics[width=0.55\textwidth]{./figs/attacks-APP-trading.pdf}
		\caption{VTD graph of the direct trading category~\cite{homoliak2019security}.}		
		\label{fig:sra-app-trading}
	\end{center}	
\end{figure}


\begin{figure}
	%	\vspace{-0.3cm}
	\begin{center}
		\includegraphics[width=0.55\textwidth]{./figs/attacks-APP-escrows.pdf}
		\caption{VTD graph of the escrows category~\cite{homoliak2019security}.}		
		\label{fig:sra-app-escrows}
	\end{center}	
\end{figure}


\begin{figure}
	%	\vspace{-0.3cm}
	\begin{center}
		\includegraphics[width=0.55\textwidth]{./figs/attacks-APP-auctions.pdf}
		\caption{VTD graph of the auctions category~\cite{homoliak2019security}.}		
		\label{fig:sra-app-auctions}
	\end{center}	
\end{figure}



