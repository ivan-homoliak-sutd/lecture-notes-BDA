This section provides explanation of Proof-of-Stake
(PoS) consensus mechanisms, which have emerged as a potential
alternative to the energy-intensive Proof-of-Work (PoW) systems. We will
explore the foundational principles of PoS, where the right to
validate transactions and create new blocks is granted based on the
quantity of cryptocurrency a participant holds and is willing to
``stake'' as collateral. This economic-driven approach to consensus
offers notable advantages in terms of energy efficiency and scalability,
but also introduces a unique set of security challenges and economic
considerations.

The section will begin by discussing the primary motivations for the
development of PoS, focusing on the well-documented energy consumption
issues associated with PoW. We will then delve into the core concepts of
PoS, including the roles of validators, the process of staking, and the
mechanisms for leader election. A balanced analysis of the advantages
and disadvantages of PoS will be presented, addressing its potential for
both decentralization and wealth concentration.
%
A significant portion of the section is dedicated to analysis
of the security vulnerabilities inherent in PoS protocols. We will
dissect critical issues such as the \textbf{``nothing-at-stake'' problem,
grinding attacks, DoS on leaders, and long-range attacks}, and discuss
the various countermeasures that have been proposed and implemented to
mitigate these risks.

We will then survey several prominent PoS protocols that have been
deployed in real-world blockchain systems, including \textbf{Peercoin, Algorand,
Ouroboros, and Ethereum's Casper and Gasper}. 
%
Finally, we will explore
the challenge of blockchain \textbf{oracles} and how systems can securely
access external data, covering solutions like \textbf{TownCrier, TLS-N, and
PDFS}. 
%Through this exploration, readers will gain an understanding of the theoretical underpinnings and practical implementations of Proof-of-Stake and its surrounding ecosystem.

\subsection{Learning Objectives}\label{learning-objectives}

\begin{itemize}
	\tightlist
	\item
	Understand the fundamental motivations for the development of
	Proof-of-Stake (PoS) as an alternative to Proof-of-Work (PoW).
	\item
	Grasp the core concepts of PoS, including staking, validators, leader
	election, and the role of economic incentives in securing the network.
	\item
	Analyze the advantages and disadvantages of PoS, including its energy
	efficiency, potential for scalability, and the risk of stake
	centralization.
	\item
	Identify and understand the security challenges specific to PoS, such
	as the nothing-at-stake problem, grinding attacks, DoS on leaders, and
	long-range attacks.
	\item
	Explore the design and implementation of prominent PoS protocols,
	including Peercoin, Algorand, Ouroboros, and Ethereum's Casper.
	\item
	Evaluate the various countermeasures developed to address the security
	vulnerabilities of PoS systems.
	\item
	Understand the ``oracle problem'' and evaluate different approaches
	for providing external data to smart contracts.
\end{itemize}

\begin{center}\rule{0.5\linewidth}{0.5pt}\end{center}

\subsection{Introduction to	Proof-of-Stake}\label{section-1-introduction-to-proof-of-stake}

\subsubsection{The Motivation}\label{the-motivation-for-proof-of-stake}

The primary impetus for the creation of Proof-of-Stake consensus
mechanisms was the growing concern over the immense energy consumption
of Proof-of-Work (PoW) blockchains -- see example of the Bitcoin in \autoref{fig:btc-energy-consumption} and \autoref{fig:btc-energy-footprint}.
%
%\pandocbounded{\includegraphics[keepaspectratio,alt={PoW Energy Consumption}]{../../../Input/BDA-11-Proof-of-Stake-protocols-24.-4.-2025_files/Image_004.jpg}}


\begin{figure}[t]
	%	\vspace{-0.3cm}
	\begin{center}
	\includegraphics[width=0.9\textwidth]{./figs/bitcoin-energy-consumpti.png}
	\caption{A visual representation of Bitcoin's energy consumption over time~
		\cite{statista_bitcoin_energy_consumption_2025}.}
		\label{fig:btc-energy-consumption}
	\end{center}	
\end{figure}

\begin{figure}[t]
	%	\vspace{-0.3cm}
	\begin{center}
		\includegraphics[width=0.8\textwidth]{./figs/bitcoin-footprints.png}
		\caption{Annualized total Bitcoin footprints~\cite{digiconomist_bitcoin_energy_consumption}.}
		\label{fig:btc-energy-footprint}
	\end{center}	
\end{figure}

%
PoS was conceived as a more energy-efficient alternative, shifting the
basis of consensus from computational power to economic stake. In a PoW
system, miners compete to solve a computationally intensive puzzle. This
process, while effective, requires vast amounts of electricity and
specialized hardware. PoS, in contrast, eliminates this computational
race. Instead, participants, known as validators, lock up a certain
amount of their cryptocurrency as a ``stake'' in the network. This
``virtual mining'' approach dramatically reduces the energy footprint of
the blockchain.

%\pandocbounded{\includegraphics[keepaspectratio,alt={PoR vs.~PoS}]{../../../Input/BDA-11-Proof-of-Stake-protocols-24.-4.-2025_files/Image_008.png}}
%\emph{A diagram comparing the resource requirements of Proof-of-Resource
	%(PoR) and Proof-of-Stake (PoS).}

\subsubsection{Core Concepts of
	Proof-of-Stake}\label{core-concepts-of-proof-of-stake}

The operation of a Proof-of-Stake system is defined by several key
concepts:

\begin{itemize}
	\tightlist
	\item
	\textbf{Goal}: To translate any Proof-of-Resource (PoR) into a
	``Proof-of-Money.'' Instead of buying hardware, miners buy stake and
	vote with it to earn interest.
	\item
	\textbf{Sybil Resistance}: Achieved by requiring a financial stake,
	making it prohibitively expensive to create numerous fake identities.
	\item
	\textbf{Efficiency}: PoS is both energy-friendly and more efficient in
	terms of transaction throughput.
	\item
	\textbf{Participation}: In theory, any token holder can be a
	``stakeholder.'' In practice, this often involves delegated staking
	due to availability and computational resource requirements.
	\item
	\textbf{Decentralization}: PoS offers partially-reduced centralization
	compared to the ASIC-dominated landscape of PoW.
	\item
	\textbf{Low Operational Costs}: The low operational costs mean
	validators can wait longer for rewards.
\end{itemize}

\subsubsection{Advantages and Disadvantages of
	PoS}\label{advantages-and-disadvantages-of-pos}

\textbf{Advantages:}

\begin{itemize}
	\tightlist
	\item
	\textbf{Energy Efficiency}: PoS is orders of magnitude more
	energy-efficient than PoW.
	\item
	\textbf{Reduced Barriers to Entry}: PoS does not require investment in
	specialized, high-cost mining hardware.
	\item
	\textbf{Higher Throughput}: Faster block creation times might lead to
	higher transaction throughput.
	\item
	\textbf{Economic Security}: Aligns validators' incentives with the
	long-term health of the blockchain.
\end{itemize}

\textbf{Disadvantages:}

\begin{itemize}
	\tightlist
	\item
	\textbf{Stake Centralization}: The ``rich get richer'' problem, where
	wealth can become concentrated among the largest stakers. A 51\% miner
	can potentially control the chain forever, as no new powerful miner
	can emerge as easily as in PoW.
	\item
	\textbf{Semi-Permissionless Nature}: One must acquire stake to
	participate, which typically means buying it from someone, making the
	system not fully permissionless.
	\item
	\textbf{The Nothing-at-Stake Problem}: A validator has an economic
	incentive to validate blocks on all competing chains in a fork, as
	there is no additional cost.
	\item
	\textbf{Long-Range Attacks}: An adversary can acquire old private keys
	to rewrite the blockchain's history.
\end{itemize}

\begin{center}\rule{0.5\linewidth}{0.5pt}\end{center}

\subsection{Security Challenges in
	Proof-of-Stake}\label{section-2-security-challenges-in-proof-of-stake}

\subsubsection{The Nothing-at-Stake
	Problem}\label{the-nothing-at-stake-problem}

The nothing-at-stake problem is a fundamental challenge where a
validator, in the event of a fork, can extend two or more conflicting
blocks without risking their stake, thereby increasing their chance of
being rewarded. This increases the number of forks and the time to
finality.

\textbf{Countermeasures:}

\begin{itemize}
	\tightlist
	\item
	\textbf{Deposit-Based Solutions}: Require nodes to make a deposit that
	is lost in the case of misbehavior (slashing).
	\item
	\textbf{Checkpoints}: Employ ``state freezing'' at periodic snapshots
	of the blockchain, making it irreversible up to the most recent
	checkpoint.
	\item
	\textbf{Backward Penalization}: Penalizing nodes that produced two
	conflicting blocks.
	\item
	\textbf{BFT Integration}: Combining PoS protocols with Byzantine Fault
	Tolerance approaches to decrease the probability of forks.
\end{itemize}

\subsubsection{Grinding Attacks}\label{grinding-attacks}

A grinding attack occurs when a leader, knowing they will produce the
next block, can bias the process to increase their chances of being
selected in the future. For example, if the next leader is determined
solely by the hash of the previous block, the current leader can
``grind'' through different block configurations to find one that
results in a favorable outcome for them.

\textbf{Countermeasures:}

\begin{itemize}
	\tightlist
	\item
	\textbf{Secure Multiparty Computation (SMPC)}: Using a committee of
	consensus nodes to generate a fresh random number for leader election
	(e.g., a secure coin-flipping protocol).
	\item
	\textbf{Verifiable Random Functions (VRFs)}: Allowing a node to
	privately check if its VRF output is below a certain stake-specific
	threshold. The VRF input combines the user's private key and
	randomness from the previous block, ensuring the current leader cannot
	bias the outcome.
\end{itemize}

\subsubsection{Denial of Service (DoS) on a
	Leader}\label{denial-of-service-dos-on-a-leader}

If a leader is publicly determined before their turn to produce a block,
an adversary can launch a DoS attack against them, forcing a round
restart. This can be repeated until the adversary's desired nodes are
elected.

\medskip
\textbf{Countermeasures:}
\begin{itemize}
	\tightlist
	\item \textbf{Private Leader Election (via VRF)}: As pioneered by Algorand~\cite{gilad2017algorand},
	a node privately determines if it is a potential leader and
	immediately releases a block candidate. By the time the leader is
	publicly known, it is too late for a DoS attack to be effective.
	
	\item \textbf{Whisk}: Designed for Ethereum's PoS. It conceals block proposers' identities until their assigned slot by shuffling a candidate list using verifiable random permutations and ZKPs. 
	Operating in a pipelined mode (shown in \autoref{fig:whisk-tereza}), current-round proposers prepare the shuffle for the next round, enabling scalability while preserving pre-slot anonymity.
	
		
	\item \textbf{Homomorphic sortition}: is a cryptographic mechanism that leverages threshold fully homomorphic encryption (ThFHE) to perform proposer selection over encrypted data~\cite{freitas2022homomorphic}. This approach ensures that proposers cannot be identified until a joint decryption is completed (see \autoref{fig:hs-tereza}).
	
	\item \textbf{Network-Level Deanonymization} --
	can be leveraged to enhance validator anonymity. 
	Dandelion++ and RLN~\cite{dandelion} aimed to obfuscate the origin of consensus messages by routing them through a private sub-network. 
	%While conceptually promising, the final analysis concluded that the approach is not feasible for Ethereum’s consensus layer due to strict latency constraints and high implementation complexity, especially under the tightened timing rules. 
	%The trade-offs ultimately limited its applicability despite its strong anonymity model. 
	%
	Polkadot's Sassafras~\cite{sassafras}  combines an SSLE-based leader election with network-layer anonymity. It complements consensus-layer protection with mechanisms to conceal proposer network traffic, offering a more comprehensive defense against deanonymization and targeted attacks.
	%
	CoPoR-PoS~\cite{homoliak2025pos} proposed native embedding of the anonymization layer into the consensus protocol itself.
	
\end{itemize}


\begin{figure}[t]
	%	\vspace{-0.3cm}
	\begin{center}
		\includegraphics[width=0.8\textwidth]{./figs/whisk-tereza.png}
		\caption{The pipeline of the Whisk proposer protection mechanism~\cite{burianova2025secret}.}
		\label{fig:whisk-tereza}
	\end{center}	
\end{figure}

\begin{figure}[t]
	%	\vspace{-0.3cm}
	\begin{center}
		\includegraphics[width=0.8\textwidth]{./figs/hs-tereza.png}
		\caption{Homomorphic sortition protocol~\cite{burianova2025secret}.}
		\label{fig:hs-tereza}
	\end{center}	
\end{figure}



\subsubsection{Long-Range Attacks}\label{long-range-attacks}

Also known as posterior corruption, this attack involves an adversary
bribing or stealing the private keys of previously influential consensus
nodes. Since these nodes may have already exchanged their tokens for
fiat currency, they have no stake left to lose and no incentive to
protect their old keys. If the attacker accumulates enough historical
stake, they can rerun the protocol and rewrite the entire history of the
blockchain.


\medskip
\textbf{Countermeasures~\cite{homoliak2020security}:}
\begin{itemize}
	\tightlist
	\item
	\textbf{Extended Deposit Locking}: Locking the stake deposit for a
	much longer time than the period of participation.
	\item
	\textbf{Frequent Checkpoints}: Making the chain irreversible with
	respect to the last checkpoint. Ethereum PoS uses justified and
	finalized checkpoints for this purpose.
	\item
	\textbf{Key-Evolving Cryptography}: Requiring users to evolve their
	private keys and erase old ones. While this prevents forging
	signatures, it doesn't stop dishonest nodes from voluntarily selling
	their old keys.
	\item
	\textbf{Time-Domain Chain Density}: Enforcing rules about the expected
	number of participants in each round.
	\item
	\textbf{Context-Sensitive Transactions}: Including the hash of a
	recent valid block within a transaction itself, tethering it to a
	specific point in history.

	
	
	
\end{itemize}

\begin{center}\rule{0.5\linewidth}{0.5pt}\end{center}

\subsection{Prominent Proof-of-Stake
	Protocols}\label{section-3-prominent-proof-of-stake-protocols}


\subsubsection{Peercoin}\label{peercoin}

Launched in 2012, Peercoin is the first technical realization of PoS,
using a hybrid PoW/PoS model. Its PoS component is based on
\textbf{coin-age}, calculated as
\texttt{(amount\ of\ UTXO)\ ×\ (\#\ of\ blocks\ it\ remains\ unspent)}.
Miners can balance PoW and coin-age, but it is much easier to find a
solution by consuming some coin-age.

\subsubsection{Algorand}\label{algorand}

Algorand is a pure PoS protocol with no punishments or locked tokens. It
uses cryptographic sortition to choose leaders via a Verifiable Random
Function (VRF) to avoid DoS on the leader attack (see \autoref{denial-of-service-dos-on-a-leader}). A leader is selected privately and proposes a new block.
Since there can exist multiple leaders meeting the threshold for VRF, Agorand requires additional round to vote on the best one.\footnote{Therefore, Algorand belongs to the family of Non-Single Secret Leader Election (NSSLE) protocols.}
For this purpose a committee of \textasciitilde1000 nodes, also selected by VRF, then
votes on the proposed block to ensure Byzantine tolerance.

\subsubsection{Ouroboros}\label{ouroboros}

Ouroboros~\cite{kiayias2017ouroboros} is a family of provably secure PoS protocols developed for the
Cardano blockchain.

\begin{itemize}
	\tightlist
	\item
	\textbf{Ouroboros Classic}: This version uses a Publicly Verifiable
	Secret Sharing (PVSS) scheme for randomness generation. Slot leaders
	are publicly known in advance, making them vulnerable to DoS attacks.
	The randomness generation is a detailed, five-step process:
	
	\begin{enumerate}
		\def\labelenumi{\arabic{enumi}.}
		\tightlist
		\item
		\textbf{Committee Formation}: Slot leaders form a committee and
		privately generate random numbers.
		\item
		\textbf{Commit}: Members post their PVSS data (a commitment and
		encrypted shares for every other member).
		\item
		\textbf{Reveal}: Members reveal their random numbers.
		\item
		\textbf{Recovery}: For any member who did not reveal, other members
		post their shares to reconstruct the secret.
		\item
		\textbf{New Epoch}: The revealed numbers are XORed together to
		create a seed for picking leaders in the next epoch.
	\end{enumerate}
	\item
	\textbf{Ouroboros Praos}: Inspired by Algorand, this version uses
	VRFs. Each node knows privately which slots they will lead, mitigating
	DoS attacks. There can be multiple or no leaders for a given slot.
\end{itemize}

\subsubsection{Ethereum's PoS (Casper \&
	Gasper)}\label{ethereums-pos-casper-gasper}

Ethereum's transition to PoS utilizes a combination of protocols.



\begin{figure}[t]
	%	\vspace{-0.3cm}
	\begin{center}
		\includegraphics[width=0.7\textwidth]{./figs/casper1.png}
		\caption{Finalization in Casper FFG~\cite{buterin2017casper}. The setup assumes 3 validators holding the majority of the stake.}
		\label{fig:casper-forks}
	\end{center}	
\end{figure}


\begin{figure}[t]
	%	\vspace{-0.3cm}
	\begin{center}
		\includegraphics[width=0.9\textwidth]{./figs/casper2.png}
		\caption{Forks and finalization in Casper FFG~\cite{buterin2017casper}. The branch with the majority of validators (the upper one) will resume finalizing checkpoints first. The setup assumes}
		\label{fig:casper-forks2}
	\end{center}	
\end{figure}


\begin{itemize}
	\tightlist
	\item
	\textbf{Casper the Friendly Finality Gadget (FFG)}: Casper FFG~\cite{buterin2017casper} is not
	a standalone consensus protocol but a ``finality gadget'' that
	overlays a block proposal mechanism. It introduces checkpoints and
	attestations (votes).
	
	\begin{itemize}
		\tightlist
		\item
		\textbf{Justification}: A checkpoint B is justified if a previously
		justified checkpoint A exists and there are attestations for the A
		-\textgreater{} B edge with a total weight of at least 2/3 of the
		total stake.
		\item
		\textbf{Finalization}: A checkpoint A is finalized if it is
		justified and its immediate successor is also justified. Finalized
		blocks are considered permanent (see \autoref{fig:casper-forks} and \autoref{fig:casper-forks2}).
		%    \pandocbounded{\includegraphics[keepaspectratio,alt={Casper FFG Finalization}]{../../../Input/BDA-11-Proof-of-Stake-protocols-24.-4.-2025_files/Image_015.png}}
		%\emph{A diagram illustrating the finalization process in CasperFFG.}
		\item
		\textbf{Slashing Conditions}: Validators are slashed for breaking
		rules, such as:
		
		\begin{figure}[b]
			%	\vspace{-0.3cm}
			\begin{center}
				\includegraphics[width=0.4\textwidth]{./figs/ghost.png}
				\caption{The GHOST fork-choice rule.}
				\label{fig:ghost}
			\end{center}	
		\end{figure}
		
		
		\begin{itemize}
			\tightlist
			\item
			\textbf{S1}: Making two distinct attestations for checkpoints at
			the same height (i.e., voting for a fork).
			\item
			\textbf{S2}: Making an attestation that ``surrounds'' another
			(e.g., voting for an edge s1 -\textgreater{} t1 and another s2
			-\textgreater{} t2 where
			\texttt{height(s1)\ \textless{}\ height(s2)\ \textless{}\ height(t2)\ \textless{}\ height(t1)}).
		\end{itemize}
	\end{itemize}
	\item
	\textbf{Gasper}: The full protocol used in Ethereum 2.0, which
	combines the Casper FFG finality gadget with the GHOST (Greediest
	Heaviest Observed SubTree) fork-choice rule (see \autoref{fig:ghost}).
	%  \pandocbounded{\includegraphics[keepaspectratio,alt={Gasper Protocol}]{../../../Input/BDA-11-Proof-of-Stake-protocols-24.-4.-2025_files/Image_016.jpg}}
\end{itemize}





\begin{center}\rule{0.5\linewidth}{0.5pt}\end{center}

\subsection{Oracles and Data
	Feeds}\label{section-4-oracles-and-data-feeds}

A blockchain (and its smart contracts) is an isolated environment and by default has no access to data in external world. The ``oracle problem'' is the
challenge of providing smart contracts with reliable data from the
external world.
There are several approaches to oracle that we will briefly describe.

\subsubsection{TownCrier (TC)}\label{towncrier-tc}

TownCrier~\cite{zhang2016town} acts as a trusted proxy between a blockchain and HTTPS
websites, using a Trusted Execution Environment (TEE) like Intel SGX (see \autoref{fig:oracles-tc}).
\vspace{-0.3cm}
\begin{itemize}
	\tightlist
	\item \textbf{Pros}: Easy integration, no changes needed for website
	operators. 
	
	\item \textbf{Cons}: Trusts the TEE manufacturer (a single point
	of failure), and several attacks have been demonstrated against Intel
	SGX.	
\end{itemize}


\begin{figure}[t]
	%	\vspace{-0.3cm}
	\begin{center}
		\includegraphics[width=0.8\textwidth]{./figs/oracle-problem.png}
		\caption{Oracle problem.}
		\label{fig:oracles}
	\end{center}	
\end{figure}


\begin{figure}[b]
	%	\vspace{-0.3cm}
	\begin{center}
		\includegraphics[width=0.85\textwidth]{./figs/tc1.png}
		\caption{TownCrier~\cite{zhang2016town}.}
		\label{fig:oracles-tc}
	\end{center}	
\end{figure}

\subsubsection{TLS-N}\label{tls-n}

TLS-N~\cite{ritzdorf2018tls} adds a non-repudiation layer to TLS, allowing for the generation
of privacy-preserving, non-interactive proofs of the contents of a TLS
session (see overview in \autoref{fig:tlsn}). 
It uses efficient Merkle-tree-based authentication and produces TLS-N proofs that can be verified by smart contracts.
\begin{itemize}
	\tightlist
	
	\item \textbf{Pros}: More general and powerful, with extra features
	like privacy. 
	\item \textbf{Cons}: Requires changes to the TLS specification
	and updates to TLS servers, making integration difficult and expensive.
	
\end{itemize}

%\pandocbounded{\includegraphics[keepaspectratio,alt={TLS-N Diagram}]{../../../Input/BDA-11-Proof-of-Stake-protocols-24.-4.-2025_files/Image_020.jpg}}

\subsubsection{Practical Data Feed Service
	(PDFS)}\label{provable-data-feeds-pdfs}

In PDFS~\cite{guarnizo2019pdfs}, data providers maintain an append-only database (like a
Merkle tree) and sign their contract locations via TLS (see \autoref{fig:pdfs}). The contract API
allows for updating the database root and proving the authenticity of
data via Merkle proofs. 
\vspace{-0.3cm}
\begin{itemize}
	\tightlist
	\item \textbf{Pros}: Direct TLS authentication, easy
	integration. 
	\item \textbf{Cons}: Website operators have to deploy it.
	
\end{itemize}
%\pandocbounded{\includegraphics[keepaspectratio,alt={PDFS Diagram}]{../../../Input/BDA-11-Proof-of-Stake-protocols-24.-4.-2025_files/Image_021.jpg}}


\begin{figure}[t]
	%	\vspace{-0.3cm}
	\begin{center}
		\includegraphics[width=0.7\textwidth]{./figs/TLSN.png}
		\caption{TLS-N overview~\cite{ritzdorf2018tls}.}
		\label{fig:tlsn}
	\end{center}	
\end{figure}


\begin{figure}[t]
	%	\vspace{-0.3cm}
	\begin{center}
		\includegraphics[width=0.8\textwidth]{./figs/pdfs.png}
		\caption{PDFS overview~\cite{guarnizo2018pdfs}.}
		\label{fig:pdfs}
	\end{center}	
\end{figure}

\begin{center}\rule{0.5\linewidth}{0.5pt}\end{center}

\subsection{Summary / Key Takeaways}\label{summary-key-takeaways}

This section has provided a brief exploration of Proof-of-Stake
consensus protocols. We established the motivation for PoS as a more
energy-efficient alternative to PoW, defined its core concepts, and
presented a balanced view of its advantages and disadvantages. A key
focus was the unique security landscape of PoS, including the
nothing-at-stake problem, grinding attacks, and long-range attacks,
along with their countermeasures. We surveyed influential PoS protocols
like Peercoin, Algorand, Ouroboros, and Ethereum's Casper. Finally, we
addressed the critical oracle problem, examining how external data can
be securely fed to blockchains.

\begin{center}\rule{0.5\linewidth}{0.5pt}\end{center}

\subsection{Keywords}\label{keywords}

\begin{itemize}
	\tightlist
	\item
	\textbf{Proof-of-Stake (PoS)}: A class of consensus mechanisms that
	select block creators based on the number of coins they hold and are
	willing to ``stake.''
	\item
	\textbf{Validator}: A participant in a PoS network responsible for
	validating transactions and creating new blocks.
	\item
	\textbf{Slashing}: A penalty mechanism where a validator who acts
	maliciously has their staked funds confiscated.
	\item
	\textbf{Nothing-at-Stake Problem}: A scenario where a validator has no
	economic disincentive to validate blocks on all competing chains in a
	fork.
	\item
	\textbf{Long-Range Attack}: An attack where an adversary uses old,
	compromised private keys to create an alternative history of the
	chain.
	\item
	\textbf{Verifiable Random Function (VRF)}: A cryptographic function
	used for secure and private leader election.
	\item
	\textbf{Finality}: The guarantee that a transaction or block is
	irreversible.
	\item
	\textbf{Oracle}: A service that provides external data to a smart
	contract.
	\item
	\textbf{Trusted Execution Environment (TEE)}: A secure area inside a
	main processor, used by systems like TownCrier.
\end{itemize}

\begin{center}\rule{0.5\linewidth}{0.5pt}\end{center}

\subsection{Further Reading}\label{further-reading}

\begin{itemize}
	\tightlist
	\item \textbf{Energy consumption of Bitcoin} \\
	\url{https://digiconomist.net/bitcoin-energy-consumption}
	\item \textbf{PeerCoin} \\
	\url{https://peercoin.net/assets/paper/peercoin-paper.pdf}
	\item \textbf{Proof-of-Stake in Ethereum} \\
	\url{https://ethereum.org/en/developers/docs/consensus-mechanisms/pos/}
	\item \textbf{Gasper} \\
	\url{https://arxiv.org/pdf/1903.04205}
	\item \textbf{Ethereum 2.0} \\
	\url{https://arxiv.org/abs/2003.03052} 
	\item \textbf{Ouroboros} \\
	\url{https://www.youtube.com/watch?v=hMgxZOsTlQc} 
	\item \textbf{TownCrier} \\
	\url{https://eprint.iacr.org/2016/168.pdf} 
	\item \textbf{Ouroboros Praos} \\
	\url{https://eprint.iacr.org/2017/578.pdf} 
	\item \textbf{PDFS} \\
	\url{https://arxiv.org/pdf/1808.06641.pdf} 
\end{itemize}
