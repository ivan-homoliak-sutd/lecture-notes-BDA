This section delves into the landscape of alternative consensus
mechanisms designed to address the limitations of traditional
Proof-of-Work (PoW), such as Bitcoin that we dealt with in the previous sections. While PoW, the backbone of Bitcoin, is lauded for
its security, it faces significant criticism for its immense energy
consumption, the centralization of mining power through
Application-Specific Integrated Circuits (ASICs), and the inherently
``useless'' nature of its computations.

Here, we will explore the broader category of \textbf{Proof-of-Resource (PoR)} protocols (also encompassing PoW), which leverage alternative resources like memory and storage
to achieve consensus. The central goal is to design mining processes
that are more equitable and energy-efficient. % and resistant to the centralization that  PoW. 
We will examine memory-hard functions
like \textbf{scrypt} and \textbf{Argon2}, investigate the concept of Proof-of-Useful-Work
through projects like \textbf{PrimeCoin}, and dissect the various forms of
\textbf{Proof-of-Storage}, including \textbf{Proof-of-Retrievability} and
\textbf{Proof-of-Replication}, as seen in protocols like Permacoin.

\subsection{Learning Objectives}\label{learning-objectives}

\begin{itemize}
	\tightlist
	\item
	Understand the fundamental drawbacks of Proof-of-Work (PoW) and the
	primary motivations for developing alternative consensus protocols.
	\item
	Identify and explain the essential requirements for a robust
	Proof-of-Resource (PoR) protocol, including fast verification,
	adjustable difficulty, fairness, and memorylessness.
	\item
	Analyze the concept of ASIC resistance and its critical role in
	fostering decentralization.
	\item
	Differentiate between memory-hard and memory-bound mining and
	understand their application in algorithms like scrypt and Argon2.
	\item
	Explore the potential of Proof-of-Useful-Work to generate value beyond
	network security.
	\item
	Distinguish between different Proof-of-Storage models, such as
	Proof-of-Retrievability, Proof-of-Replication, Proof-of-Space, and
	Proof-of-Spacetime.
	\item
	Gain familiarity with influential altcoins that pioneered these
	alternative consensus mechanisms.
\end{itemize}

\begin{center}\rule{0.5\linewidth}{0.5pt}\end{center}

\subsection{PoW vs. PoR}\label{section-1-the-case-against-proof-of-work}

\subsubsection{Limitations of PoW}\label{limitations-of-pow}

The Proof-of-Work model, while foundational to blockchain technology,
presents several significant challenges:

\begin{itemize}
	\tightlist
	\item
	\textbf{Extreme Energy Consumption}: The computational arms race in
	PoW mining has led to an energy footprint comparable to that of entire
	countries. 
	\ih{Put example from PoUW paper.}
	\item
	\textbf{Centralization via ASICs}: The efficiency of ASICs has
	rendered mining with commodity hardware (like CPUs and GPUs)
	unprofitable, concentrating power in the hands of a few large-scale
	mining operators / centralized mining pools. This undermines the core principle of decentralization.
	\item
	\textbf{Wasted Computational Effort}: The hashing computations in PoW
	are performed solely to secure the network and have no external value,
	representing a massive expenditure of resources on problems with no
	intrinsic utility.
\end{itemize}

These issues have spurred a wave of innovation aimed at creating more
sustainable, decentralized, and efficient consensus mechanisms.

\subsubsection{Core Requirements for Proof-of-Resource (PoR)
	Protocols}\label{core-requirements-for-proof-of-resource-por-protocols}

Any viable alternative to PoW must satisfy a set of core principles to
ensure the integrity and functionality of the blockchain:

\begin{itemize}
	\tightlist
	\item
	\textbf{Fast Verification}: While finding a solution may be difficult,
	verifying its correctness must be computationally trivial for all
	nodes.
	\item
	\textbf{Adjustable Difficulty}: The protocol must be able to
	dynamically adjust the difficulty of the puzzle to maintain a
	consistent block time, regardless of fluctuations in total network
	resource commitment.
	\item
	\textbf{Fairness}: The probability of successfully mining a block
	should be directly proportional to the amount of the specified
	resource a miner contributes.
	\item
	\textbf{Independence of Solutions}: Finding one solution should not
	provide any information or advantage in finding subsequent solutions.
	\item
	\textbf{Memorylessness}: The time it takes to find a solution should
	follow an exponential distribution, meaning the probability of finding
	a solution is independent of the time already spent searching. This
	ensures that even miners with fewer resources have a chance to
	succeed.
\end{itemize}

\begin{center}\rule{0.5\linewidth}{0.5pt}\end{center}

\subsection{Memory-Hard Mining and ASIC
	Resistance}\label{section-2-memory-hard-mining-and-asic-resistance}

A primary objective in designing alternative PoR protocols is achieving
\textbf{ASIC resistance}. The idea is to create a puzzle that is
difficult and expensive to implement in specialized hardware, thereby
allowing commodity hardware like CPUs and GPUs to remain competitive.
This fosters greater decentralization by lowering the barrier to entry
for miners.

\subsubsection{Memory-Hard and Memory-Bound
	Functions}\label{memory-hard-and-memory-bound-functions}

Two key concepts in achieving ASIC resistance are:

\begin{itemize}
	\tightlist
	\item
	\textbf{Memory-Hard Mining}: This approach makes the mining process
	dependent on having a large amount of memory (RAM). The difficulty is
	tied to the \emph{quantity} of memory available.
	\item
	\textbf{Memory-Bound Mining}: This focuses on the \emph{speed} of
	memory access. The mining algorithm is designed such that performance
	is limited by how quickly data can be read from and written to memory.
\end{itemize}

Combining these two concepts makes building cost-effective ASICs
significantly more challenging, as high-capacity, high-speed memory is
expensive to integrate into custom chips.


\begin{figure}[t]
	\centering
	\includegraphics[width=0.7\textwidth]{./figs/scrypt.png}
	\caption{The scrypt algorithm, illustrating the time-memory trade-off.}\label{fig:scrypt}
\end{figure}

\subsubsection{Scrypt}\label{scrypt}

Scrypt \ih{todo citation} is a \textbf{password-based key derivation function (PBKDF)} designed in
2009 and later adopted as an IETF standard (RFC 7914). Unlike other
PBKDFs, it was explicitly designed to have high memory requirements to
defend against brute-force attacks.

The algorithm generates a large, pseudo-random vector of data that must
be held in RAM for efficient computation (see \autoref{fig:scrypt}). Accessing elements of this
vector in a pseudo-random order is core to the algorithm. This creates a
\textbf{time-memory trade-off}:

\begin{itemize}
	\tightlist
	\item
	\textbf{With sufficient RAM}: The algorithm has a linear time
	complexity, O(N).
	\item
	\textbf{Without sufficient RAM}: The elements must be re-generated
	on-the-fly, leading to a quadratic time complexity, O(N²), making the
	process prohibitively slow.
\end{itemize}



While adopted by cryptocurrencies like Litecoin and Dogecoin to resist
ASICs, specialized scrypt ASICs were eventually developed, demonstrating
the persistent difficulty of achieving long-term ASIC resistance.

\subsubsection{Argon2}\label{argon2}

Argon2 \ih{citation}, the winner of the 2015 Password Hashing Competition, is a more
modern and robust memory-hard function. It is optimized for modern CPU
architectures and their cache/memory systems.

Based on the password $P$ and salt $S$, Argon2 fills a large memory array with compression function $G$:
\begin{eqnarray}
	B[0] &=& H(P, S) \\
	for ~~~j~ &=&~ [1,\ldots, t] \\
	&&B[j] = G(B[\phi_1(j)], B[\phi_2(j)], \ldots, B[\phi_k(j	)]),
\end{eqnarray}
where $\phi$ represents indexing function.
Argon2 and has two main variants, based on the indexing function $\phi$:
\begin{itemize}
	\tightlist
	\item
	\textbf{Argon2d}: Uses \textbf{data-dependent} memory access. It's
	faster but vulnerable to side-channel attacks, making it suitable for
	applications like cryptocurrencies where the ``password'' (the block
	header and nonce) is public.
	\item
	\textbf{Argon2i}: Uses \textbf{data-independent} memory access. It's
	slower but resistant to side-channel attacks, making it ideal for
	traditional password hashing and password-based key derivation functions.
\end{itemize}
See the sequential and parallel mode of operation of Argon2 in \autoref{fig:argon2}.


\begin{figure}
	\centering
	\includegraphics[width=0.9\textwidth]{./figs/argon2.png}
	\caption{The parallel and sequential modes of the Argon2 algorithm.}\label{fig:argon2}
\end{figure}

\begin{center}\rule{0.5\linewidth}{0.5pt}\end{center}

\subsection{Proof-of-Useful-Work}\label{section-3-proof-of-useful-work}

The concept of Proof-of-Useful-Work (PoUW) aims to harness the immense
computational power of a blockchain network for tasks that have
intrinsic value beyond securing the ledger. Instead of solving arbitrary
puzzles, miners contribute to solving real-world scientific or
mathematical problems.

A notable example is \textbf{PrimeCoin}\ih{citation}, which uses its mining process
to search for special sequences of prime numbers known as
\textbf{Cunningham chains}. While these chains have applications in
number theory, a significant challenge in PoUW is integrating a secure
and fair authentication mechanism to prove \emph{who} found the
solution, a step that is not inherent in many scientific computations, which keeps this proposal flawed.

In particular, the algorithm of mining requires three parameters $m, n, k$.
For the previous block hash $x$, take $m$ bits of $x$ and consider any $k$-long chain in which the first prime in the chain is an $n$-bit prime with $m$ leading bits (i.e., the same prefix as $x$).

Found several world records that are depicted in \autoref{fig:primecoin}.

\begin{figure}
	\centering
	\includegraphics[width=0.4\textwidth]{./figs/primecoin.png}
	\caption{World records of Cunningham chains found by Primecoin.}\label{fig:primecoin}
\end{figure}


\begin{center}\rule{0.5\linewidth}{0.5pt}\end{center}

\subsection{Proof-of-Storage}\label{section-4-proof-of-storage}

Proof-of-Storage (PoS) protocols shift the resource requirement from
computation to data storage. Miners must prove they are dedicating
storage space to the network.

\subsubsection{Key Concepts in
	Proof-of-Storage}\label{key-concepts-in-proof-of-storage}

\begin{itemize}
	\tightlist
	\item
	\textbf{Proof-of-Retrievability (PoRet)}: A prover demonstrates to a
	verifier that they possess a specific file and can correctly retrieve
	it. The proofs often leak small pieces of the data, allowing for
	eventual reconstruction.
	\item
	\textbf{Proof-of-Replication (PoRep)}: A more robust form that ensures
	a prover is storing a \emph{unique physical copy} of the data,
	preventing them from pretending to store multiple copies while only
	holding one (a form of Sybil attack).
	\item
	\textbf{Proof-of-Space (PoSpace)}: A prover convinces a verifier that
	they have allocated a certain amount of storage, where the data itself
	does not need to be useful.
	\item
	\textbf{Proof-of-Spacetime}: An extension where the prover shows they
	have used a certain amount of storage \emph{over a specific duration}.
\end{itemize}

\subsubsection{Permacoin: A Proof-of-Retrievability
	System}\label{permacoin-a-proof-of-retrievability-system}

\begin{figure}[t]
	\centering
	\includegraphics[width=0.99\textwidth]{./figs/permacoin1.png}
	\caption{A simplified ``strawman'' approach to Permacoin, which is
		vulnerable to outsourcing.}\label{fig:permacoin1}	
\end{figure}


\begin{figure}[b]
	\centering
	\includegraphics[width=0.8\textwidth]{./figs/permacoin2.png}
	\caption{The improved Permacoin lottery, which ties the puzzle to the miner's private.}\label{fig:permacoin2}	
\end{figure}

Permacoin \ih{citation} is a protocol designed to repurpose mining efforts for
long-term, distributed data archiving. The core idea is that miners
collectively store a large, valuable dataset (e.g., a web archive).

To prevent outsourcing (where a miner fetches data from another source
on-demand) and ensure fair play, Permacoin's design ties the puzzle
solution to a miner's private key and requires sequential, pseudo-random
access to the stored data. This makes it computationally expensive and
slow to generate proofs without possessing the data locally.

Differences of PermaCoin in contrast to simplified PoR are as follows:
\begin{compactitem}
	\item Verifier $V$ is entire Permacoin network.
	\item  Challenge $c[r1, ..., rk]$ generated non-interactively.
	\item Every node can act as prover $P$.
	\item $F$ is large, so miners store portions of $F$ in a distributed manner.
	\item If $F$ is useful then miners’ equipment is useful.
\end{compactitem}

\medskip
The strawman approach of Permacoin does not resolve outsourcing (i.e., no erasure encoding) and it is depicted in \autoref{fig:permacoin1}.
The improvement of PermaCoin that incorporates erasure coding to boost
data recoverability while binding private key to the puzzle solution is depicted in \autoref{fig:permacoin2}.
Such a binding would require sharing of the private key with the outsourcing party, and thus giving a full control of the earned rewards to such a party, rendering outsourcing attacks economically infeasible.






%\begin{figure}
%\centering
%\pandocbounded{\includegraphics[keepaspectratio,alt={The improved Permacoin lottery, which ties the puzzle to the miner's private key and local data.}]{../../../Input/BDA-10-Proof-of-Resource protocols-17.-4.-2025_files/Image_024.png}}
%\caption{The improved Permacoin lottery, which ties the puzzle to the
	%miner's private key and local data.}
%\end{figure}

%\begin{center}\rule{0.5\linewidth}{0.5pt}\end{center}

\begin{center}\rule{0.5\linewidth}{0.5pt}\end{center}

\subsection{Summary / Key Takeaways}\label{summary-key-takeaways}

This section provided an overview of the motivations,
designs, and challenges of Proof-of-Resource protocols. We moved from
the limitations of PoW to the requirements for viable alternatives,
focusing on the critical goal of ASIC resistance. We analyzed
memory-hard functions like scrypt and Argon2, which leverage RAM to
level the playing field for miners. We also explored the innovative
concepts of Proof-of-Useful-Work and the various forms of
Proof-of-Storage, which aim to make the resources spent on consensus
more productive. The evolution of these protocols, as seen in pioneering
altcoins, highlights the ongoing quest for a consensus mechanism that is
secure, decentralized, and sustainable.

\begin{center}\rule{0.5\linewidth}{0.5pt}\end{center}

\subsection{Keywords}\label{keywords}

\begin{itemize}
	\tightlist
	\item
	\textbf{Proof-of-Resource (PoR)}: A class of consensus algorithms
	where miners dedicate resources other than computation (e.g., memory,
	storage) to secure the network.
	\item
	\textbf{ASIC Resistance}: A design goal for mining algorithms to
	prevent them from being efficiently implemented on specialized
	hardware, thus promoting decentralization.
	\item
	\textbf{Memory-Hard Function}: A function whose computational cost is
	dominated by the amount of memory required, not CPU speed.
	\item
	\textbf{Time-Memory Trade-off}: A computational scenario where an
	algorithm's execution time can be reduced at the cost of increased
	memory usage, and vice-versa.
	\item
	\textbf{Proof-of-Useful-Work (PoUW)}: A protocol where the
	computational work performed by miners has intrinsic value beyond
	securing the blockchain.
	\item
	\textbf{Proof-of-Storage (PoS)}: A protocol where miners prove they
	are dedicating a certain amount of digital storage.
	\item
	\textbf{Erasure Coding}: A method of data protection that transforms
	data into a longer form, allowing the original data to be recovered
	even if parts of the longer form are lost.
\end{itemize}

\begin{center}\rule{0.5\linewidth}{0.5pt}\end{center}

\subsection{Further Reading}\label{further-reading}

\begin{itemize}
	\tightlist
	\item
	\textbf{scrypt: A new key derivation function}: \\
	\url{https://www.tarsnap.com/scrypt/scrypt.pdf}
	\item
	\textbf{Argon2: the memory-hard function for password hashing and
		other applications}:\\
	\url{https://github.com/P-H-C/phc-winner-argon2/blob/master/argon2-specs.pdf}
	\item
	\textbf{Permacoin: Repurposing Bitcoin Work for Long-Term Data
		Preservation}: \\
	\url{https://www.cs.umd.edu/\textasciitilde amiller/permacoin.pdf}
\end{itemize}
